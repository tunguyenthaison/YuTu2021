\documentclass[11pt,reqno]{amsart}
%============%============%============%============%

%============%============%============%============%
%\setlength{\columnseprule}{0.4pt}
%\setlength{\topmargin}{0cm}
%\setlength{\oddsidemargin}{.25cm}
%\setlength{\evensidemargin}{.25cm}
%\setlength{\textheight}{22.5cm}
%\setlength{\textwidth}{15.5cm}
\renewcommand{\baselinestretch}{1.05}
%============%============%============%============%
\usepackage[toc,page]{appendix}
%============%============%============%============%
%\usepackage{romannum}
\usepackage{xcolor}
\usepackage{placeins}
\usepackage{amsfonts,amsmath,amsthm}
\usepackage{amssymb,epsfig}
\usepackage{enumerate} 
\usepackage[notcite,notref]{showkeys}
\usepackage{fullpage}
%============%============%============%============%
\usepackage[utf8]{inputenc}
%\usepackage{mathpazo}

%\usepackage[libertine,cmintegrals,cmbraceFunction Analysis - Haim brezis.pdfFunction Analysis - Haim brezis.pdfFunction Analysis - Haim brezis.pdfFunction Analysis - Haim brezis.pdfFunction Analysis - Haim brezis.pdfFunction Analysis - Haim brezis.pdfFunction Analysis - Haim brezis.pdfFunction Analysis - Haim brezis.pdfFunction Analysis - Haim brezis.pdfs,vvarbb]{newtxmath}
\usepackage{eucal}
\usepackage[euler-digits]{eulervm}
\usepackage[unicode=true]{hyperref}
\hypersetup{colorlinks = true}
%\hypersetup{hidelinks=true}
\hypersetup{
     colorlinks,
     linkcolor={black!10!red},
     linkbordercolor = {black!100!red},
%    <your other options...>,
     citecolor={blue}
}





%graphic
%\usepackage[text={425pt,650pt},centering]{geometry}

\usepackage{pdfsync}

\usepackage{geometry}
\geometry{verbose,tmargin=2.5cm,bmargin=2.5cm,lmargin=2.5cm,rmargin=2.5cm,headheight=3.5cm}

\usepackage{graphicx}
\usepackage{epsfig}
\usepackage{tikz}
\usepackage{caption}
\usepackage{color} %color
\definecolor{vert}{rgb}{0,0.6,0}

\usepackage{comment}
\numberwithin{figure}{section}
%\pagestyle{plain}


\theoremstyle{plain}
\newtheorem{thm}{Theorem}[section]
\newtheorem{ass}{Assumption}
\renewcommand{\theass}{}
\newtheorem{defn}{Definition}
\newtheorem{quest}{Question}
\newtheorem{com}{Comment}
\newtheorem{ex}{Example}
\newtheorem{lem}[thm]{Lemma}
\newtheorem{cor}[thm]{Corollary}
\newtheorem{prop}[thm]{Proposition}
\theoremstyle{remark}
\newtheorem{rem}{\bf{Remark}}
\numberwithin{equation}{section}



%\renewcommand{\thefootnote}{\fnsymbol{footnote}}




%Characters -- Shortcuts
\newcommand{\E}{\mathbb{E}}
\newcommand{\M}{\mathbb{M}}
\newcommand{\N}{\mathbb{N}}
\newcommand{\bP}{\mathbb{P}}
\newcommand{\R}{\mathbb{R}}
\newcommand{\bS}{\mathbb{S}}
\newcommand{\T}{\mathbb{T}}
\newcommand{\Z}{\mathbb{Z}}
\newcommand{\bfS}{\mathbf{S}}
\newcommand{\cA}{\mathcal{A}}
\newcommand{\cB}{\mathcal{B}}
\newcommand{\cC}{\mathcal{C}}
\newcommand{\cF}{\mathcal{F}}
\newcommand{\cH}{\mathcal{H}}
\newcommand{\cL}{\mathcal{L}}
\newcommand{\cM}{\mathcal{M}}
\newcommand{\cP}{\mathcal{P}}
\newcommand{\cS}{\mathcal{S}}
\newcommand{\cT}{\mathcal{T}}
\newcommand{\cE}{\mathcal{E}}
\newcommand{\I}{\mathrm{I}}


%Functional spaces
\newcommand{\AC}{{\rm AC\,}}
\newcommand{\ACl}{{\rm AC}_{{\rm loc}}}
\newcommand{\BUC}{{\rm BUC\,}}
\newcommand{\USC}{{\rm USC\,}}
\newcommand{\LSC}{{\rm LSC\,}}
\newcommand{\Li}{L^{\infty}}
\newcommand{\Lip}{{\rm Lip\,}}
\newcommand{\W}{W^{1,\infty}}
\newcommand{\Wx}{W_x^{1,\infty}}


%Domains
\newcommand{\bO}{\partial\Omega}
\newcommand{\cO}{\overline\Omega}
\newcommand{\Q}{\mathbb{T}^{n}\times(0,\infty)}
\newcommand{\iQ}{\mathbb{T}^{n}\times\{0\}}
\newcommand{\cQ}{\mathbb{T}^{n}\times[0,\infty)}


%Greek alphabets -- Shortcuts
\newcommand{\al}{\alpha}
\newcommand{\gam}{\gamma}
\newcommand{\del}{\delta}
\newcommand{\ep}{\varepsilon}
\newcommand{\kap}{\kappa}
\newcommand{\lam}{\lambda}
\newcommand{\sig}{\sigma}
\newcommand{\om}{\omega}
\newcommand{\Del}{\Delta}
\newcommand{\Gam}{\Gamma}
\newcommand{\Lam}{\Lambda}
\newcommand{\Om}{\Omega}
\newcommand{\Sig}{\Sigma}



%Overlines, Underlines -- Shortcuts
\newcommand{\ol}{\overline}
\newcommand{\ul}{\underline}
\newcommand{\pl}{\partial}
\newcommand{\supp}{{\rm supp}\,}
\newcommand{\inter}{{\rm int}\,}
\newcommand{\loc}{{\rm loc}\,}
\newcommand{\co}{{\rm co}\,}
\newcommand{\diam}{{\rm diam}\,}
\newcommand{\diag}{{\rm diag}\,}
\newcommand{\dist}{{\rm dist}\,}
\newcommand{\Div}{{\rm div}\,}
\newcommand{\sgn}{{\rm sgn}\,}
\newcommand{\tr}{{\rm tr}\,}
\newcommand{\Per}{{\rm Per}\,}

\newcommand{\rmC}{\mathrm{C}}
\newcommand{\rup}{\rightharpoonup}


\renewcommand{\subjclassname}{%
\textup{2010} Mathematics Subject Classification} 

%Hyperlink in PDF file
%\usepackage[dvipdfm,
%  colorlinks=false,
%  bookmarks=true,
%  bookmarksnumbered=false,
%  bookmarkstype=toc]{hyperref}
%\makeatletter
%\def\@pdfm@dest#1{%
%  \Hy@SaveLastskip
%  \@pdfm@mark{dest (#1) [@thispage /\@pdfview\space @xpos @ypos null]}%
%  \Hy@RestoreLastskip
%}


%BibLatex
%\usepackage[
%backend=biber,
%style=alphabetic,
%sorting=ynt
%]{biblatex}
%\addbibresource{rate.bib}

%%%%%%%%%%%%%%%%%%%%%%%%%%%%%%%%%%%%%%%%%%%%%%%%%%%%%%%%%%%%%%%%%%%%%%%%%%%%%%%%%%%%%%%%%%%%%%%%%%%%%%%%%%%%%%%%%%%%%%%%%%%%%%%%%%%%%%%%


%%%%%%%%%%%%%%%%%%%%%%%%%%%%%%%%%%%%%%%%%%%%%%%%%%%%%%%%%%%
\usepackage{import}
\usepackage{xifthen}
\usepackage{pdfpages}
\usepackage{transparent}
\newcommand{\incfig}[1]{%
    \def\svgwidth{\columnwidth}
    \import{./figs/}{#1.pdf_tex}
}
%%%%%%%%%%%%%%%%%%%%%%%%%%%%%%%%%%%%%%%%%%%%%%%%%%%%%%%%%%%
\begin{document}
\title[Rate of convergence]
{\textsc{Notes on the project}}
\thanks{The authors are supported in part by NSF grant DMS-1664424.}
%\begin{abstract}
%We investigate qualitatively the convergence of large, or state-constraint solution to nonlinear elliptic equation as the viscosity vanish.
%\end{abstract}
%%%%%%%%%%%%%%%%%%%%%%%%%%%%%%%%%%%%%%%%%%%%%%%%%%%%%%%%%%%
\author{Yuxi Han}
\address[Y. Han]
{
Department of Mathematics, 
University of Wisconsin Madison, 480 Lincoln  Drive, Madison, WI 53706, USA}
\email{yuxi.han@wisc.edu}
\author{Son N. T. Tu}
\address[S. N.T. Tu]
{
Department of Mathematics, 
University of Wisconsin Madison, 480 Lincoln  Drive, Madison, WI 53706, USA}
\email{thaison@math.wisc.edu}
\date{\today}
\keywords{first-order Hamilton--Jacobi equations; state-constraint problems; optimal control theory; rate of convergence; viscosity solutions.}
\subjclass[2010]{
35B40, %Asymptotic behavior of solutions, 
35D40, %Viscosity solutions
49J20, %Optimal control problems involving partial differential equations
49L25, %Viscosity solutions
70H20 %Hamilton-Jacobi equations
}
\maketitle
\setcounter{tocdepth}{1}
%\tableofcontents

%%%%%%%%%%%%%%%%%%%%%%%%%%%%%%%%%%%%%%%%%%%%%%%%%%%%%%%%%%%



\section{Doubling variable - Attempt No. 30}
\noindent (February 15, 2021) \\
Let $\alpha = \frac{1}{2}$, which corresponds to $p = \frac{5}{3}$. In this case
\begin{equation*}
    C_\varepsilon = \frac{(\alpha+1)^{\alpha+1}}{\alpha}\varepsilon^{\alpha+1} = \left(\frac{3\sqrt{6}}{2}\right)\varepsilon^{3/2} = A\varepsilon^{3/2}.
\end{equation*}
Recall from \cite{alessio_asymptotic_2006}) we have
\begin{equation*}
|\nabla u^\varepsilon(x)|  \leq C\frac{\varepsilon^{3/2}}{d(x)^{3/2}} \qquad\text{for}\;x\in \Omega.
\end{equation*}
Let us define
\begin{equation*}
    \Phi(x,y) = u^\varepsilon(x) - u_\delta(x) -\frac{C_0|x-y|^2}{\sigma} - \frac{C_\varepsilon}{d(x)^{1/2}}, \qquad (x,y)\in \overline{\Omega}\times \overline{\Omega}_\delta.
\end{equation*}
Assume it has maximum at $(x_\sigma,y_\sigma)\in \overline{\Omega}\times \overline{\Omega}_\delta$. We want
\begin{equation*}
    \frac{\varepsilon^{3/2}}{d(x)^{3/2}} \rightarrow 0.
\end{equation*}
Therefore the goal is
\begin{equation*}
    d(x)\geq C\varepsilon^\theta \qquad\text{for} \qquad 0<\theta < 1.
\end{equation*}
Let $z\in \overline{\Omega}_\delta$ so that $u^\varepsilon(x) - u_\delta(x)$ has a maximum over $\overline{\Omega}_\delta$ at $z$, then $d(z)\geq \delta$. \\
\paragraph{Step 1.} From $\Phi(x_\sigma,y_\sigma)\geq \Phi(z,z)$ we have
\begin{align*}
    u^\varepsilon(x_\sigma) - u_\delta(y_\sigma) - \frac{C_0|x_\sigma - y_\sigma|^2}{\sigma} - \frac{A\varepsilon^{3/2}}{d(x_\sigma)^{1/2}} &\geq u^\varepsilon(z) - u_\delta(z) - \frac{A\varepsilon^{3/2}}{d(z)^{1/2}}\\
    &\geq u^\varepsilon(x_\sigma) - u_\delta(x_\sigma) - \frac{A\varepsilon^{3/2}}{d(z)^{1/2}}.
\end{align*}
Therefore
\begin{equation*}
    \frac{C_0|x_\sigma - y_\sigma|^2}{\sigma} + \frac{A\varepsilon^{3/2}}{d(x_\sigma)^{1/2}} \leq C_0|x_\sigma - y_\sigma|+\frac{A\varepsilon^{3/2}}{d(z)^{1/2}} \leq C_0|x_\sigma - y_\sigma|+\frac{A\varepsilon^{3/2}}{\delta^{1/2}} .
\end{equation*}
Using
\begin{equation*}
    C_0|x_\sigma-y_\sigma| \leq C_0\left(\frac{|x_\sigma - y_\sigma|^2}{\sigma}+\frac{\sigma}{4}\right) 
\end{equation*}
we deduce that
\begin{equation}\label{sleep}
    \frac{A\varepsilon^{3/2}}{d(x_\sigma)^{1/2}} \leq \frac{C_0\sigma}{4} + \frac{A\varepsilon^{3/2}}{\delta^{1/2}}.
\end{equation}
Choose $\sigma = \frac{4A}{C_0}\varepsilon^{5/4}$ and $\delta = \varepsilon^{1/2}$, then
\begin{equation*}
     \frac{A\varepsilon^{3/2}}{d(x_\sigma)^{1/2}} \leq A\varepsilon^{5/4} + \frac{A\varepsilon^{3/2}}{\varepsilon^{1/4}} = A(\varepsilon^{5/4} +\varepsilon^{5/4}) = 2A\varepsilon^{5/4}.
\end{equation*}
Therefore
\begin{equation*}
    d(x_\sigma)^{1/2} \geq \frac{1}{2}\varepsilon^{1/4} \qquad\Longrightarrow\qquad d(x_\sigma) \geq \frac{1}{4}\varepsilon^{1/2}.
\end{equation*}

\paragraph{Step 2.} From $\Phi(x_\sigma,y_\sigma)\geq \Phi(x_\sigma,(1-\theta)x_\sigma)$ where
\begin{equation*}
    \theta = 1-\frac{1-\varepsilon^{1/2}}{1-1/4\varepsilon^{1/2}} = \frac{3/4\varepsilon^{1/2}}{1-1/4\varepsilon^{1/2}} \qquad\text{so that}\qquad (1-\theta) x_\sigma \in \overline{\Omega}_\delta,
\end{equation*}
we have
\begin{align*}
     u^\varepsilon(x_\sigma) - u_\delta(y_\sigma) - \frac{C_0|x_\sigma - y_\sigma|^2}{\sigma}  &\geq u^\varepsilon(x_\sigma) - u_\delta((1-\theta)x_\sigma)  - \frac{C_0\theta^2|x_\sigma|^2}{\sigma}
\end{align*}
and therefore
\begin{equation*}
    \frac{C_0|x_\sigma - y_\sigma|^2}{\sigma} \leq C_0|(1-\theta)x_\sigma - y_\sigma| + \frac{C_0\theta^2 |x_\sigma|^2}{\sigma}
\end{equation*}
This line of argument will give us eventually that
\begin{equation*}
    |x_\sigma - y_\sigma| = \mathcal{O}\left(\varepsilon^{1/2}\right).
\end{equation*}
This is apparently \emph{bad}, since $\frac{(x_\sigma - y_\sigma)}{\sigma}$ now is unbounded.

\paragraph{Step 3.} Using $x\mapsto \Phi(x,y_\sigma)$ has a maximum over $\Omega$ at $x=x_\sigma$, we have
\begin{equation*}
    u^\varepsilon(x) - \left(\frac{C_0|x-y_\sigma|^2}{\sigma}  + \frac{A\varepsilon^{3/2}}{d(x)^{1/2}}\right) \qquad\text{has a max at}\;x_\sigma \in \Omega_{\frac{1}{8}\varepsilon^{1/2}}.
\end{equation*}
Since $u^\varepsilon$ is Lipschitz in $\Omega_{\frac{1}{8}\varepsilon^{1/2}}$ with constant given by 
\begin{equation*}
|\nabla u^\varepsilon(x)|  \leq C\frac{\varepsilon^{3/2}}{d(x)^{3/2}} \leq C\varepsilon^{3/4} \qquad\text{for}\;x\in \Omega.
\end{equation*}
we deduce that 
\begin{equation*}
    |\xi|\leq C\varepsilon^{3/4} \qquad\text{for all}\qquad \xi\in D^+u^\varepsilon(x)\;\text{and for all}\; x\in \Omega_{\frac{1}{8}\varepsilon^{1/2}}.
\end{equation*}
Therefore we conclude that
\begin{equation*}
    \frac{2C_0(x_\sigma-y_\sigma)}{\sigma} -\frac{1}{2}\frac{A\varepsilon^{3/2}}{d(x_\sigma)^{3/2}\nabla d(x_\sigma)} \in D^+u^\varepsilon(x_\sigma).
\end{equation*}
and thus
\begin{equation*}
    \left|\frac{2C_0(x_\sigma-y_\sigma)}{\sigma} -\frac{1}{2}\frac{A\varepsilon^{3/2}\nabla d(x_\sigma)}{d(x_\sigma)^{3/2}}\right| \leq C\varepsilon^{3/4}
\end{equation*}
and thus (we stop keeping track of the constant at this point)
\begin{equation*}
    |\xi_\sigma| = \left|\frac{2C_0(x_\sigma - y_\sigma)}{\sigma}\right| \leq C\varepsilon^{3/4}  + \frac{A\varepsilon^{3/2}}{d(x_\sigma)^{3/2}} \leq C\varepsilon^{3/4}.
\end{equation*}
This is way better than Step 2. We define
\begin{equation*}
    \xi_\sigma = \frac{2C_0(x_\sigma - y_\sigma)}{\sigma} \qquad\text{and}\qquad \zeta_\sigma =-\frac{A\varepsilon^{3/2}\nabla d(x_\sigma)}{2d(x_\sigma)^{3/2}}.
\end{equation*}
\paragraph{Step 4.} The subsolution test also gives us
\begin{align}\label{e:subb1}
    \lambda u^\varepsilon(x_\sigma) + |\xi_\sigma+\zeta_\sigma|^p - f(x_\sigma) -  \varepsilon \left(\frac{2nC_0}{\sigma} + \frac{3A}{4}\frac{\varepsilon^{3/2}}{d(x_\sigma)^{5/2}} + \frac{A\varepsilon^{3/2}\Delta d(x_\sigma)}{2d(x_\sigma)^{3/2}}\right) \leq 0.
\end{align}

\paragraph{Step 5.} We have $y\mapsto \Phi^\sigma(x_\sigma,y)$ has a maximum at $y_\sigma\in \overline{\Omega}_\delta$, thus by supersolution test we have 
\begin{equation}\label{e:super1}
    \lambda u_\delta(y_\sigma) + \left|\xi_\sigma\right|^p - f(y_\sigma) \geq 0.
\end{equation}

\paragraph{Step 6.} Combining \eqref{e:subb1} and \eqref{e:super1} we have
\begin{align}
    &\lambda u^\varepsilon(x_\sigma) - \lambda u_\delta(y_\sigma) \leq |\xi_\sigma|^p - |\xi_\sigma+\zeta_\sigma|^p + f(y_\sigma) - f(x_\sigma) \nonumber \\
    & \qquad\qquad\qquad\qquad\qquad\qquad + \varepsilon \left(\frac{2nC_0}{\sigma} + \frac{3A}{4}\frac{\varepsilon^{3/2}}{d(x_\sigma)^{5/2}} + \frac{AK_2\varepsilon^{3/2}}{2d(x_\sigma)^{3/2}}\right) 
\end{align}
where $K_2 = \max_{x\in \overline{\Omega}}\Delta d(x)$. Using the estimate
\begin{align*}
    \left||x+y|^p - |x|^p \right| \leq  p\big(|x|+|y|\big)^{p-1}|y|
\end{align*}
with $x = \xi_\sigma$, $y = \zeta_\sigma$ and the fact that $|\zeta_\sigma|\leq \mathfrak{C}_\alpha$ we obtain
\begin{align}
    \Big||\xi_\sigma+\zeta_\sigma|^p - |\xi|^p\Big| &\leq p\Big(|\xi_\sigma|+|\zeta_\sigma|\Big)^{p-1}|\zeta_\sigma| \leq K_3\varepsilon ^{5/4}.\label{e:est2}
\end{align}
The whole effort gets wrong here, as we pick 
\begin{equation*}
    \sigma = \varepsilon^{5/4}
\end{equation*}
is \emph{bad}. Some options
\begin{enumerate}
    \item Use 
    \begin{equation*}
        \frac{C_0C_\varepsilon|x-y|^2}{\sigma}.
    \end{equation*}
    \item Use 
    \begin{equation*}
        \frac{C_0|x-y|^3}{\sigma}.
    \end{equation*}
    \item Trying to optimize \eqref{sleep}.\\ \emph{I already tried! Doesn't work!}
\end{enumerate}

\bibliography{zzzzlibrary}{}
%\bibliographystyle{ieeetr}
\bibliographystyle{acm}










\end{document}