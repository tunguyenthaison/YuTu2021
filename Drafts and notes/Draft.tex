\documentclass[11pt,reqno]{amsart}
%============%============%============%============%
\usepackage{marktext}
%============%============%============%============%
%\setlength{\columnseprule}{0.4pt}
%\setlength{\topmargin}{0cm}
%\setlength{\oddsidemargin}{.25cm}
%\setlength{\evensidemargin}{.25cm}
%\setlength{\textheight}{22.5cm}
%\setlength{\textwidth}{15.5cm}
\renewcommand{\baselinestretch}{1.05}
%============%============%============%============%
\usepackage[toc,page]{appendix}
%============%============%============%============%
\usepackage{xcolor}
\usepackage{placeins}
\usepackage{amsfonts,amsmath,amsthm}
\usepackage{amssymb,epsfig}
\usepackage{enumerate} 
\usepackage[notcite,notref]{showkeys}
\usepackage{fullpage}
%============%============%============%============%
\usepackage[utf8]{inputenc}
\usepackage{mathpazo}
%\usepackage[euler-digits]{eulervm}

\usepackage{eucal}
\usepackage[unicode=true]{hyperref}
\hypersetup{colorlinks = true}
%\hypersetup{hidelinks=true}
\hypersetup{
     colorlinks,
     linkcolor={black!10!red},
     linkbordercolor = {black!100!red},
%    <your other options...>,
     citecolor={blue}
}





%graphic
%\usepackage[text={425pt,650pt},centering]{geometry}

\usepackage{pdfsync}

\usepackage{geometry}
\geometry{verbose,tmargin=2.5cm,bmargin=2.5cm,lmargin=2.5cm,rmargin=2.5cm,headheight=3.5cm}

\usepackage{graphicx}
\usepackage{epsfig}
\usepackage{tikz}
\usepackage{caption}
\usepackage{color} %color
\definecolor{vert}{rgb}{0,0.6,0}

\usepackage{comment}
\numberwithin{figure}{section}
%\pagestyle{plain}


\theoremstyle{plain}
\newtheorem{thm}{Theorem}[section]
\newtheorem{ass}{Assumption}
\renewcommand{\theass}{}
\newtheorem{defn}{Definition}
\newtheorem{quest}{Question}
\newtheorem{com}{Comment}
\newtheorem{ex}{Example}
\newtheorem{lem}[thm]{Lemma}
\newtheorem{cor}[thm]{Corollary}
\newtheorem{prop}[thm]{Proposition}
\theoremstyle{remark}
\newtheorem{rem}{\bf{Remark}}
\numberwithin{equation}{section}



%\renewcommand{\thefootnote}{\fnsymbol{footnote}}




%Characters -- Shortcuts
\newcommand{\E}{\mathbb{E}}
\newcommand{\M}{\mathbb{M}}
\newcommand{\N}{\mathbb{N}}
\newcommand{\bP}{\mathbb{P}}
\newcommand{\R}{\mathbb{R}}
\newcommand{\bS}{\mathbb{S}}
\newcommand{\T}{\mathbb{T}}
\newcommand{\Z}{\mathbb{Z}}
\newcommand{\bfS}{\mathbf{S}}
\newcommand{\cA}{\mathcal{A}}
\newcommand{\cB}{\mathcal{B}}
\newcommand{\cC}{\mathcal{C}}
\newcommand{\cF}{\mathcal{F}}
\newcommand{\cH}{\mathcal{H}}
\newcommand{\cL}{\mathcal{L}}
\newcommand{\cM}{\mathcal{M}}
\newcommand{\cP}{\mathcal{P}}
\newcommand{\cS}{\mathcal{S}}
\newcommand{\cT}{\mathcal{T}}
\newcommand{\cE}{\mathcal{E}}
\newcommand{\I}{\mathrm{I}}


%Functional spaces
\newcommand{\AC}{{\rm AC\,}}
\newcommand{\ACl}{{\rm AC}_{{\rm loc}}}
\newcommand{\BUC}{{\rm BUC\,}}
\newcommand{\USC}{{\rm USC\,}}
\newcommand{\LSC}{{\rm LSC\,}}
\newcommand{\Li}{L^{\infty}}
\newcommand{\Lip}{{\rm Lip\,}}
\newcommand{\W}{W^{1,\infty}}
\newcommand{\Wx}{W_x^{1,\infty}}


%Domains
\newcommand{\bO}{\partial\Omega}
\newcommand{\cO}{\overline\Omega}
\newcommand{\Q}{\mathbb{T}^{n}\times(0,\infty)}
\newcommand{\iQ}{\mathbb{T}^{n}\times\{0\}}
\newcommand{\cQ}{\mathbb{T}^{n}\times[0,\infty)}


%Greek alphabets -- Shortcuts
\newcommand{\al}{\alpha}
\newcommand{\gam}{\gamma}
\newcommand{\del}{\delta}
\newcommand{\ep}{\varepsilon}
\newcommand{\kap}{\kappa}
\newcommand{\lam}{\lambda}
\newcommand{\sig}{\sigma}
\newcommand{\om}{\omega}
\newcommand{\Del}{\Delta}
\newcommand{\Gam}{\Gamma}
\newcommand{\Lam}{\Lambda}
\newcommand{\Om}{\Omega}
\newcommand{\Sig}{\Sigma}



%Overlines, Underlines -- Shortcuts
\newcommand{\ol}{\overline}
\newcommand{\ul}{\underline}
\newcommand{\pl}{\partial}
\newcommand{\supp}{{\rm supp}\,}
\newcommand{\inter}{{\rm int}\,}
\newcommand{\loc}{{\rm loc}\,}
\newcommand{\co}{{\rm co}\,}
\newcommand{\diam}{{\rm diam}\,}
\newcommand{\diag}{{\rm diag}\,}
\newcommand{\dist}{{\rm dist}\,}
\newcommand{\Div}{{\rm div}\,}
\newcommand{\sgn}{{\rm sgn}\,}
\newcommand{\tr}{{\rm tr}\,}
\newcommand{\Per}{{\rm Per}\,}

\newcommand{\rmC}{\mathrm{C}}



\renewcommand{\subjclassname}{%
\textup{2010} Mathematics Subject Classification} 

%Hyperlink in PDF file
%\usepackage[dvipdfm,
%  colorlinks=false,
%  bookmarks=true,
%  bookmarksnumbered=false,
%  bookmarkstype=toc]{hyperref}
%\makeatletter
%\def\@pdfm@dest#1{%
%  \Hy@SaveLastskip
%  \@pdfm@mark{dest (#1) [@thispage /\@pdfview\space @xpos @ypos null]}%
%  \Hy@RestoreLastskip
%}


%BibLatex
%\usepackage[
%backend=biber,
%style=alphabetic,
%sorting=ynt
%]{biblatex}
%\addbibresource{rate.bib}

%%%%%%%%%%%%%%%%%%%%%%%%%%%%%%%%%%%%%%%%%%%%%%%%%%%%%%%%%%%%%%%%%%%%%%%%%%%%%%%%%%%%%%%%%%%%%%%%%%%%%%%%%%%%%%%%%%%%%%%%%%%%%%%%%%%%%%%%%%%%%%%%%%



\usepackage{import}
\usepackage{xifthen}
\usepackage{pdfpages}
\usepackage{transparent}
\newcommand{\incfig}[1]{%
    \def\svgwidth{\columnwidth}
    \import{./figs/}{#1.pdf_tex}
}



\begin{document}

\title[Rate of convergence]
{\textsc{Second-order state-constraint Hamilton-Jacobi equations}}




\thanks{The authors are supported in part by NSF grant DMS-1664424.}

\begin{abstract}

\end{abstract}




\author{Yuxi Han}
\address[Y. Han]
{
Department of Mathematics, 
University of Wisconsin Madison, 480 Lincoln  Drive, Madison, WI 53706, USA}
\email{yuxi.han@wisc.edu}



\author{Son N. T. Tu}
\address[S. N.T. Tu]
{
Department of Mathematics, 
University of Wisconsin Madison, 480 Lincoln  Drive, Madison, WI 53706, USA}
\email{thaison@math.wisc.edu}

\date{\today}
\keywords{first-order Hamilton--Jacobi equations; state-constraint problems; optimal control theory; rate of convergence; viscosity solutions.}
\subjclass[2010]{
35B40, %Asymptotic behavior of solutions, 
35D40, %Viscosity solutions
49J20, %Optimal control problems involving partial differential equations
49L25, %Viscosity solutions
70H20 %Hamilton-Jacobi equations
}


\maketitle

%\tableofcontents







%%%%%%%%%%%%%%%%%%%%%%%%%%%%%%%%%%%%%%%%%%%%%%%%%%%%%%%%%%%%%%%


\section{Introduction}\label{sec:intro}

Let $\Omega$ be an open, bounded and connected with $\mathrm{C}^2$ boundary domain of $\mathbb{R}^n$. Assume that $f\in \mathrm{C}(\overline{\Omega})\cap W^{1,\infty}(\Omega)$ and $1<p<2$. Let $u^\varepsilon$ be the solution to 
\begin{equation}\label{eq:PDE}
    \begin{cases}
    \lambda u^\varepsilon(x) + |Du^\varepsilon(x)|^p - f(x) - \varepsilon \Delta u^\varepsilon(x) = 0 \qquad
    \text{in}\;\Omega, \vspace{0cm}\\
    \displaystyle  \lim_{\mathrm{dist}(x,\partial \Omega)\to 0} u^\varepsilon(x) = +\infty.
    \end{cases} \tag{PDE$_\varepsilon$}
\end{equation}
We are interested in studying the asymptotic behavior of $\{u^\varepsilon\}_{\varepsilon>0}$ as the vanishing viscosity happens: $\varepsilon\rightarrow 0$.

Paper to read \cite{Ishii2017a} and \cite{Lasry1989}.

\section{Preliminaries}\label{sec:prelim}




\subsection{Convergence result}
Assume $\Omega$ is an open, bounded and connected with $\mathrm{C}^2$ boundary domain of $\mathbb{R}^n$. Assume that $f\in \mathrm{C}(\overline{\Omega})\cap W^{1,\infty}(\Omega)$ and $1<p<2$. 

Let $u^\varepsilon$ be the solution to \eqref{eq:PDE}, we observe that if$u^\varepsilon\rightarrow u^0$ uniformly on compact subsets of $\Omega$, we see that $u(x)\rightarrow +\infty$ as $\mathrm{dist}(x,\partial \Omega)\to 0$ as well, therefore the limiting solution is not the state-constraint solution to a limiting first order equation.

\begin{lem}\label{lem:max} Let $u\in \rmC(\overline{\Omega})$ be a viscosity subsolution of 
\begin{equation}\label{S_0}
 \lambda u(x) + H(x,Du(x)) = 0\;\qquad\text{in}\;\Omega
\end{equation}
such that, for all viscosity subsolution $v\in \rmC(\overline{\Omega})$ of \eqref{S_0} one has $v\leq u\in \overline{\Omega}$, then $u$ is a viscosity supersolution of \eqref{S_0} on $\overline{\Omega}$.
\end{lem}



\begin{thm}[Qualitative result] Assume $\Omega$ is an open, bounded and connected with $\mathrm{C}^2$ boundary domain of $\mathbb{R}^n$. Assume that $f\in \mathrm{C}(\overline{\Omega})\cap W^{1,\infty}(\Omega)$ and $1<p<2$. Let $u^\varepsilon$ be the solution to \eqref{eq:PDE}, then $u^\varepsilon \rightarrow u$ locally uniformly in $\Omega$ as $\varepsilon\rightarrow 0$ and $u$ solves
\begin{equation}\label{eq:PDE0}
    \begin{cases}
    \lambda u(x) + |Du(x)|^p - f(x) \leq 0 \qquad\text{in}\;\Omega,\\
    \lambda u(x) + |Du(x)|^p - f(x) \geq 0 \qquad\text{on}\;\overline{\Omega}.
    \end{cases}\tag{PDE$_0$}   
\end{equation}
\end{thm}


\color{blue}
\begin{proof} (\textit{Added on Jan 16, 2021 - Heuristic idea}) Assume that we can get some sort of equi-continuity of the family $\{u^\varepsilon\}_{\varepsilon>0}$ in a local way, that is, for any compact sub-domain $\Omega'\subset\subset \Omega$ there is a constant $C(\Omega')$ such that
\begin{equation}\label{eq:bound}
    \Vert u^\varepsilon\Vert_{L^\infty(\Omega')} +\Vert Du^\varepsilon\Vert_{L^\infty(\Omega')} \leq C(\Omega').
\end{equation}
This should be done using Berstein's method (fill in later, and as $u$ blows up on $\partial \Omega$, this bound \eqref{eq:bound} may not be improved). Now let us assume via some subsequence $\varepsilon_j\to 0$ we have $u^{\varepsilon_j}\rightarrow u^0$ uniformly on compact subsets of $\Omega$, the interior stability of viscosity solution should yield us that 
\begin{equation*}
    \lambda u^0(x) + |Du^0(x)|^p - f(x) = 0 \qquad\text{in}\;\Omega.
\end{equation*}
The real difficulty is describing what is the behavior of $u^0$ on the boundary. As we expect it to solves \eqref{eq:PDE0}, it should be the maximal subsolution (hence, solution) to 
\begin{equation}\label{eq:S_lambda}
    \lambda u(x) + |Du(x)|^p - f(x) = 0 \qquad\text{in}\;\Omega.
\end{equation}
Let $v^\varepsilon\in \rmC^2(\Omega)\cap \rmC(\overline{\Omega})$ be the unique solution to the Dirichlet problem
\begin{equation}\label{eq:v_eps}
\begin{cases}
    \lambda v^\varepsilon(x) + |Dv^\varepsilon(x)|^p - f(x) = \varepsilon \Delta v^\varepsilon(x) &\qquad\text{in}\;\Omega,\\
    \;\;\,\quad\quad\qquad\qquad\qquad\qquad v^\varepsilon = u &\qquad \text{on}\;\partial\Omega.
\end{cases}
\end{equation}
where $u\in \rmC(\overline{\Omega})$ is the unique solution of \eqref{eq:PDE0}. Note that $v^\varepsilon\rightarrow v^0$ solves the corresponding Dirichlet problem
\begin{equation}\label{eq:v_0}
\begin{cases}
    \lambda v^0(x) + |Dv^0(x)|^p - f(x) = 0 &\qquad\text{in}\;\Omega,\\
    \;\;\,\quad\quad\qquad\qquad\qquad\qquad v^0 = u &\qquad \text{on}\;\partial\Omega.
\end{cases}
\end{equation}
\begin{lem} $v^0\equiv u$ as the unique solution to \eqref{eq:PDE0}.
\end{lem}
\begin{proof} We observe that $u$ solves \eqref{eq:PDE0} is the unique solution to the Dirichlet problem \eqref{eq:v_0} as well, thus $v^0 \equiv u$ by comparison principle.
\end{proof}
\noindent By comparison principle, we see that
\begin{equation*}
    v^\varepsilon(x)\leq u^\varepsilon(x) \qquad\text{for}\;x\in \Omega.
\end{equation*}
Let $\varepsilon\to 0$, we deduce that
\begin{equation*}
    u(x)\leq u^0(x) \qquad\text{for}\;x\in \Omega.
\end{equation*}
\end{proof}
\color{black}



%\nocite{*}



\section{Questions}
\begin{quest} [Jan 12, 2021.] Why do we use the distance functions to get boundary estimates? 
\end{quest}

\begin{quest} [Jan 13, 2021.] Maximum principle for sub-quadratic case.
\end{quest}


\begin{appendices}
\section{Construction of solution for sub-quadratic case}
In this section, with the specific form of the Hamiltonian $H(x,\varrho) = |\varrho|^p - f(x)$ where $1<p< 2$ and $f\in \mathrm{C}(\overline{\Omega})\cap W^{1,\infty}(\Omega)$, we show the existence and uniqueness of solutions to \eqref{eq:PDE}. 
\begin{thm} If $H(x,\varrho) = |\varrho|^p - f(x)$ where $1<p< 2$ and $f\in \mathrm{C}(\overline{\Omega})\cap W^{1,\infty}(\Omega)$ then there exists a unique solution $u\in \mathrm{C}^2(\Omega)$ of \eqref{eq:PDE}.
\end{thm}


\begin{proof} For $\delta>0$, let us define $\Omega_\delta = \{x\in \Omega: \mathrm{dist}(x,\Omega) < \delta\}$ and $\Omega^\delta = \{x\in \mathbb{R}^n: \mathrm{dist}(x,\overline{\Omega}) < \delta\}$. 
\begin{figure}[ht]
    \centering
    %\incfig{Domains}
    \def\svgwidth{0.5\columnwidth}
    \import{./figs/}{Domains.pdf_tex}
    \caption{The domain $\Omega$ variations $\Omega_\delta, \Omega^\delta$.}
    \label{fig:Domains}
\end{figure}
We consider $\delta$ small enough so that the distance function $x\mapsto \mathrm{dist}(x,\partial \Omega)$ is $\mathrm{C}^2$ in the strip $\Omega^\delta\backslash \overline{\Omega}_\delta$, we recall that $|D d(x)| = 1$ in that region. We extend the signed distance function into a $\mathrm{C}^2(\mathbb{R}^n)$ function, denoted by $d(x)$ such that 
\begin{equation*}
    \begin{cases}
    d(x)\geq 0\;\text{for}\;x\in\Omega\;\text{with}\;d(x) = +\mathrm{dist}(x,\partial\Omega)\;\text{for}\;x\in \Omega\backslash \Omega_\delta,\\
    d(x)\leq 0\;\text{for}\;x\notin \Omega\;\text{with}\;d(x) = -\mathrm{dist}(x,\partial\Omega)\;\text{for}\;x\in \Omega^\delta\backslash \Omega.
    \end{cases}
\end{equation*}
We can choose $d(\cdot)$ so that $\Omega_\delta = \{x\in \mathbb{R}^n: d(x)-\delta >0 \}$ and $\Omega^\delta = \{x\in \mathbb{R}^n: d(x) +\delta>0\}$.

To find candidate for subsolution and supersolution to \eqref{eq:PDE}, we use the ansatz
\begin{equation}\label{eq:ansatz}
    u(x) = C_0 d(x)^{-\alpha}, \qquad x\in \Omega
\end{equation}
as we expect it blows up near the boundary like some thing proportionate to inverse of the distance function, due to the structure of $H(x,\varrho) = |\varrho|^p - f(x)$. Let us plug \eqref{eq:ansatz} into \eqref{eq:PDE} we obtain that the highest order terms are
\begin{equation*}
        -C_0 \alpha(\alpha+1)d^{-(\alpha+2)} + C_0^p \alpha^p d^{-(\alpha+1)p}.
\end{equation*}
Setting them to zero, we deduce that
\begin{equation*}
    \displaystyle\alpha = \frac{2-p}{p-1} \qquad\text{and}\qquad C_0 = \frac{1}{\alpha}(\alpha+1)^\frac{1}{p-1}.
\end{equation*}
We can obtain the following families of subsolution on $\Omega^\delta$ and supersolution on $\Omega^\delta$ as follow.
\begin{equation*}
        \overline{w}_{\eta,\delta}(x) = \frac{C_0+\eta}{(d(x)-\delta)^\alpha} +\frac{C_\eta}{\lambda} \quad\text{for}\;x\in \Omega_\delta \qquad\text{and}\qquad \underline{w}_{\eta,\delta}(x) = \frac{C_0-\eta}{(d(x)+\delta)^\alpha} -\frac{C_\eta}{\lambda} \quad\text{for}\;x\in \Omega^\delta.
\end{equation*}
We divide the rest of the proof into 3 steps.
\begin{itemize}
    \item Step 1. Building a maximal solution to \eqref{eq:PDE}. 
\end{itemize}
\end{proof}




\section{Gradient bounds}
\subsection{Local interior gradient bound for elliptic equation}
Let $\Omega\subset \R^n$ be an open, bounded, connected set with $\rmC^2$ boundary and $H(x,p):\overline{\Omega}\times \R^n\rightarrow \mathbb{R}$ be a continuously differentiable Hamiltonian satisfying
\begin{equation}\label{eq:grow}
\lim_{|p\rightarrow \infty} \left(\frac{1}{2}H(x,p)^2 + D_xH(x,p)\cdot p\right) = +\infty \qquad\text{uniformly in}\;x\in \overline{\Omega}. \tag{H1}
\end{equation}
We consider the following equation
\begin{equation}\label{eq:C_eps}
    \lambda u^\varepsilon(x) + H(x,Du^\varepsilon(x)) - \varepsilon \Delta u^\varepsilon(x) = 0 \qquad\text{in}\;\Omega.
\end{equation}
Let $u^\varepsilon\in \rmC^2(\Omega)\cap \mathrm{C}^1(\overline{\Omega})$ be a bounded solution to \eqref{eq:C_eps}, say $|\lambda u^\varepsilon(x)| \leq C_1$ for all $x\in \overline{\Omega}$. In this section we show that an interior gradient bound also holds, i.e., if $x\mapsto|Du^\varepsilon(x)|$ has a maximum over $\overline{\Omega}$ at $x_0\in \Omega$ then $|Du^\varepsilon(x_0)| \leq C_2$ for all $x\in \Omega$. Here $C_1,C_2$ are independent of $\varepsilon>0$.\\


\noindent We will use the classical Bernstein's argument. Let $\varphi(x) = \frac{1}{2}|Du^\varepsilon(x)|^2$ for $x\in \Omega$. Differentiate \eqref{eq:C_eps} in $x_i$ then multiply the resulting equation with $u^\varepsilon_{x_i}$ and then sum over $i=1,2,\ldots, n$ we obtain 
\begin{equation*}
    2\lambda \varphi(x) + D_pH(x,Du^\varepsilon(x))\cdot D\varphi(x) - \varepsilon \Delta \varphi(x) + \Big(\varepsilon |D^2u^\varepsilon(x)| + D_xH(x,Du^\varepsilon(x))\cdot Du^\varepsilon(x)\Big) = 0.
\end{equation*}
If $\varphi(x)$ achieves its maximum over $\overline{\Omega}$ at $x_0\in \Omega$, then $D\varphi(x_0) = 0$ and $\Delta \varphi(x_0)\leq 0$, together with $\varepsilon |D^2u^\varepsilon(x)|^2\geq \frac{1}{n\varepsilon}(\varepsilon\Delta u^\varepsilon(x))^2\geq (\varepsilon\Delta u^\varepsilon(x))^2$ if $n\varepsilon < 1$ we deduce that
\begin{equation*}
    \lambda |D u^\varepsilon(x_0)|^2 + \Big(\lambda u^\varepsilon(x_0)+H(x_0,Du^\varepsilon(x_0))\Big)^2 + D_xH(x_0,Du^\varepsilon(x_0))\cdot Du^\varepsilon(x_0) \leq 0.
\end{equation*}
Assume $|\lambda u^\varepsilon(x_0)|\leq C_1$, using \eqref{eq:grow} we deduce that
\begin{equation*}
    \frac{1}{2}H(x_0,Du^\varepsilon(x_0))^2 + D_xH(x_0,Du^\varepsilon(x_0))\cdot Du^\varepsilon(x_0) + \left(\frac{1}{\sqrt{2}}H(x_0,Du^\varepsilon(x_0) - \sqrt{2}C_1\right)^2 - \left(C_1\right)^2\leq 0
\end{equation*}
which gives us $|Du^\varepsilon(x_0)|\leq C_2$ for some $C_2$ independent of $\varepsilon$. 
\begin{rem} Assumption \eqref{eq:grow} is weaker than the combination of $p\mapsto H(x,p)$ is superlinear and $|D_xH(x,p)|\leq C(1+|p|)$.
\end{rem}


\section{Differentiability with respect to the parameter}
For the vanishing viscosity problem with the Dirichlet boundary condition,
\begin{equation}
\label{dir}
\left\{
  \begin{aligned}
    H(x, Du^\epsilon(x)) &= \epsilon \Delta u^\epsilon \quad \, \text{in } U, \\
              u^\epsilon &= 0 \quad \qquad \text{on } \partial U,
  \end{aligned}
\right.
\end{equation}
we want to show the smooth dependence of $u^\epsilon$ on $\epsilon$.
Formally, if we differentiate \eqref{dir} with respect to $\epsilon$, we get
\begin{equation}
\label{dir_dif}
\left\{
  \begin{aligned}
    D_pH(x, Du^\epsilon(x))\cdot Du^\epsilon_\epsilon &= \epsilon \Delta u^\epsilon_\epsilon +\Delta u^\epsilon \quad \, \text{in } U, \\
              u^\epsilon_\epsilon &= 0 \quad \qquad  \qquad  \text{on } \partial U.
  \end{aligned}
\right.
\end{equation}
By Schaefer's fixed point theorem and the maximal principle, $u^\epsilon_\epsilon$ is the unique solution in $C^2(\overline{U})$ of \eqref{dir_dif} \textcolor{orange}{(needs double check)}. The main idea is, we look at the difference quotients $\displaystyle \frac{u^{\epsilon+h}-u^\epsilon}{h}$and prove that as $h \to 0^+$, they converge to a limiting function $w^{\ast}$ in the uniform norm \textcolor{orange}{(maybe weaker convergence is fine)} such that $w^{\ast}$ solves \eqref{dir_dif}. Since \eqref{dir_dif} has a unique solution, we have $$u^\epsilon_\epsilon=\lim_{h \to 0^+}\frac{u^{\epsilon+h}-u^\epsilon}{h}.$$

Fix $\epsilon >0$. Let $$w^h(x):=\frac{u^{\epsilon+h}(x)-u^\epsilon(x)}{h} \in C^2(\overline{U}).$$

A little computation will show that $w^h$ solves

\begin{equation}
\label{dir_quo}
\left\{
  \begin{aligned}
   \epsilon \Delta w^h(x) + \frac{\epsilon}{\epsilon + h}\Delta u^\epsilon &= \frac{\epsilon}{\epsilon +h} \int_0^1 D_pH(x, Du^\epsilon+\theta (Du^{\epsilon+h}-Du^\epsilon)) d\theta \cdot Dw^h \quad \, \text{in } U, \\
              w^h &= 0 \quad \qquad \text{on } \partial U.
  \end{aligned}
\right.
\end{equation}
Heuristically, we see as $h \to 0^+$, \eqref{dir_quo} "converges" to \eqref{dir_dif}.


\textcolor{orange}{Still thinking about how to show $w^h$ converges uniformly as $h \to 0^+$. I am tring to show $\| w^h-w^\ast\|_\infty \to 0$ by looking at the equation solved by $v^h:=w^h-w^\ast$. More to fill in...}

\end{appendices}

%\section*{Acknowledgement}




\bibliography{zzzzlibrary}{}
%\bibliographystyle{ieeetr}
\bibliographystyle{acm}










\end{document}