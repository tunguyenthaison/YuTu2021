\documentclass[10pt]{article}
\usepackage[T1]{fontenc}

\usepackage{xcolor}
\usepackage{fullpage}
\usepackage{mathrsfs}
\usepackage{amsmath, amsthm}
%\usepackage[utf8]{vietnam} 
%\usepackage[utf8]{inputenc}
%\usepackage[english,vietnam]{babel}

%\usepackage{fontspec}
%\setmainfont[Ligatures=TeX]{Linux Libertine O}
%\usepackage{polyglossia}
%\setmainlanguage{vietnamese}

%\usepackage[bitstream-charter]{mathdesign}
%\usepackage{eucal}
\usepackage{geometry}
\geometry{verbose,tmargin=2cm,bmargin=2cm,lmargin=2.5cm,rmargin=2.2cm,headheight=2.5cm}

%\usepackage[tracking]{microtype}
\usepackage[sc,osf]{mathpazo}   % With old-style figures and real %smallcaps.
\linespread{1.025}              % Palatino leads a little more leading
% Euler for math and numbers
\usepackage[euler-digits,small]{eulervm}


\usepackage{frcursive}
\usepackage{calligra}
\newcommand{\setfont}[2]{{\fontfamily{#1}\selectfont #2}}



\theoremstyle{plain}
\newtheorem{thm}{Theorem}
\newtheorem{ass}{Assumption}
\renewcommand{\theass}{}
\newtheorem{defn}{Definition}
\newtheorem{quest}{Question}
\newtheorem{com}{Comment}
\newtheorem{ex}{Example}
\newtheorem{lem}[thm]{Lemma}
\newtheorem{cor}[thm]{Corollary}
\newtheorem{prop}[thm]{Proposition}
\theoremstyle{remark}
\newtheorem{rem}{\bf{Remark}}
%\numberwithin{equation}{section}



\usepackage[framemethod=TikZ]{mdframed}
%\newcounter{theo}[section]\setcounter{theo}{0}

%\newcounter{theo}[]\setcounter{theo}{0}
%\renewcommand{\thetheo}{\arabic{section}.\arabic{theo}}
%\renewcommand{\thetheo}{\arabic{theo}}
\newenvironment{theo}[2][]{%
\refstepcounter{theo}%
\ifstrempty{#1}%
{\mdfsetup{%
frametitle={%
\tikz[baseline=(current bounding box.east),outer sep=0pt]
\node[anchor=east,rectangle,fill=blue!20]
%{\strut Theorem~\thetheo};}}
{\strut Question~\thetheo};}}
}%
{\mdfsetup{%
frametitle={%
\tikz[baseline=(current bounding box.east),outer sep=0pt]
\node[anchor=east,rectangle,fill=blue!20]
{\strut Theorem~\thetheo:~#1};}}%
}%
\mdfsetup{innertopmargin=10pt,linecolor=blue!20,%
linewidth=2pt,topline=true,%
frametitleaboveskip=\dimexpr-\ht\strutbox\relax
}
\begin{mdframed}[]\relax%
\label{#2}}{\end{mdframed}}



\begin{document}



\begin{center}
{\LARGE \textsc{Special data}}\\
%{Từ Nguyễn Thái Sơn}\\
%{\setfont{calligra}{Son Nguyen Thai Tu}}\\
%{\setfont{frc}{March 13, 2021}}\\
{\textit{April 09, 2021}}
\end{center}



\begin{center}
--------------------------------------------------------------------------------------------------------------------
\end{center}


\section{General case}
Let us consider $p = 2$ and $\delta_0>0$ such that $d(x) = \mathrm{dist}(x,\partial\Omega)$ for $x$ near the boundary such that $0<d(x)<\delta_0$. We extend $d(\cdot)\in \mathrm{C}^2(\mathbb{R}^n)$. We consider the equation:
\begin{equation}\label{eq:PDEeps}
    \begin{cases}
   \mathcal{L}[u^\varepsilon] =  u^\varepsilon(x) + |Du^\varepsilon(x)|^2 - f(x) - \varepsilon \Delta u^\varepsilon(x) = 0 \qquad
    \text{in}\;\Omega, \vspace{0cm}\\
    \displaystyle  \lim_{\mathrm{dist}(x,\partial \Omega)\to 0} u^\varepsilon(x) = +\infty.
    \end{cases} \tag{PDE$_\varepsilon$}
\end{equation}
Let $K_0 = \Vert d\Vert_{L^\infty}, K_1 = \Vert\nabla d\Vert_{L^\infty}$ and $K_2 = \Vert \Delta d\Vert_{L^\infty}$. We summarize some results from \cite{Lasry1989} in a clearer manner here for convenience. Let us without loss of generality assume $f\geq 0$ a.e. in $\Omega$ (\textit{positivity}). 
\begin{enumerate}
    \item If $f = 0$, we are good to go, as 
    \begin{equation}\label{e:spirit}
        u(x) = \frac{\nu C_\alpha\varepsilon^{\alpha+1}}{d(x)^\alpha} + \left(\frac{C\varepsilon^{\alpha+2}}{\delta_{0,\Omega}^{\alpha+2}}\right)
    \end{equation}
    is a supersolution. 
    \item If $f$ is compactly supported in $\Omega_\kappa$ we are good to go as well.
    \item Instead, if $f$ is a constant we should be able to do the same. Can we improve the result to the case that $f$ is constant in the strip $\Omega\backslash \overline{\Omega}_\kappa$?
    \item Now here is the deal, in the spirit of \eqref{e:spirit} above, we see that when we plug in the solution $u$ from \eqref{e:spirit} into the equation, we simply ignore the contribution of the monotone term $u(x)$, even though it is positive. That leads to the following observation, if we choose 
    \begin{equation*}
        f(x) =  \frac{\theta C_\alpha\varepsilon^{\alpha+1}}{d(x)^\alpha}, \qquad x\in \Omega
    \end{equation*}
    for any $\theta > 1$ then clearly
    \begin{equation*}
        \mathcal{L}^\varepsilon[\theta u] \geq 0
    \end{equation*}
    therefore $\theta u$ from \eqref{e:spirit} is still a supersolution. While if $\theta\leq 1$, we simply use $u$ as a supersolution. Furthermore, if 
    \begin{equation*}
    \fbox{$\displaystyle    f(x) = \frac{\theta C_\alpha\varepsilon^{\alpha+1}}{d(x)^\beta} $}
    \end{equation*}
    for any $0<\beta\leq \alpha$ this argument still works. Can we still have a rate of convergence?
\end{enumerate}





\bibliography{zzzzlibrary}{}
%\bibliographystyle{ieeetr}
\bibliographystyle{acm}
\end{document}

    
    
    
    