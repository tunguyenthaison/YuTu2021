\documentclass[english,reqno]{amsart}
%\renewcommand{\baselinestretch}{1.2}
\usepackage{hyperref}
\usepackage[margin=1in]{geometry}
\usepackage{graphicx}
\raggedbottom
\usepackage{subfig}
\usepackage{amssymb,amsbsy,amsmath,amsfonts,amssymb,amscd}
\usepackage{comment}
\usepackage{verbatim}
\usepackage[dvipsnames]{xcolor}
\usepackage{enumitem}
\usepackage[numbers]{natbib} % for alphabetically numbered lists
\usepackage{algorithm,algpseudocode}
\usepackage{setspace}
\newtheorem{theorem}{Theorem}[section]
\newtheorem{proposition}{Proposition}[section]
\newtheorem{remark}{Remark}[section]
\newtheorem{lemma}{Lemma}[section]
\newtheorem{assum}{Assumption}[section]
\newtheorem{cor}{Corollary}[section]

\title{Progress Report - Arp 2}

\begin{document}


\maketitle
Let $u^\epsilon$ solve the equation
\begin{equation}
\left\{
  \begin{aligned}
    u^\epsilon + |Du^\epsilon|^p -f(x) - \epsilon \Delta u^\epsilon &=0 \quad \, \text{in } \Omega, \\
            u^\epsilon &= \infty \quad \text{on } \partial \Omega,
  \end{aligned}
\right.
\end{equation}

and $u_0$ solve the equation
\begin{equation}
\left\{
  \begin{aligned}
    u_0 + |Du_0|^p -f(x) &\leq 0 \quad \, \text{in } \Omega, \\
             u_0 + |Du_0|^p -f(x) &\geq 0 \quad \text{on } \bar{\Omega}.
  \end{aligned}
\right.
\end{equation}

\begin{itemize}
    \item[1.] Before, we had the conclusion $ u \leq u^\epsilon$, which is in fact false. What we have is only $u \leq u^\epsilon +C\epsilon^\frac{1}{2}$. Let $w^\epsilon$ be the solution of 
    \begin{equation}
\left\{
  \begin{aligned}
     w^\epsilon\ + |Dw^\epsilon|^p -f(x) - \epsilon \Delta w^\epsilon  &=0 \quad \, \text{in } \Omega, \\
   w^\epsilon &= u \quad \text{on } \partial \Omega.
  \end{aligned}
\right.
\end{equation} 
By the estimate of vanishing viscosity problem with Dirichlet boundary condition, we have \begin{equation}
    |w^\epsilon - u| \leq C \epsilon ^\frac{1}{2}.
\end{equation}
Hence, we have 
\begin{equation}
    u \leq C\epsilon^\frac{1}{2}+w^\epsilon\leq C\epsilon^\frac{1}{2} + u^\epsilon
\end{equation}
by comparison principle.

\bigbreak
\bigbreak


\item[2.] Consider
\begin{equation}
\left\{
  \begin{aligned}
    L(\varphi_\alpha) = \varphi_\alpha + |D\varphi_\alpha|^p -f(x) - \alpha \Delta \varphi_\alpha &=0 \quad \, \text{in } \Omega_\epsilon, \\
   \varphi_\alpha &= C\epsilon \quad \text{on } \partial \Omega_\epsilon.
  \end{aligned}
\right.
\end{equation}

It is not clear to us why $\varphi_\epsilon \geq u_\epsilon$. If we fix $\epsilon$, $\displaystyle u_\epsilon \approx \frac{C_\alpha \epsilon^{\alpha+1}}{d(x)^\alpha} +M $ in the interior and we don't quite have $u_\epsilon \leq C\epsilon$ when $d(x)=\epsilon$ because of the constant $M$.


Note that $\varphi_0$ solves
\begin{equation}
\label{vis}
\left\{
  \begin{aligned}
    \varphi_0 + |D\varphi_0|^p -f(x) &=0 \quad \, \text{in } \Omega_\epsilon, \\
             \varphi_0 &= C\epsilon \quad \text{on } \partial \Omega_\epsilon.
  \end{aligned}
\right.
\end{equation}
We are still trying to see what is the relation between $u_0$ and $\varphi_0$ for now.

\bigbreak
\bigbreak

\item[3.]Suppose $f$ has compact support and we try to find the minimum of $u^\epsilon$. Suppose $u^\epsilon$ has min at $x_0$, then $\epsilon \Delta u^\epsilon(x_0) \geq 0$ and $$u^\epsilon(x_0) -f(x_0) - \epsilon \Delta u^\epsilon(x_0) =0 ,$$
which gives us $u^\epsilon(x_0) \geq f(x_0)$. This seems not very useful since we already know $u^\epsilon (x) \geq \min f$. Another attempt is to bound $\epsilon \Delta u^\epsilon(x_0)$ by Bernstein's method. However, this doesn't work out for now. After we get the equation for $\displaystyle w :=\frac{|Du^\epsilon|^2}{2}$, we don't know how to get a bound for $\epsilon \Delta u^\epsilon(x_0)$ from the equation of $\displaystyle w$.

\bigbreak
\bigbreak

    \item[4.] 
Let's assume $f$ is radial symmetric and $n=1$, then $u^\epsilon(x) = v(|x|)$ for some function $v$. Then $v$ solves
\begin{equation}
\left\{
  \begin{aligned}
    v(r) + |v^\prime (r)|^p -f(r) - \epsilon v^{\prime \prime} \left(\frac{n-1}{r}\right) &=0 \quad \, \text{in } (0,1), \\
            v &= \infty \quad \text{at  } x=1,
  \end{aligned}
\right.
\end{equation}

This ODE looks more complicated than the one of $u^\epsilon$.
We tried an example with 
\begin{equation}
f=
\left\{
  \begin{aligned}
     &1-|x| \quad \text{in } (-1,1), \\
          & 0 \quad \text{in  } (-2,2) \setminus (-1, 1),
  \end{aligned}
\right.
\end{equation}
and the boundary condition $u(0)=0$ and $u^\prime(0)=1$ in WolframAlpha. $u^\epsilon$ does blow up at $x=2$ and $x=-2$ according to the pictures. But we didn't manage to find out $u$ by manually computing. 

We also tried Bernstein's method for the radial symmetric case but didn't get anything useful for now. 

\item[5.] We also looked at the stochastic optimal control problem. We are still confused about the effect of $dB_t$ term on the path $X_t$.

\end{itemize}






\end{document}