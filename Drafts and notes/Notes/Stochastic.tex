\documentclass[10pt]{article}
\usepackage[T1]{fontenc}

\usepackage{xcolor}
\usepackage{fullpage}
\usepackage{mathrsfs}
\usepackage{amsmath, amsthm}
%\usepackage[utf8]{vietnam} 
%\usepackage[utf8]{inputenc}
%\usepackage[english,vietnam]{babel}

%\usepackage{fontspec}
%\setmainfont[Ligatures=TeX]{Linux Libertine O}
%\usepackage{polyglossia}
%\setmainlanguage{vietnamese}

%\usepackage[bitstream-charter]{mathdesign}
%\usepackage{eucal}
\usepackage{geometry}
\geometry{verbose,tmargin=2cm,bmargin=2cm,lmargin=2.5cm,rmargin=2.2cm,headheight=2.5cm}

%\usepackage[tracking]{microtype}
\usepackage[sc,osf]{mathpazo}   % With old-style figures and real %smallcaps.
% Euler for math and numbers
\usepackage[euler-digits,small]{eulervm}
\linespread{1.025}              % Palatino leads a little more leading



\usepackage{frcursive}
\usepackage{calligra}
\newcommand{\setfont}[2]{{\fontfamily{#1}\selectfont #2}}



\theoremstyle{plain}
\newtheorem{thm}{Theorem}
\newtheorem{ass}{Assumption}
\renewcommand{\theass}{}
\newtheorem{defn}{Definition}
\newtheorem{quest}{Question}
\newtheorem{com}{Comment}
\newtheorem{ex}{Example}
\newtheorem{lem}[thm]{Lemma}
\newtheorem{cor}[thm]{Corollary}
\newtheorem{prop}[thm]{Proposition}
\theoremstyle{remark}
\newtheorem{rem}{\bf{Remark}}
%\numberwithin{equation}{section}



\usepackage[framemethod=TikZ]{mdframed}
%\newcounter{theo}[section]\setcounter{theo}{0}

%\newcounter{theo}[]\setcounter{theo}{0}
%\renewcommand{\thetheo}{\arabic{section}.\arabic{theo}}
%\renewcommand{\thetheo}{\arabic{theo}}
\newenvironment{theo}[2][]{%
\refstepcounter{theo}%
\ifstrempty{#1}%
{\mdfsetup{%
frametitle={%
\tikz[baseline=(current bounding box.east),outer sep=0pt]
\node[anchor=east,rectangle,fill=blue!20]
%{\strut Theorem~\thetheo};}}
{\strut Question~\thetheo};}}
}%
{\mdfsetup{%
frametitle={%
\tikz[baseline=(current bounding box.east),outer sep=0pt]
\node[anchor=east,rectangle,fill=blue!20]
{\strut Theorem~\thetheo:~#1};}}%
}%
\mdfsetup{innertopmargin=10pt,linecolor=blue!20,%
linewidth=2pt,topline=true,%
frametitleaboveskip=\dimexpr-\ht\strutbox\relax
}
\begin{mdframed}[]\relax%
\label{#2}}{\end{mdframed}}



\begin{document}



\begin{center}
{\LARGE \textsc{A stochastic formula}}\\
%{Từ Nguyễn Thái Sơn}\\
%{\setfont{calligra}{Son Nguyen Thai Tu}}\\
%{\setfont{frc}{March 13, 2021}}\\
{\textit{March 28, 2021}}
\end{center}



\begin{center}
--------------------------------------------------------------------------------------------------------------------
\end{center}


\noindent
Let us consider $1<p<2$ and $\delta_0>0$ such that $d(x) = \mathrm{dist}(x,\partial\Omega)$ for $x$ near the boundary such that $0<d(x)<\delta_0$. Assume $f\in \mathrm{C}^1(\overline{\Omega})$, we consider
\begin{equation}\label{eq:PDEeps}
    \begin{cases}
   \mathcal{L}[u^\varepsilon] =  u^\varepsilon(x) + |Du^\varepsilon(x)|^p - f(x) - \varepsilon \Delta u^\varepsilon(x) = 0 \qquad
    \text{in}\;\Omega, \vspace{0cm}\\
    \displaystyle  \lim_{\mathrm{dist}(x,\partial \Omega)\to 0} u^\varepsilon(x) = +\infty.
    \end{cases} \tag{PDE$_\varepsilon$}
\end{equation}
Let $\alpha = \frac{2-p}{p-1}$ and $C_\alpha = \frac{1}{\alpha}(\alpha+1)^{\alpha+1}$. If $f\in \mathrm{C}^1(\overline{\Omega})$ then there exists a unique solution $u^\varepsilon\in \mathrm{C}^2(\Omega)$ such that 
\begin{equation*}
    u^\varepsilon(x)d(x)^\alpha \to C_\alpha \varepsilon^{\alpha+1}    \qquad\text{as}\qquad d(x)\to 0.
\end{equation*}
Unfortunately, obtaining a uniform convergence without the dependence on $\varepsilon$ seems to be not easy or even possible now.






%\bibliography{zzzzlibrary}{}
%\bibliographystyle{ieeetr}
%\bibliographystyle{acm}
\end{document}

    
    
    
    