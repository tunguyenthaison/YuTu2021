\documentclass[10pt]{article}
\usepackage[T1]{fontenc}

\usepackage{xcolor}
\usepackage{fullpage}
\usepackage{mathrsfs}
\usepackage{amsmath, amsthm}
%\usepackage[utf8]{vietnam} 
%\usepackage[utf8]{inputenc}
%\usepackage[english,vietnam]{babel}

%\usepackage{fontspec}
%\setmainfont[Ligatures=TeX]{Linux Libertine O}
%\usepackage{polyglossia}
%\setmainlanguage{vietnamese}

%\usepackage[bitstream-charter]{mathdesign}
%\usepackage{eucal}
\usepackage{geometry}
\geometry{verbose,tmargin=2cm,bmargin=2cm,lmargin=2.5cm,rmargin=2.2cm,headheight=2.5cm}

%\usepackage[tracking]{microtype}
\usepackage[sc,osf]{mathpazo}   % With old-style figures and real %smallcaps.
\linespread{1.025}              % Palatino leads a little more leading
% Euler for math and numbers
\usepackage[euler-digits,small]{eulervm}


\usepackage{frcursive}
\usepackage{calligra}
\newcommand{\setfont}[2]{{\fontfamily{#1}\selectfont #2}}



\theoremstyle{plain}
\newtheorem{thm}{Theorem}
\newtheorem{ass}{Assumption}
\renewcommand{\theass}{}
\newtheorem{defn}{Definition}
\newtheorem{quest}{Question}
\newtheorem{com}{Comment}
\newtheorem{ex}{Example}
\newtheorem{lem}[thm]{Lemma}
\newtheorem{cor}[thm]{Corollary}
\newtheorem{prop}[thm]{Proposition}
\theoremstyle{remark}
\newtheorem{rem}{\bf{Remark}}
%\numberwithin{equation}{section}



\usepackage[framemethod=TikZ]{mdframed}
%\newcounter{theo}[section]\setcounter{theo}{0}

%\newcounter{theo}[]\setcounter{theo}{0}
%\renewcommand{\thetheo}{\arabic{section}.\arabic{theo}}
%\renewcommand{\thetheo}{\arabic{theo}}
\newenvironment{theo}[2][]{%
\refstepcounter{theo}%
\ifstrempty{#1}%
{\mdfsetup{%
frametitle={%
\tikz[baseline=(current bounding box.east),outer sep=0pt]
\node[anchor=east,rectangle,fill=blue!20]
%{\strut Theorem~\thetheo};}}
{\strut Question~\thetheo};}}
}%
{\mdfsetup{%
frametitle={%
\tikz[baseline=(current bounding box.east),outer sep=0pt]
\node[anchor=east,rectangle,fill=blue!20]
{\strut Theorem~\thetheo:~#1};}}%
}%
\mdfsetup{innertopmargin=10pt,linecolor=blue!20,%
linewidth=2pt,topline=true,%
frametitleaboveskip=\dimexpr-\ht\strutbox\relax
}
\begin{mdframed}[]\relax%
\label{#2}}{\end{mdframed}}



\begin{document}



\begin{center}
{\LARGE \textsc{Large solution to quadratic quasilinear equations}}\\
%{Từ Nguyễn Thái Sơn}\\
%{\setfont{calligra}{Son Nguyen Thai Tu}}\\
%{\setfont{frc}{March 13, 2021}}\\
{\textit{March 21, 2021}}
\end{center}



\begin{center}
--------------------------------------------------------------------------------------------------------------------
\end{center}


\section{General case}
Let us consider $p = 2$ and $\delta_0>0$ such that $d(x) = \mathrm{dist}(x,\partial\Omega)$ for $x$ near the boundary such that $0<d(x)<\delta_0$. We extend $d(\cdot)\in \mathrm{C}^2(\mathbb{R}^n)$. We consider the equation:
\begin{equation}\label{eq:PDEeps}
    \begin{cases}
   \mathcal{L}[u^\varepsilon] =  u^\varepsilon(x) + |Du^\varepsilon(x)|^2 - f(x) - \varepsilon \Delta u^\varepsilon(x) = 0 \qquad
    \text{in}\;\Omega, \vspace{0cm}\\
    \displaystyle  \lim_{\mathrm{dist}(x,\partial \Omega)\to 0} u^\varepsilon(x) = +\infty.
    \end{cases} \tag{PDE$_\varepsilon$}
\end{equation}
Let $K_0 = \Vert d\Vert_{L^\infty}, K_1 = \Vert\nabla d\Vert_{L^\infty}$ and $K_2 = \Vert \Delta d\Vert_{L^\infty}$. We summarize some results from \cite{Lasry1989} in a clearer manner here for convenience. Let us without loss of generality assume $f\geq 0$ a.e. in $\Omega$ (\textit{positivity}). We look for the ansatz
\begin{equation*}
    u(x) = -\varepsilon\,\mathrm{log}(d(x)), \qquad x\in \Omega.
\end{equation*}
\subsection{Building supersolutions for bounded data}
Look for 
\begin{equation*}
\begin{split}
    w^+_\eta(x) &= -(1+\eta)\varepsilon\, \mathrm{log}(d(x))+C_\eta, \qquad x\in \Omega.
\end{split}
\end{equation*}
We have
\begin{equation*}
    \nabla w^+_\eta(x) = -(1+\eta)\varepsilon\frac{\nabla d(x)}{d(x)}, \qquad \text{and}\qquad \Delta w^+_\eta(x) =(1+\eta)\varepsilon \frac{|\nabla d(x)|^2}{d(x)^2} - (1+\eta)\varepsilon \frac{\Delta d(x)}{d(x)}.
\end{equation*}
\begin{equation*}
    \mathcal{L}\left[w^+_\eta\right] = -(1+\eta)\varepsilon\log(d(x))+C_\eta - f(x) + (1+\eta)\varepsilon^2\frac{|\nabla d(x)|^2}{d(x)^2} \underbrace{\Big((1+\eta) - 1+  \Delta d(x) d(x) \Big)}_{\eta + \Delta d(x)d(x)}.
\end{equation*}
\begin{itemize}
    \item If $d(x) \leq \frac{\eta}{K_2}$ then $\eta + \Delta d(x)d(x) \geq \eta - K_2 d(x) \geq 0$.
    \item If $d(x)\geq \frac{\eta}{K_2}$ then
\begin{equation*}
    \left|(1+\eta)\varepsilon^2\frac{|\nabla d(x)|^2}{d(x)^2} \Big((1+\eta) - 1+  \Delta d(x) d(x) \Big)\right| \leq \underbrace{\left((1+\eta) \frac{K_1^2K_2^2}{\eta^2}(\eta+K_0K_2)\right)}_{C_{1,\eta}}\varepsilon^2.
\end{equation*}
\end{itemize}
Therefore if we choose $C_\eta = \max f^+ + C_{1,\eta} \varepsilon^2$ then $w^+_{\eta}$ is a supersolution.

\begin{rem} \quad
\begin{itemize}
    \item If we choose $\eta = \sqrt{\varepsilon}$ then 
\begin{equation*}
    C_{1,\varepsilon} = \max f^+ + \underbrace{(1+\sqrt{\varepsilon})K_1^2K_2^2(\sqrt{\varepsilon}+K_0K_2)}_{C_1}\varepsilon.
\end{equation*}
I.e., the constant $C$ here is independent of $\eta = \sqrt{\varepsilon}$.
    \item If we choose $\eta = \varepsilon^\gamma$ for some $\gamma \ll 1$ then
    \begin{equation*}
    C_{1,\varepsilon} = \max f^+ + \underbrace{(1+\varepsilon^\gamma)K_1^2K_2^2(\varepsilon^\gamma+K_0K_2)}_{C_1}\varepsilon^{2(1-\gamma)} \leq \max f^+ + \underbrace{2(K_1K_2)^2(1+K_0K_2)}_{C_1}\varepsilon^{2(1-\gamma)}.
\end{equation*}
\end{itemize}
\end{rem}

\begin{cor} If $f\geq 0$ is bounded then $w^+(x) = -(1+\varepsilon^\gamma)\varepsilon \log(d(x)) +\max f^+ + C_1\varepsilon^{2(1-\gamma)}$ is a supersolution. In other words, we have
\begin{equation*}
    u^\varepsilon(x) - \varepsilon(1+\varepsilon^\gamma) \log\left(\frac{1}{d(x)}\right) \leq  \max f^+ +C_1\varepsilon^{2(1-\gamma)}.
\end{equation*}
\end{cor}



%\subsection{Building subsolutions}


\subsection{Doubling variable and vanishing viscosity when $|f(x)|\leq C\varepsilon$}
Let $\gamma,\alpha\in (0,1)$ arbitrary, we recall that $\varepsilon(1+\varepsilon)\leq \varepsilon(1+\varepsilon^\alpha)$ for all $\alpha > 0$ since $\varepsilon\in (0,1)$ and also
\begin{equation*}
    \lim_{d(x)\to 0^+} d(x)^\alpha \log\left(\frac{1}{d(x)}\right) = 0 \qquad \text{for all}\;\alpha >0.
\end{equation*}
Therefore we can always find $r_\alpha>0$ such that $d(x)<r_\alpha$ implies
 \begin{equation*}
     \log\left(\frac{1}{d(x)}\right) \leq \frac{1}{d(x)^\alpha}.
 \end{equation*}
Consider
\begin{equation*}
    \Phi(x,y) = u^\varepsilon(x)- u(y) - \frac{C_0|x-y|^2}{\sigma} - \frac{(1+\varepsilon^\gamma+\delta^\alpha)\varepsilon}{d(x)^\alpha} , \qquad (x,y)\in \overline{\Omega}\times\overline{\Omega}.
\end{equation*}
We see that if $d(x)<r_\alpha$ then
\begin{equation*}
    u^\varepsilon(x) - \frac{(1+\varepsilon^\gamma+\delta^\alpha)\varepsilon}{d(x)^\alpha}  \leq \left[u^\varepsilon(x) - (1+\varepsilon^\gamma)\varepsilon \log\left(\frac{1}{d(x)}\right)\right] - \frac{\delta^\alpha \varepsilon}{d(x)^\alpha} \leq  C\varepsilon + C_1\varepsilon^{2(1-\gamma)} - \frac{\delta^\alpha\varepsilon}{d(x)^\alpha}.
\end{equation*}
Therefore $\Phi(x,y)\to -\infty$ as $x\to \partial \Omega$. Assume it has maximum at $(x_\sigma,y_\sigma)\in \Omega\times \overline{\Omega}$. From the estimate $\Phi(x_\sigma,y_\sigma) \geq \Phi(x_0,x_0)$ where $d(x_0) = K_0 = \max d(\cdot)$ we obtain
\begin{equation*}
    \left(u^\varepsilon(x_\sigma) - \frac{(1+\varepsilon^\gamma+\delta^\alpha)\varepsilon}{d(x_\sigma)^\alpha}\right)  - u(y_\sigma) \geq u^\varepsilon(x_0) - u(x_0) - \frac{(1+\varepsilon^\gamma+\delta^\alpha)\varepsilon}{K_0} \geq -C\varepsilon
\end{equation*}
Here using the fact that $u^\varepsilon(x_0)\geq 0$ and $|u(x_0)|\leq C\varepsilon$ and $u(y_\sigma) \geq 0$ we have
\begin{equation*}
    \frac{\delta^\alpha \varepsilon}{d(x_\sigma)^\alpha} \leq C\left(\varepsilon+\varepsilon^{2(1-\gamma)}\right) \qquad\Longrightarrow\qquad\frac{\delta^\alpha}{d(x_\sigma)^\alpha} \leq  C\left(1+\varepsilon^{1-2\gamma}\right) \leq C
\end{equation*}
fir $\gamma \leq \frac{1}{2}$. We deduce that
 \begin{equation*}
     d(x_\sigma)^\alpha \geq C\delta^\alpha \qquad\Longrightarrow\qquad d(x_\sigma) \geq C^{1/\alpha}\delta.
 \end{equation*}
This constant $C^{1/\alpha}$ may blow up as $\alpha\to 0$. Using $\Phi(x_\sigma,y_\sigma)\geq \Phi(x_\sigma,y_\sigma)$ we have
\begin{equation*}
    |x_\sigma - y_\sigma|\leq \sigma.
\end{equation*}
Now $x\mapsto \Phi(x,y_\sigma)$ has a max at $x_\sigma\in \Omega$ gives us
\begin{equation}\label{e:sub1}
\begin{split}
    u^\varepsilon(x_\sigma) &+ \left|\frac{2C_0(x_\sigma-y_\sigma)}{\sigma} - \frac{\alpha(1+\varepsilon^\gamma+\delta^\alpha)\varepsilon \nabla d(x_\sigma)}{d(x_\sigma)^{\alpha+1}}\right|^2 - f(x_\sigma) \\
    &-\varepsilon \left(\frac{2nC_0}{\sigma} + \frac{\alpha(\alpha+1)(1+\varepsilon^\gamma+\delta^\alpha)\varepsilon |\nabla d(x_\sigma)|^2}{d(x_\sigma)^{\alpha+2}} - \frac{\alpha(1+\varepsilon^\gamma + \delta^\alpha)\varepsilon\Delta d(x_\sigma)}{d(x_\sigma)^{\alpha+1}}\right) \leq 0.
\end{split}
\end{equation}
Now $x\mapsto \Phi(x,y_\sigma)$ has a max at $x_\sigma\in \Omega$ gives us
\begin{equation}\label{e:super1}
    u(y_\sigma) + \left|\frac{2C_0(x_\sigma-y_\sigma)}{\sigma}\right|^2  - f(y_\sigma)\geq 0.
\end{equation}
From \eqref{e:sub1} and \eqref{e:super1} we have
\begin{align*}
    u^\varepsilon(x_\sigma) - u(y_\sigma)& \leq \frac{K_1\alpha(1+\varepsilon^\gamma+\delta^\alpha)\varepsilon}{d(x_\sigma)^{\alpha+1}}\left(2C_0+ \frac{K_1\alpha(1+\varepsilon^\gamma+\delta^\alpha)\varepsilon}{d(x_\sigma)^{\alpha+1}}\right) + f(x_\sigma) - f(y_\sigma)\\
    & \qquad\qquad +\frac{2nC_0\varepsilon}{\sigma} + \frac{\alpha(\alpha+1)(1+\varepsilon^\gamma+\delta^\alpha)K_1^2\varepsilon^2}{d(x_\sigma)^{\alpha+2}} + \frac{\alpha (1+\varepsilon^\alpha+\delta^\alpha)K_2 \varepsilon^2}{d(x_\sigma)^{\alpha+1}}\\
    &\leq \frac{C\alpha \varepsilon}{\delta^{\alpha+1}}\left(C+\frac{C\alpha \varepsilon}{\delta^{\alpha+1}}\right) + C\varepsilon  + C\varepsilon + \frac{C\alpha \varepsilon^2}{\delta^{\alpha+2}} + \frac{C\alpha \varepsilon^2}{\delta^{\alpha+1}}
\end{align*}
by choosing $\sigma = 1$ and $C$ depends only on $\alpha$. Let $\delta = \varepsilon^\kappa$ then
\begin{equation*}
    u^\varepsilon(x_\sigma) - u(y_\sigma) \leq C\alpha \varepsilon^{1-\kappa(\alpha+1)} + C\alpha^2 \varepsilon^{2(1-\kappa(\alpha+1))} + C\varepsilon + C\alpha \varepsilon^{2-\kappa(\alpha+2)} + C\alpha \varepsilon^{2-\kappa(\alpha+1)}.
\end{equation*}
Now $\Phi(x,x)\leq \Phi(x_\sigma,y_\sigma)$ gives us
\begin{equation*}
    u^\varepsilon(x) - u(x)\leq C \Big(\alpha \varepsilon^{1-\kappa(\alpha+1)} + \alpha^2 \varepsilon^{2(1-\kappa(\alpha+1))} + \varepsilon + \alpha \varepsilon^{2-\kappa(\alpha+2)} + \alpha \varepsilon^{2-\kappa(\alpha+1)} \Big) + \frac{3\varepsilon}{d(x)^\alpha}.
\end{equation*}
We deduce that
\begin{equation*}
    0\leq u^\varepsilon(x) - u(x)\leq C\varepsilon^{1-\kappa(\alpha+1)} + \frac{3\varepsilon}{d(x)^\alpha}.
\end{equation*}
Send $\kappa\to 0$ we deduce that
\begin{equation*}
    0\leq u^\varepsilon(x) - u(x)\leq \left(C(\alpha) +\frac{3}{d(x)^\alpha}\right)\varepsilon.
\end{equation*}

\section{The Hopf-Cole transformation}
By assuming $f\geq 0$, we have $u^\varepsilon\geq 0$. Let us define
\begin{equation*}
    v^\varepsilon(x) = e^{-\frac{u^\varepsilon(x)}{\varepsilon}}, \qquad x\in \Omega.
\end{equation*}
It is clear that $v^\varepsilon\in \mathrm{C}^2(\Omega)$ and since $u^\varepsilon(x) = +\infty$ on $\partial\Omega$, we have $v^\varepsilon(x) = 0$ on $\partial\Omega$, hence
\begin{equation*}
    \begin{cases}
    u^\varepsilon(x)v^\varepsilon(x) - f(x)v^\varepsilon(x) +\varepsilon^2 \Delta v^\varepsilon(x)= 0 &\qquad\text{in}\;\Omega\\
    v^\varepsilon(x) = 0 &\qquad\text{on}\;\partial\Omega.
    \end{cases}
\end{equation*}

\section{Special case}
If $\Omega = B(0,1)$ then $d(x) = 1-|x|$, hence 
\begin{equation*}
    \nabla d(x) = \frac{x}{|x|} \qquad\text{and}\qquad \Delta d(x) = \frac{1-n}{|x|} \qquad \text{for}\;x\in \Omega\backslash \{0\}.
\end{equation*}
We have $d(0)=1$.
\bibliography{zzzzlibrary}{}
%\bibliographystyle{ieeetr}
\bibliographystyle{acm}
\end{document}

    
    
    
    