\documentclass[11pt,reqno]{amsart}
%============%============%============%============%

%============%============%============%============%
%\setlength{\columnseprule}{0.4pt}
%\setlength{\topmargin}{0cm}
%\setlength{\oddsidemargin}{.25cm}
%\setlength{\evensidemargin}{.25cm}
%\setlength{\textheight}{22.5cm}
%\setlength{\textwidth}{15.5cm}
\renewcommand{\baselinestretch}{1.05}
%============%============%============%============%
\usepackage[toc,page]{appendix}
%\setcounter{secnumdepth}{4}
%\setcounter{tocdepth}{3}
%============%============%============%============%
%\usepackage{romannum}
\usepackage{xcolor}
\usepackage{placeins}
\usepackage{amsfonts,amsmath,amsthm}
\usepackage{amssymb,epsfig}
\usepackage{enumerate} 
\usepackage[notcite,notref]{showkeys}
\usepackage{fullpage}
%============%============%============%============%
\usepackage[utf8]{inputenc}
\usepackage{mathpazo}

%\usepackage[libertine,cmintegrals,cmbrac,vvarbb]{newtxmath}
\usepackage{eucal}
%\usepackage[utopia]{mathdesign}
\usepackage[euler-digits]{eulervm}
\usepackage[unicode=true]{hyperref}
\hypersetup{colorlinks = true}
%\hypersetup{hidelinks=true}
\hypersetup{
     colorlinks,
     linkcolor={black!10!red},
     linkbordercolor = {black!100!red},
%    <your other options...>,
     citecolor={blue}
}





%graphic
%\usepackage[text={425pt,650pt},centering]{geometry}

\usepackage{pdfsync}

\usepackage{geometry}
\geometry{verbose,tmargin=2.5cm,bmargin=2.5cm,lmargin=2.5cm,rmargin=2.5cm,headheight=3.5cm}

\usepackage{graphicx}
\usepackage{epsfig}
\usepackage{tikz}
\usepackage{caption}
\usepackage{color} %color
\definecolor{vert}{rgb}{0,0.6,0}

\usepackage{comment}
\numberwithin{figure}{section}
%\pagestyle{plain}


\theoremstyle{plain}
\newtheorem{thm}{Theorem}[section]
\newtheorem{ass}{Assumption}
\renewcommand{\theass}{}
\newtheorem{defn}{Definition}
\newtheorem{quest}{Question}
\newtheorem{com}{Comment}
\newtheorem{ex}{Example}
\newtheorem{lem}[thm]{Lemma}
\newtheorem{cor}[thm]{Corollary}
\newtheorem{prop}[thm]{Proposition}
\theoremstyle{remark}
\newtheorem{rem}{\bf{Remark}}
\numberwithin{equation}{section}



%\renewcommand{\thefootnote}{\fnsymbol{footnote}}




%Characters -- Shortcuts
\newcommand{\E}{\mathbb{E}}
\newcommand{\M}{\mathbb{M}}
\newcommand{\N}{\mathbb{N}}
\newcommand{\bP}{\mathbb{P}}
\newcommand{\R}{\mathbb{R}}
\newcommand{\bS}{\mathbb{S}}
\newcommand{\T}{\mathbb{T}}
\newcommand{\Z}{\mathbb{Z}}
\newcommand{\bfS}{\mathbf{S}}
\newcommand{\cA}{\mathcal{A}}
\newcommand{\cB}{\mathcal{B}}
\newcommand{\cC}{\mathcal{C}}
\newcommand{\cF}{\mathcal{F}}
\newcommand{\cH}{\mathcal{H}}
\newcommand{\cL}{\mathcal{L}}
\newcommand{\cM}{\mathcal{M}}
\newcommand{\cP}{\mathcal{P}}
\newcommand{\cS}{\mathcal{S}}
\newcommand{\cT}{\mathcal{T}}
\newcommand{\cE}{\mathcal{E}}
\newcommand{\I}{\mathrm{I}}


%Functional spaces
\newcommand{\AC}{{\rm AC\,}}
\newcommand{\ACl}{{\rm AC}_{{\rm loc}}}
\newcommand{\BUC}{{\rm BUC\,}}
\newcommand{\USC}{{\rm USC\,}}
\newcommand{\LSC}{{\rm LSC\,}}
\newcommand{\Li}{L^{\infty}}
\newcommand{\Lip}{{\rm Lip\,}}
\newcommand{\W}{W^{1,\infty}}
\newcommand{\Wx}{W_x^{1,\infty}}


%Domains
\newcommand{\bO}{\partial\Omega}
\newcommand{\cO}{\overline\Omega}
\newcommand{\Q}{\mathbb{T}^{n}\times(0,\infty)}
\newcommand{\iQ}{\mathbb{T}^{n}\times\{0\}}
\newcommand{\cQ}{\mathbb{T}^{n}\times[0,\infty)}


%Greek alphabets -- Shortcuts
\newcommand{\al}{\alpha}
\newcommand{\gam}{\gamma}
\newcommand{\del}{\delta}
\newcommand{\ep}{\varepsilon}
\newcommand{\kap}{\kappa}
\newcommand{\lam}{ }
\newcommand{\sig}{\sigma}
\newcommand{\om}{\omega}
\newcommand{\Del}{\Delta}
\newcommand{\Gam}{\Gamma}
\newcommand{\Lam}{ }
\newcommand{\Om}{\Omega}
\newcommand{\Sig}{\Sigma}



%Overlines, Underlines -- Shortcuts
\newcommand{\ol}{\overline}
\newcommand{\ul}{\underline}
\newcommand{\pl}{\partial}
\newcommand{\supp}{{\rm supp}\,}
\newcommand{\inter}{{\rm int}\,}
\newcommand{\loc}{{\rm loc}\,}
\newcommand{\co}{{\rm co}\,}
\newcommand{\diam}{{\rm diam}\,}
\newcommand{\diag}{{\rm diag}\,}
\newcommand{\dist}{{\rm dist}\,}
\newcommand{\Div}{{\rm div}\,}
\newcommand{\sgn}{{\rm sgn}\,}
\newcommand{\tr}{{\rm tr}\,}
\newcommand{\Per}{{\rm Per}\,}

\newcommand{\rmC}{\mathrm{C}}
\newcommand{\rup}{\rightharpoonup}


\renewcommand{\subjclassname}{%
\textup{2010} Mathematics Subject Classification} 

%Hyperlink in PDF file
%\usepackage[dvipdfm,
%  colorlinks=false,
%  bookmarks=true,
%  bookmarksnumbered=false,
%  bookmarkstype=toc]{hyperref}
%\makeatletter
%\def\@pdfm@dest#1{%
%  \Hy@SaveLastskip
%  \@pdfm@mark{dest (#1) [@thispage /\@pdfview\space @xpos @ypos null]}%
%  \Hy@RestoreLastskip
%}


%BibLatex
%\usepackage[
%backend=biber,
%style=alphabetic,
%sorting=ynt
%]{biblatex}
%\addbibresource{rate.bib}

%%%%%%%%%%%%%%%%%%%%%%%%%%%%%%%%%%%%%%%%%%%%%%%%%%%%%%%%%%%%%%%%%%%%%%%%%%%%%%%%%%%%%%%%%%%%%%%%%%%%%%%%%%%%%%%%%%%%%%%%%%%%%%%%%%%%%%%%


%%%%%%%%%%%%%%%%%%%%%%%%%%%%%%%%%%%%%%%%%%%%%%%%%%%%%%%%%%%
\usepackage{import}
\usepackage{xifthen}
\usepackage{pdfpages}
\usepackage{transparent}
\newcommand{\incfig}[1]{%
    \def\svgwidth{\columnwidth}
    \import{./figs/}{#1.pdf_tex}
}
%%%%%%%%%%%%%%%%%%%%%%%%%%%%%%%%%%%%%%%%%%%%%%%%%%%%%%%%%%%
\begin{document}
\title[Rate of convergence]
{\textsc{Remarks on the vanishing viscosity of state-constraint Hamilton--Jacobi equations}}
\thanks{The authors is supported in part by NSF grant DMS-1664424 and NSF CAREER grant DMS-1843320
to Hung Vinh Tran. The work of Son N. T. Tu is supported in part by the GSSC Fellowship, University of Wisconsin--Madison.}
\begin{abstract}
We investigate qualitatively the convergence of large solution (state-constraint solution) to the subquadratic quasilinear elliptic equation as the viscosity vanishes. We show by a simple proof that for Lipschitz data that vanish on the boundary, the rate of convergence is at least $\mathcal{O}(\sqrt{\varepsilon})$ with a specific boundary layer. The rate can be improved in some region to $\mathcal{O}(\varepsilon)$, for instance, on the whole domain with zero data. The proof relies on a domain decomposition technique and classical tools in the viscosity solution theory.
\end{abstract}
%%%%%%%%%%%%%%%%%%%%%%%%%%%%%%%%%%%%%%%%%%%%%%%%%%%%%%%%%%%
\author{Yuxi Han}
\address[Y. Han]
{
Department of Mathematics, 
University of Wisconsin Madison, 480 Lincoln  Drive, Madison, WI 53706, USA}
\email{yuxi.han@wisc.edu}
\author{Son N. T. Tu}
\address[S. N.T. Tu]
{
Department of Mathematics, 
University of Wisconsin Madison, 480 Lincoln  Drive, Madison, WI 53706, USA}
\email{thaison@math.wisc.edu}
\date{\today}
\keywords{first-order Hamilton--Jacobi equations; state-constraint problems; optimal control theory; rate of convergence; viscosity solutions.}
\subjclass[2010]{
35B40, %Asymptotic behavior of solutions, 
35D40, %Viscosity solutions
49J20, %Optimal control problems involving partial differential equations
49L25, %Viscosity solutions
70H20 %Hamilton-Jacobi equations
}
\maketitle
\setcounter{tocdepth}{1}
\tableofcontents

%%%%%%%%%%%%%%%%%%%%%%%%%%%%%%%%%%%%%%%%%%%%%%%%%%%%%%%%%%%
\section{Introduction}\label{sec:intro}
\subsection{Motivation} Let $\Omega$ be an open, bounded and connected domain of $\mathbb{R}^n$ with $\mathrm{C}^2$ boundary, %Let us consider the following Hamiltonian $ H(x,\varrho) = |\varrho|^p-f(x)$ for $(x,p)\in \overline{\Omega}\times \mathbb{R}^n$,
 $f\in \mathrm{C}(\overline{\Omega})\cap W^{1,\infty}(\Omega)$, and $u^\varepsilon\in \mathrm{C}^2(\Omega)$ (see \cite{Lasry1989}) be the solution to
 \begin{equation}\label{eq:PDEepsa}
    \begin{cases}
      u^\varepsilon(x) + H(Du^\varepsilon(x)) - f(x) - \varepsilon \Delta u^\varepsilon(x) = 0 \qquad
    \text{in}\;\Omega, \vspace{0cm}\\
    \displaystyle  \lim_{\mathrm{dist}(x,\partial \Omega)\to 0} u^\varepsilon(x) = +\infty
    \end{cases} 
\end{equation}
 where $H:\R^n\to\R^n$ is a continuous Hamiltonian. A typical subquadratic Hamiltonian that has been considered in the literature is $H(\xi) = |\xi|^p$ for $\xi\in \R^n$ where $1<p\leq 2$. For simplicity of the presentation, we will consider this special case.  
 
 %and generalize the result later to all Hamiltonian $H(\xi)$ for which (see \cite{sardarli_ergodic_2021} for a similar generalization)
 %\begin{equation*}
 %    \lim_{\delta\to 0}\delta^{\frac{p}{p-1}}H\left(\delta^{-\frac{1}{p-1}}\xi\right) = |\xi|^p \qquad\text{locally uniformly in}\;\xi\in \R^n.
 %\end{equation*}
 \noindent 
 With $H(\xi) = |\xi|^p$, the equation of interest then becomes
\begin{equation}\label{eq:PDEeps}
    \begin{cases}
      u^\varepsilon(x) + |Du^\varepsilon(x)|^p - f(x) - \varepsilon \Delta u^\varepsilon(x) = 0 \qquad
    \text{in}\;\Omega, \vspace{0cm}\\
    \displaystyle  \lim_{\mathrm{dist}(x,\partial \Omega)\to 0} u^\varepsilon(x) = +\infty.
    \end{cases} \tag{PDE$_\varepsilon$}
\end{equation}
When $1<p\leq 2$, equation \eqref{eq:PDEeps} describes the value function associated with a minimization problem in stochastic optimal control with state constraint. We are interested in studying the asymptotic behavior of $\{u^\varepsilon\}_{\varepsilon>0}$ as $\varepsilon\rightarrow 0^+$. Heuristically, the solution of the second-order state-constraint problem converges to that of a first-order state-constraint problem associated with deterministic optimal control, namely,
\begin{equation}\label{eq:PDE0}
    \begin{cases}
       u(x) + |Du(x)|^p - f(x) \leq 0\;\qquad\text{in}\;\Omega,\\
       u(x) + |Du(x)|^p - f(x) \geq 0\;\qquad\text{on}\;\overline{\Omega},
    \end{cases} \tag{PDE$_0$}
\end{equation}
as is described in the framework of viscosity solution.
Equation \eqref{eq:PDE0} admits a unique viscosity solution in the space $\rmC(\overline{\Omega})$, which is also the maximal viscosity subsolution among all the viscosity subsolutions in $\rmC(\overline{\Omega})$. The problem is interesting since in the limit we no longer have blowing up behavior. In this paper, we investigate the rate of convergence of $u^\varepsilon \to u$ as $\varepsilon\to 0^+$. A boundary layer is expected to describe the behavior of this convergence near the boundary.
\subsection{Assumptions} We will always assume $\Omega$ is an open, bounded and connected domain in $\mathbb{R}^n$ with $\mathrm{C}^2$ boundary satisfying either %$H:\mathbb{R}^n\times \mathbb{R}^n$ is a continuous Hamiltonian. 
%\begin{itemize}
%    \item[$\mathrm{(A1)}$] $H(x,\varrho) = |\varrho|^p - f(x)$ where $1<p\leq 2$ and $f\in \mathrm{C}(\overline{\Omega})\cap W^{1,\infty}(\Omega)$.
%\end{itemize}
%    The following assumptions on $\Omega$ will also be used.
\begin{itemize}    
    \item[(A1)] There exists a universal pair of positive numbers $(r,h)$ and $\eta\in \mathrm{BUC}(\overline{\Omega};\mathbb{R}^n)$ such that $B(x+t\eta(x), rt)\subset\Omega$ for all $t\in (0,h]$.
    \item[(A2)] $\Omega$ is a bounded star-shaped (with respect to the origin) open subset of $\mathbb{R}^n$ such that
    \begin{equation*}
        \mathrm{dist}(x,\overline{\Omega}) \geq \kappa r \qquad\text{for all}\; x\in (1+r) \partial\Omega, \;\text{for all}\;r>0,
    \end{equation*}
for some $\kappa > 0$.
\end{itemize}
Equation \eqref{eq:PDEeps} follows the setting of \cite{Lasry1989}, where the specific structure of the Hamiltonian $H(x,\xi) = |\xi|^p - f(x)$ enables more explicit estimates on the solution of \eqref{eq:PDEeps}. Assumption $\mathrm{(A1)}$ or $\mathrm{(A2)}$ guarantees that a comparison principle holds for the state-constraint problem \eqref{eq:PDE0}. We note that $\mathrm{(A1)}$ was first studied by M. Soner in \cite{Soner1986} (and $\mathrm{(A1)}$ holds for any smooth bounded domain), while $\mathrm{(A2)}$ was posed and studied by I. Capuzzo-Dolcetta and P. L. Lions in \cite{Capuzzo-Dolcetta1990}. 
%We list the assumptions on a general Hamiltonian as follow.
%\begin{itemize}
%    \item[$\mathrm{(H1)}$] There exists $C_1 > 0$ such that $H(x,p) \geq -C_1$ for all $(x,p)\in \overline{\Omega}\times\mathbb{R}^n$.
%    \item[$\mathrm{(H2)}$] There exists $C_2>0$ such that $|H(x,0)|\,\leq \,C_2$ for all $(x,p)\in \overline{\Omega}\times \mathbb{R}^n$.
%    \item[$\mathrm{(H3)}$] For each $R>0$ there exists a modulus $\omega_{R}[0,\infty)\to [0,\infty)$ such that $\omega_R(0^+) = 0$ and 
%    \begin{equation*}
%        \begin{cases}
%        |H(x,p) - H(y,p)| \leq \omega_R(|x-y|),\\
%        |H(x,p) - H(x,q)| \leq \omega_R(|p-q|),
%        \end{cases} \qquad\text{for all}\;x,y\in \overline{\Omega}, p,q \in \mathbb{R}^n\;\text{with}\;|p|,|q|\leq R.
%    \end{equation*}
%    \item[$\mathrm{(H4)}$] $H(x,p)\rightarrow \infty$ as $|p|\to \infty$ uniformly in $x\in \overline{\Omega}$.
%\end{itemize}

\subsection{Literature} There is a
vast amount of work in the literature on large solutions and second-order viscosity solutions with state constraints. The problem \eqref{eq:PDEeps} is first studied in \cite{Lasry1989} and subsequently many works have been done in understanding deeper the properties of solutions. We refer the readers to \cite{alessio_asymptotic_2006} and the references therein. 

In terms of rate of convergence, up to our knowledge such a question has not been studied in the literature. For the case \eqref{eq:PDEeps} is equipped with Dirichlet boundary condition, a rate $\mathcal{O}(\sqrt{\varepsilon})$ is well known with multiple proofs (see \cite{tran_hamilton-jacobi_2021}). 

\subsection{Main results} The main result of the paper is the following theorem.


\begin{thm} Let $\Omega$ be an open, bounded and connected subset of $\R^n$ with $\mathrm{C}^2$ boundary. Assume that $1 < p\leq 2$ and $f$ is Lipschitz with the condition $f = 0$ on $\partial\Omega$. Let $u^\varepsilon$ be the unique solution to \eqref{eq:PDEeps} and $u$ be the unique solution to \eqref{eq:PDE0}. Then there exists a constant $C$ such that for $x\in \Omega$,
\begin{align*}
    &C\sqrt{\varepsilon}\leq  u^\varepsilon(x) - u(x)\leq C\sqrt{\varepsilon} + C\varepsilon \left(\frac{\varepsilon}{d(x)}\right)^{\frac{2-p}{p-1}} &\qquad\text{if}\; p < 2,\\
    &C\sqrt{\varepsilon}\leq  u^\varepsilon(x) - u(x)\leq C\sqrt{\varepsilon} + \varepsilon|\log(d(x))| &\qquad\text{if}\; p = 2.
\end{align*}
\end{thm}

\begin{rem} The condition $f = 0$ on $\partial\Omega$ can be explained as follows. One can see that the solution to \eqref{eq:PDEeps} is continuous with respect to data $f$ in the weak$^*$ topology of $L^\infty(\Omega)$ (see \cite{Lasry1989}). If $f = 0$ on $\partial\Omega$, we are able to approximate $f$ \emph{uniformly} in $L^\infty(\Omega)$ by a sequence of compactly supported functions, where a rate of $u^\varepsilon-u$ is easier to obtained. Indeed, using the usual doubling variable technique, it is natural to consider the doubling variable for
\begin{equation*}
    u^\varepsilon(x) - \frac{C\varepsilon^{\alpha+1}}{d(x)^{\alpha}}
\end{equation*}
and $u(x)$, where $C\varepsilon^{\alpha+1}d(x)^{-\alpha}$ is the leading order term in the asymptotic expansion of $u^\varepsilon(x)$ as $\mathrm{dist}(x,\partial\Omega)\to 0$. We can see a \emph{correct scale} heuristically happens around $d(x) \approx \varepsilon$. However, to control the extra leading term above, one needs to have $d(x)\approx \varepsilon^\gamma$ for $\gamma < 1$ to make the argument work. This major difficulty can be overcome when we decompose the solution into an inside part and an outside part that lies in an annulus, where we have reduced to the case of $f=0$.
\end{rem}

\section{Preliminaries}\label{sec:prelim} 
\subsection{Setting and simplifications} Let $\Omega$ be an open, bounded and connected subset of $\mathbb{R}^n$ with boundary $\partial\Omega$ of class $C^2$. For small $\delta>0$, we denote $\Omega_\delta = \{x\in \Omega: \mathrm{dist}(x,\Omega) > \delta\}$ and $\Omega^\delta = \{x\in \mathbb{R}^n: \mathrm{dist}(x,\overline{\Omega}) < \delta\}$. 
\begin{figure}[ht]
    \centering
    %\incfig{Domains}
    \def\svgwidth{0.47\columnwidth}
    \import{./figs/}{Domains.pdf_tex}
    \caption{The domain $\Omega$ variations $\Omega_\delta, \Omega^\delta$.}
    \label{fig:Domains}
\end{figure}

\begin{defn} Define
\begin{equation}\label{def:delta_0}
    \delta_{0,\Omega} =\frac{1}{2}\sup \big\{ \delta > 0: x\mapsto\mathrm{dist}(x,\partial\Omega)\;\text{is}\;\rmC^2\;\text{in}\;\Omega\backslash\overline{\Omega}_{\delta} \big\}.
\end{equation}
%The distance function is of class $\mathrm{C}^2$ in the region where $0<\mathrm{dist}(x,\partial\Omega) < \delta_0$. 
We will simply write $\delta_0$ instead of $\delta_{0,\Omega}$ if the underlying domain is understood.
\end{defn}

We refer the readers to \cite{gilbarg_elliptic_2001} for the regularity of the distance function defined in the above definition. We then extend $\mathrm{dist}(x,\partial\Omega)$ to a function $d(x)\in \mathrm{C}^2(\mathbb{R}^n)$ such that 
\begin{equation}\label{e:distance_def}
    \begin{cases}
    d(x)\geq 0\;\text{for}\;x\in\Omega\;\text{with}\;d(x) = +\mathrm{dist}(x,\partial\Omega)\;\text{for}\;x\in \Omega\backslash \Omega_{\delta_0},\\
    d(x)\leq 0\;\text{for}\;x\notin \Omega\;\text{with}\;d(x) = -\mathrm{dist}(x,\partial\Omega)\;\text{for}\;x\in \Omega^{\delta_0}\backslash \Omega.
    \end{cases}
\end{equation}
Recall that $|D d(x)| = 1$ in the viscosity sense (and thus in classical sense) in $\Omega^{\delta_0}\backslash \Omega_{\delta_0}$. Let 
\begin{equation}\label{boundond}
   K_0:= \max_{x\in \overline{\Omega}}|d(x)|, \qquad K_1 := \max_{x\in \overline{\Omega}} |D d(x)| \qquad\text{and}\qquad K_2 := \max_{x\in \overline{\Omega}} |\Delta d(x)|.
\end{equation}
\noindent We denote by $\mathcal{L}^\varepsilon:\rmC^2(\Omega)\to \rmC(\Omega)$ the operator
\begin{equation*}
    \mathcal{L}^\varepsilon[u](x) :=   u(x) + |Du(x)|^p - f(x) - \varepsilon \Delta u(x), \qquad x\in \Omega.
\end{equation*}



\subsection{Local gradient estimate} 
The following theorem is about the local gradient bound as in \cite[Appendix]{Lasry1989}. The technique being used is the classical Bernstein's method. Let $\Omega$ be an open, bounded subset of $\R^n$, $\varepsilon \in (0,1)$. We consider $p>1$ in this section.
%We consider the Hamiltonian $H(x,\rho) = |\rho|^p - f(x)$ where $1<p < \infty$.

\begin{thm}\label{thm:grad_1} Let $f\in \rmC(\overline{\Omega})\cap W^{1,\infty}(\Omega)$ and $u^\varepsilon \in \mathrm{C}^2(\Omega)$ be a solution to $\mathcal{L}^\varepsilon[u^\varepsilon] = 0$ in $\Omega$. Assume that $  u^\varepsilon(x)-f(x)\geq -m$ in $\Omega$. Then for $\delta>0$, there exists $C_\delta = C(m,p,\delta, \Vert D f\Vert_{L^\infty(\Omega)})$ such that 
\begin{equation*}
    \sup_{x\in \overline{\Omega}_\delta} \Big(|u^\varepsilon(x)|+|Du^\varepsilon(x)|\Big) \leq C_\delta.
\end{equation*}
\end{thm}
\noindent We provide a proof for this theorem in Appendix for the reader's convenience.




\subsection{Well-posedness of large solution for subquaratic case} In this section we recall the existence and uniqueness of solutions to \eqref{eq:PDEeps} for $1<p\leq 2$ and $f\in \mathrm{C}(\overline{\Omega})\cap W^{1,\infty}(\Omega)$. We note that the assumption of $f$ can be relaxed to $f\in L^\infty(\Omega)$ (\cite{Lasry1989}).

\begin{thm}\label{thm:wellposed1<p<2} Let $f\in \mathrm{C}(\overline{\Omega})\cap W^{1,\infty}(\Omega)$. There exists a unique solution $u^\varepsilon\in \mathrm{C}^2(\Omega)$ of \eqref{eq:PDEeps} such that:
\begin{itemize}
    \item[(i)] If $1<p< 2$, then 
\begin{equation}\label{rate_p<2}
    %\frac{C_\varepsilon-\eta}{d(x)^\alpha}-\frac{M_\eta}{ } \leq u(x)\leq \frac{C_\varepsilon+\eta}{d(x)^\alpha}+\frac{M_\eta}{ } 
    \lim_{d(x)\to 0}\left( u^\varepsilon(x) \,d(x)^\alpha \right)= C_\alpha \varepsilon^{\alpha+1}
\end{equation}
where $\alpha = (p-1)^{-1}(2-p)$ and $C_\alpha = \alpha^{-1}(\alpha+1)^{\alpha+1}$.
%\begin{equation*}
%    \displaystyle\alpha = \frac{2-p}{p-1} \qquad\text{and}\qquad C_\alpha = \frac{1}{\alpha}(\alpha+1)^{\alpha+1}.
%\end{equation*}
\item[(ii)] If $p=2$, then
\begin{equation}\label{rate_p=2}
    \lim_{d(x)\to 0} \left(-\frac{u^\varepsilon(x)}{\log(d(x))}\right) = \varepsilon.
\end{equation}
\end{itemize}
Furthermore, $u^\varepsilon$ is the maximal subolution among all subsolution $v\in W^{2,r}(\Omega)$ (for all $r>0$) of \eqref{eq:PDEeps}.
\end{thm}
\noindent This is Theorem I.1 in \cite{Lasry1989} with an explicit dependence on $\varepsilon$. We provide a proof in Appendix for later use. It is useful to note that $\alpha+1 = (p-1)^{-1}$. 

\subsection{Convergence result} We show qualitatively the convergence of \eqref{eq:PDEeps} to \eqref{eq:PDE0}. We first state the following Lemma (\cite{Capuzzo-Dolcetta1990}), which characterizes the solution to the first-order state-constraint  equation
\begin{equation}\label{S_0}
\begin{cases}
       u(x) + |Du(x)|^p - f(x) = 0\;\qquad\text{in}\;\Omega,\\
       u(x) + |Du(x)|^p - f(x) \geq 0\;\qquad\text{on}\;\partial\Omega. 
\end{cases}\tag{$S_0$}   
\end{equation}

\begin{lem}\label{lem:max} Let $u\in \rmC(\overline{\Omega})$ be a viscosity subsolution of \eqref{S_0} such that, for all viscosity subsolution $v\in \rmC(\overline{\Omega})$ of \eqref{S_0} one has $v\leq u$ in $\overline{\Omega}$. Then $u$ is a viscosity supersolution of \eqref{S_0} on $\overline{\Omega}$.
\end{lem}
\noindent We give a proof of Lemma \ref{lem:max} in Appendix for the sake of completeness. 

\begin{lem}\label{lem:lower-bound} Assume $1<p\leq 2$. Let $u^\varepsilon\in \mathrm{C}^2(\Omega)$ be a solution to \eqref{eq:PDEeps}. We have $\{  u^\varepsilon\}_{\varepsilon>0}$ is uniformly bounded from below by a constant independent of $\varepsilon$. More precisely, we have $  u^\varepsilon \geq \min_\Omega f$ and $  u\geq \min_\Omega f$.
%\begin{equation}\label{e:lower_bound_u_eps}
%    \inf_{x\in \Omega}   u^\varepsilon(x) \geq  \inf_{\Omega} f
%\end{equation}
%and also
%\begin{equation}\label{e:lower_bound_u}
%    \inf_{x\in \Omega}   u(x) \geq  \inf_{\Omega} f
%\end{equation}
\end{lem}
\begin{proof} For $m\in \mathbb{N}$, let $u^{\varepsilon,m}\in \mathrm{C}^2(\Omega)\cap \rmC(\overline{\Omega})$ solve the Drichlet problem
\begin{equation}\label{e:uepsm}
    \begin{cases}
      u(x) + |Du(x)|^p - f(x) - \varepsilon \Delta u(x) = 0 &\qquad
    \text{in}\;\Omega, \vspace{0cm}\\
    \quad \qquad\quad\qquad\qquad\qquad\qquad u(x) = m &\qquad
    \text{on}\;\partial\Omega.
    \end{cases} \tag{PDE$_{\varepsilon,m}$}
\end{equation}
We have $u^{\varepsilon}_m(x) \to u^\varepsilon(x)$ as $m\to \infty$ in $\Omega$. Let $\varphi(x) \equiv  \inf_{\Omega} f$ for $x\in \overline{\Omega}$. Then $\varphi(x)$ is a classical subsolution of \eqref{e:uepsm} in $\Omega$ with
\begin{equation*}
    \varphi(x) =   \inf_\Omega f \leq m = u^\varepsilon_m(x) \qquad\text{for}\;x\in \partial\Omega.
\end{equation*}
By comparison principle for the uniformly elliptic equation \eqref{e:uepsm}, we have
\begin{equation*}
     \inf_\Omega f \leq u^{\varepsilon}_m(x) \qquad\text{for all}\;x\in \Omega.
\end{equation*}
As $m\to \infty$, we obtain $  u^\varepsilon \geq \min_\Omega f$. The inequality $  u\geq \min_{\Omega}f$ follows from the comparison principle of \eqref{eq:PDE0} between supersolution $u$ on $\overline{\Omega}$ and subsolution $\varphi(x)$ in $\Omega$.
\end{proof}


We present a simple proof of the convergence $u^\varepsilon \to u$ using Lemma \ref{lem:max} for the reader's convenience. See also \cite[Theorem VII.3]{Capuzzo-Dolcetta1990}.
\begin{thm}[Vanishing viscosity]\label{thm:qual} Assume $\mathrm{(A1)}$. Let $u^\varepsilon$ be the solution to \eqref{eq:PDEeps}. Then there exists $u \in \mathrm{C}(\overline{\Omega})$ such that $u^\varepsilon \rightarrow u$ locally uniformly in $\Omega$ as $\varepsilon\rightarrow 0$, where $u$ solves \eqref{eq:PDE0} in $\Omega$.
\end{thm}

\begin{proof}[Proof of Theorem \ref{thm:qual}] By a priori estimate (Theorem \ref{thm:grad_1}), we have
\begin{equation}\label{e:priorie_eps}
    |u^\varepsilon(x)| + |Du^\varepsilon(x)| \leq C_\delta \qquad\text{for}\;x\in \overline{\Omega}_\delta.
\end{equation}
By the Arzel\'a--Ascoli theorem, there exists a subsequence $\varepsilon_j\to 0$ and a function $u\in \rmC(\Omega)$ such that $u^{\varepsilon_j}\to u$ locally uniformly in $\Omega$. By the stability of viscosity solution, we easily deduce that 
\begin{equation}\label{eq:u0int}
     u(x) + |Du(x)| - f(x) = 0 \qquad\text{in}\;\Omega.
\end{equation}
From Lemma \ref{lem:lower-bound}, we have $  u^\varepsilon(x)\geq \min_{\Omega} f$ and  $  u(x)\geq \min_{\Omega} f$ for all $x\in \Omega$. Together with \eqref{eq:u0int}, we obtain $|\xi|\leq \max_\Omega f - \min_\Omega f$ for all $\xi\in D^+u(x)$ and $x\in \Omega$. In other words, there is a constant $C_0$ such that
\begin{equation}\label{e:C0}
    |u(x) - u(y)| \leq C_0|x-y| \qquad\text{for all}\;x,y\in \Omega.
\end{equation}
Thus, we can extend $u$ uniquely to $u\in \rmC(\overline{\Omega})$. We will use Lemma \ref{lem:max} to show that $u$ is a supersolution of \eqref{S_0} on $\overline{\Omega}$.

It suffices to show that $u\geq w$ on $\overline{\Omega}$, where $w\in \rmC(\overline{\Omega})$ is the unique solution to \eqref{eq:PDE0}. For $\delta>0$, let $\Omega_\delta = \{x\in \Omega: \mathrm{dist}(x,\partial \Omega) > \delta\}$ and $u_\delta\in\rmC(\overline{\Omega}_\delta)$ be the  unique viscosity solution to
\begin{equation}\label{e:v_v}
    \begin{cases}
      u_\delta(x) + |Du_\delta(x)|-f(x) \leq 0 &\qquad\text{in}\;\Omega_\delta,\\
      u_\delta(x) + |Du_\delta(x)| - f(x) \geq 0 &\qquad\text{on}\;\overline{\Omega}_\delta.
    \end{cases}
\end{equation}
Since $u_\delta\rightarrow w$ locally uniformly as $\delta\rightarrow 0^+$ (see \cite{kim_state-constraint_2020}) and $w$ is bounded, $\{u_\delta\}_{\delta>0}$ is uniformly bounded. Let $v^\varepsilon_\delta\in \rmC^2(\Omega_\delta)\cap \rmC(\overline{\Omega}_\delta)$ be the unique solution to the Dirichlet problem
\begin{equation}\label{eq:vv_eps}
\begin{cases}
      v_\delta^\varepsilon(x) + |Dv_\delta^\varepsilon(x)|^p - f(x) = \varepsilon \Delta v_\delta^\varepsilon(x) &\qquad\text{in}\;\Omega_\delta,\\
    \;\;\;\,\quad\qquad\qquad\qquad\qquad v_\delta^\varepsilon = v_\delta &\qquad \text{on}\;\partial\Omega_\delta.
\end{cases}
\end{equation}
It is well-known that $v^\varepsilon_\delta\to v_\delta$ uniformly on $\overline{\Omega}_\delta$ as $\varepsilon\to 0$.

For all $\delta$ small enough, $v_\delta\leq u^\varepsilon$ on $\partial \Omega_\delta$. Hence by maximum principle, $v^\varepsilon_\delta \leq u^\varepsilon$ on $\overline{\Omega}_\delta$. Let $\varepsilon\to 0$ and we have $v_\delta \leq u$ on $\overline{\Omega}_\delta$.
Let $\delta\rightarrow 0$ and we obtain $w\leq u$ in $\Omega$, which implies $w\leq u$ on $\overline{\Omega}$ since both $w,u$ belong to $\rmC(\overline{\Omega})$.
\end{proof}


\section{Rate of convergence via the doubling variables method}
In this section, we focus on the rate of convergence for the case $f\in \mathrm{W}^{1,\infty}(\Omega)\cap\mathrm{C}(\overline{\Omega})$. By adding a constant to the solution $u^\varepsilon$ of \eqref{eq:PDEeps}, without loss of generality we can assume the \emph{nonnegativity}
\begin{equation}\label{e:minf}
    \min_{\overline{\Omega}} f(x) = 0.
\end{equation}
As a consequence, $u^\varepsilon(x),u(x)\geq 0$ for $x\in \Omega$ by Lemma \ref{lem:lower-bound}. We start with the lower bound of $u^\varepsilon - u$.
\begin{thm} Let $\Omega$ be an open, bounded and connected subset of $\R^n$ with $\mathrm{C}^2$ boundary. Assume \eqref{e:minf} and $H(x,\xi) = |\xi|^p - f(x)$ where $p\in\left(1,2\right]$. Let $u^\varepsilon$ be the unique solution to \eqref{eq:PDEeps} and $u$ be the unique solution to \eqref{eq:PDE0}. Then there exists a constant $C$ independent of $\varepsilon$ such that
\begin{equation}\label{e:lower1}
    -C\sqrt{\varepsilon} \leq u^\varepsilon(x) - u(x) \qquad\text{for all}\;x\in \Omega.
\end{equation} 
\end{thm}
\begin{proof} The proof relies on a well-known rate of convergence for vanishing viscosity of viscous Hamilton--Jacobi equation with Dirichlet boundary condition (see \cite{Calder2021,crandall_two_1984,evans_adjoint_2010,fleming_convergence_1961,Tran2011}). As $u\in \mathrm{C}(\overline{\Omega})$, let $g\in \mathrm{C}(\partial\Omega)$ defined by $g(x) = u(x)$ for $x\in \partial\Omega$. Let $v^\varepsilon\in \mathrm{C}^2(\Omega)\cap \mathrm{C}(\overline{\Omega})$ be the unique viscosity solution to
\begin{equation*}
    \begin{cases}
          v^\varepsilon(x) + |Dv^\varepsilon(x)|^p - f(x) - \varepsilon \Delta v^\varepsilon(x) = 0 &\qquad\text{in}\;\Omega,\\
        \qquad\qquad\qquad\qquad\qquad \qquad\quad v^\varepsilon(x) = g(x) &\qquad\text{on}\;\partial\Omega.
    \end{cases}
\end{equation*}
It is well-known that $v^\varepsilon \to u$ uniformly where $v$ is the viscosity solution to
\begin{equation*}
\begin{cases}
       v(x) + |Dv(x)|^p - f(x) = 0 &\qquad\text{in}\;\Omega,\\
     \qquad\qquad\qquad\quad\;\;\; v(x) = g(x)&\qquad\text{on}\;\partial\Omega. 
\end{cases}
\end{equation*}
By comparison principle, it is clear that $v = u$, the solution of \eqref{eq:PDE0}. Furthermore, there exists a positive constant $C$ such that 
\begin{equation}\label{e:cp1}
     |v^\varepsilon(x)  - u(x)| \leq C\sqrt{\varepsilon} \qquad\text{for}\;x\in \overline{\Omega}.
\end{equation}
By comparison principle for \eqref{eq:PDEeps}, we have
\begin{equation}\label{e:cp2}
    v^\varepsilon(x)\leq u^\varepsilon(x) \qquad\text{for}\;x\in \Omega.
\end{equation}
From \eqref{e:cp1} and \eqref{e:cp2} we obtain the lower bound \eqref{e:lower1}.
\end{proof}

%We state some observations  regarding the property of the solution $u$ to \eqref{eq:PDE0}.

\begin{lem}\label{lem:f=0} Assume $f\geq 0$ in $\Omega$. Then $u(x) = 0$ whenever $f(x) = 0$. In particular, $f \equiv 0$ implies $u \equiv 0$.
\end{lem}
\begin{proof} It is clear that $u\geq 0$ in $\overline{\Omega}$. Using the optimal control formula (see \cite{Bardi1997,tran_hamilton-jacobi_2021}), we have
\begin{equation}\label{e:foru}
    u(x) = \inf \left\lbrace \int_0^\infty e^{-  s}\Big(c|\dot{\eta}(s)|^{q} +f(\eta(s))\Big)ds: \eta\in \mathrm{AC}([0,\infty);\overline{\Omega}), \eta(0) = x\right\rbrace
\end{equation}
where 
\begin{equation*}
    c = \left(qp^\frac{1}{p-1}\right)^{-1} \qquad\text{and}\qquad \frac{1}{p} + \frac{1}{q} = 1.
\end{equation*}
%\textcolor{red}{(Here, the constant $c=\frac{1}{qp^\frac{1}{p-1}}$? )}


Let $x\in \overline{\Omega}$ such that $f(x) = 0$. We can choose $\eta(s) = x$ for all $s\in [0,\infty)$ as an admissible path in \eqref{e:foru} to obtain that
\begin{equation*}
    u(x) \leq  \int_0^\infty e^{-  s}\Big(c|\dot{\eta}(s)|^{q} +f(\eta(s))\Big)ds = f(x) =  0.
\end{equation*}
As a consequence, if $f\equiv 0$, then $u\equiv 0$.
\end{proof}
\noindent We state the following lemma as a crucial estimate we will be using. It is a refined construction of the supersolution for \eqref{eq:PDEeps}.
\begin{lem}\label{lem:super_refined} Let $\delta_0$ be defined as in \eqref{def:delta_0}. There exist positive constants $\nu = \nu(\delta_0)> 1$ and $C_\nu =\mathcal{O}\left(\delta_0^{-(\alpha+2)}\right)$ such that
\begin{equation}\label{e:superwa}
w(x) = \begin{cases}
    \displaystyle\frac{\nu C_\alpha \varepsilon^{\alpha+1}}{d(x)^\alpha} + \max f + C_\nu \varepsilon^{\alpha+2}, \qquad\;\;\,  p<2,\vspace{0.2cm}\\
    \displaystyle\nu \varepsilon \log\left(\frac{1}{d(x)}\right) + \max f+ C_\nu \varepsilon^2, \qquad  p=2,
\end{cases}
\end{equation}
is a supersolution of \eqref{eq:PDEeps} in $\Omega$. 
%\item[(ii)] If $x\mapsto \mathrm{dist}(x,\partial\Omega)$ is continuously differentiable with $|\nabla \mathrm{dist}(x,\partial\Omega)| = 1$ everywhere in $\Omega$ then we can take $C_\nu = 0$ in \eqref{e:superwa}.
\end{lem}
%\textcolor{orange}{(Why is $\nu \in (1,2)$? I think when $\alpha$ is large, it can be larger than 2. But $\nu$ is certainly bounded.---Oh, you can choose $\delta_0$ so small that $\nu \in (1,2)$. But then $\delta_0$ depends on $\alpha$.)} -  \textcolor{red}{I think dependence on $\alpha$ is fine, as we fix $p$ to start with.} \textcolor{orange}{- Do we include that in the definition of $\delta_0$?} --  \textcolor{red}{I will just write $\nu$ for now, not requiring $\nu \leq 2$ anymore.}

\begin{proof} Let us first consider $1<p<2$. We have $|\nabla d(x)| = 1$ for all $x\in \Omega_\delta\backslash \overline{\Omega}_{\delta_0}$. Recall from Theorem \ref{thm:wellposed1<p<2} that $C_\alpha^p \alpha^p = C_\alpha \alpha (\alpha+1)$ and $p(\alpha+1) = \alpha+2$. Compute 
\begin{equation*}
    |\nabla w(x)|^p = \nu^p\frac{(C_\alpha\alpha)^p\varepsilon^{p(\alpha+1)}}{(d(x)-\delta)^{p(\alpha+1)}} = \nu^p\frac{C_\alpha \alpha(\alpha+1)\varepsilon^{\alpha+2}}{(d(x)-\delta)^{\alpha+2}}
\end{equation*}
and
\begin{equation*}
    \varepsilon\Delta w(x) = \nu\frac{C_\alpha\alpha(\alpha+1)\varepsilon^{\alpha+2}}{(d(x)-\delta)^{\alpha+2}} - \nu\frac{C_\alpha\alpha\varepsilon^{\alpha+2}\Delta d(x)}{(d(x)-\delta)^{\alpha+1}}.
\end{equation*}
Recall that $K_2= \Vert \Delta d\Vert_{L^\infty}$, we have
\begin{equation*}
    \begin{split}
        \mathcal{L}^\varepsilon\left[w_\delta\right] =  \frac{\nu C_\alpha\varepsilon^{\alpha+1}}{(d(x)-\delta)^\alpha} + \max f - f(x) + C_\nu \varepsilon^{\alpha+2} + \frac{C_\alpha \alpha (\alpha+1)\varepsilon^{\alpha+2}}{(d(x)-\delta)^{\alpha+2}}\left[\nu^p-\nu +\nu\frac{(d(x)-\delta)\Delta d(x)}{\alpha+1}\right] .
    \end{split}
\end{equation*}
\paragraph{\textbf{Case 1.}} If $\delta< d(x)\leq \delta_0$, we observe that
\begin{equation*}
    \left|\frac{(d(x)-\delta)\Delta d(x)}{\alpha+1}\right| \leq \frac{\delta_0\Vert \Delta d\Vert_{L^\infty}}{\alpha+1} \leq \frac{K_2\delta_0}{\alpha+1}\leq K_2\delta_0.
\end{equation*}
Therefore,
\begin{equation}\label{e:choose_nu}
    \begin{split}
        \nu^p-\nu +\nu\frac{(d(x)-\delta)\Delta d(x)}{(\alpha+1)} \geq \nu^p - \nu - \nu K_2\delta_0 = \nu\Big(\nu^{p-1} - (1+K_2\delta_0)\Big).
    \end{split}
\end{equation}
We will choose $\nu$ as follows. For $\gamma>1$, we have the inequality
\begin{equation}\label{e:ineq}
    \Big||x+y|^\gamma - |x|^\gamma\Big|\leq \gamma\Big(|x|+|y|\Big)^{\gamma-1}|y|
\end{equation}
for $x,y\in \mathbb{R}$, which implies that
\begin{equation*}
     0 \leq (1+K_2\delta_0)^{\alpha+1} - 1 \leq\underbrace{(\alpha+1)\left(1+K_2\delta_0\right)^\alpha K_2}_{C_2}\delta_0.
\end{equation*}
Hence $(1+K_2\delta_0)^{\alpha+1} \leq 1 + C_2\delta_0$. Since $\alpha+1 = \frac{1}{p-1}$,
\begin{equation}\label{e:cru1}
    (1+K_2\delta_0) \leq (1+C_2\delta_0)^{\frac{1}{\alpha+1}} = (1+C_2\delta_0)^{p-1}. 
\end{equation}
Choose $\nu = 1+C_2\delta_0$ in \eqref{e:choose_nu}and we obtain $\mathcal{L}[w]\geq 0$ in $\{x\in \Omega_\delta: \delta <d(x)\leq \delta_0\}$. 

\paragraph{\textbf{Case 2.}} If $d(x)\geq \delta_0$, recall that $K_0 = \Vert d\Vert_{L^\infty}$ and $K_1 = \Vert \nabla d\Vert_{L^\infty}$, we have
\begin{align*}
    \mathcal{L}[w] &= \frac{\nu C_\alpha\alpha\varepsilon^{\alpha+1}}{(d(x)-\delta)^\alpha} + \max_{\Omega} f - f(x)\\
    &+  \nu^p\frac{C_\alpha\alpha(\alpha+1)\varepsilon^{\alpha+1}}{(d(x)-\delta)^{\alpha+2}}|\nabla d(x)|^p - \nu \frac{C_\alpha \alpha(\alpha+1)\varepsilon^{\alpha+2}}{(d(x)-\delta)^{\alpha+2}}|\nabla d(x)|^2 
    + \nu \frac{C_\alpha \alpha \varepsilon^{\alpha+2}\Delta d(x)}{(d(x)-\delta)^{\alpha+1}} + C_\nu \varepsilon^{\alpha+2}\\
    &\geq \frac{C_\alpha \alpha(\alpha+1)\varepsilon^{\alpha+2}}{(d(x)-\delta)^{\alpha+2}}\left(\nu^p|\nabla d(x)|^p - \nu |\nabla d(x)|^2 + \nu \frac{\Delta d(x)(d(x)-\delta)}{\alpha+1}\right) + C_\nu \varepsilon^{\alpha+2}\\
    &\geq \left[C_\nu - C_3\left(\frac{1}{\delta_0}\right)^{\alpha+2}\right]\varepsilon^{\alpha+2}
\end{align*}
where 
\begin{equation*}
    C_3 = C_\alpha\alpha(\alpha+1) \left(\nu^pK_1^p + \nu K_1^2 + \nu \frac{K_0K_2}{\alpha+1}\right).
\end{equation*}
We can choose $C_\nu = C_3\delta_0^{-(\alpha+2)}$ to obtain $\mathcal{L}[w]\geq 0$ in this region $\{x\in \Omega_\delta:d(x)\geq \delta_0\}$. 
\smallskip

\noindent
If $p=2$, then $\alpha = 0$. We can easily see that the same calculation holds true with $\nu = 1+K_2\delta_0$ and we choose $C_\nu = \delta_0^{-2}\nu (\nu K_1+K_0K_2)$.
\end{proof}

We start with a rate of convergence when $f = C_f$ for some constant $C_f$ in $\Omega$.
%\begin{thm}[Zero data]\label{thm:rate_doubling0} Let $\Omega$ be an open, bounded and connected subset of $\R^n$ with $\mathrm{C}^2$ boundary. Assume $f\equiv 0$ in $\Omega$. Let $u^\varepsilon$ be the unique solution to \eqref{eq:PDEeps} and $u \equiv 0 $ be the unique solution to \eqref{eq:PDE0}. If $1<p<2$ then there exists a constant $C$ such that
%    \begin{equation*}
%    0\leq u^\varepsilon(x)\leq C \left(\frac{ \varepsilon^{\alpha+1}}{d(x)^\alpha} + \frac{\varepsilon^{\alpha+2}}{\delta_{0,\Omega}^{\alpha+2}}\right), \qquad x\in \Omega
%\end{equation*}
%where $\delta_{0,\Omega}$ is defined as in \eqref{def:delta_0}. In particular $0\leq  u^\varepsilon(x) \leq C\varepsilon$ for $x\in \Omega_\varepsilon$, but $|u^\varepsilon(x)| = \mathcal{O}\left(\varepsilon^{\alpha+1}\right)$ on each compact subset of $\Omega$.
%\end{thm}
%\textcolor{orange}{(Did you mean "In particular $0\leq  u^\varepsilon(x) \leq C\varepsilon$ for $x\in \Omega_\varepsilon$" ?)}



\begin{thm}[Constant data]\label{thm:rate_doubling0} Let $\Omega$ be an open, bounded and connected subset of $\R^n$ with $\mathrm{C}^2$ boundary. Assume $f\equiv C_f$ in $\Omega$. Let $u^\varepsilon$ be the unique solution to \eqref{eq:PDEeps} and $u \equiv C_f$ be the unique solution to \eqref{eq:PDE0}. Then there exists a constant $C$ such that 
    \begin{equation*}
    \begin{split}
    &0\leq u^\varepsilon(x) - u(x)\leq C \left(\frac{ \varepsilon^{\alpha+1}}{d(x)^\alpha} + \frac{\varepsilon^{\alpha+2}}{\delta_{0,\Omega}^{\alpha+2}}\right),  \qquad\qquad \;\;\; \text{if}\; 1<p<2,\\
    &0\leq u^\varepsilon(x) - u(x)\leq C \left(\varepsilon \mathrm{log}\left(\frac{1}{d(x)}\right) + \frac{\varepsilon^{2}}{\delta_{0,\Omega}^{2}}\right),  \qquad \text{if}\; p=2,
    \end{split}
\end{equation*}
for $x\in \Omega$, where $\delta_{0,\Omega}$ is defined as in \eqref{def:delta_0}. In particular if $1<p<2$ we have $C_f\leq u^\varepsilon(x)\leq C_f + C\varepsilon$ for $x\in \Omega_\varepsilon$, but $|u^\varepsilon(x)| = \mathcal{O}\left(\varepsilon^{\alpha+1}\right)$ on each compact subset of $\Omega$.
\end{thm}
%\textcolor{orange}{(Did you mean "In particular $0\leq  u^\varepsilon(x) \leq C\varepsilon$ for $x\in \Omega_\varepsilon$" ?)}
\begin{proof} By Lemma \ref{lem:f=0} we have $u \equiv C_f$ in $\Omega$. From Lemma \ref{lem:lower-bound} and comparison principle, we have $u^\varepsilon - C_f$ is a solution to \eqref{eq:PDEeps} with data $f = 0$. So the conclusions follow from Lemma \ref{lem:super_refined}.
%\begin{equation*}
%    0\leq u^\varepsilon(x) - u(x)\leq \frac{\nu C_\alpha \varepsilon^{\alpha+1}}{d(x)^\alpha} + C_\nu\varepsilon^{\alpha+2}, \qquad x\in \Omega.
%\end{equation*}
\end{proof}




\begin{lem} Let $0<\kappa < \delta_0$ and $U_\kappa = \big\{x\in \Omega: 0<\mathrm{dist}(x,\partial\Omega) < \kappa\big\} = \Omega\backslash \overline{\Omega}_\kappa$. There holds
\begin{equation*}
    \mathrm{dist}(x,\partial\Omega_\kappa) = \kappa - \mathrm{dist}(x,\partial\Omega) \qquad\text{for all}\;x\in U_\kappa.
\end{equation*}
As a consequence, $x\mapsto \mathrm{dist}(x,\partial U_\kappa) = \min\big\{\mathrm{dist}(x,\partial \Omega_k),\mathrm{dist}(x,\partial \Omega)\big\}$ is twice continuously differentiable for $x\in \Omega\backslash \overline{\Omega}_{\frac{\kappa}{2}}$. Hence we can choose 
\begin{equation}\label{e:delta_kappa}
    \delta_{0,U_\kappa} \geq \frac{\kappa}{4}.
\end{equation}
\end{lem}
%\textcolor{orange}{I feel "...twice continuously differentiable away from $\partial \Omega_\frac{k}{2}$" is a little ambiguous. How about "...twice continuoously differentiable for $x \in \Omega \setminus \overline{\Omega}_\frac{\kappa}{2}$"?}
\begin{proof} By the definition of $\delta_0 = \delta_{0,\Omega}$, we have $d(x) = \mathrm{dist}(x,\partial\Omega)$ is twice continuously differentiable in the region $U_{\delta_0} = \Omega\backslash \overline{\Omega}_{\delta_0}$. The proof follows from \cite[p. 355]{gilbarg_elliptic_2001}. 
\end{proof}

Next, we show the rate of convergence for compactly supported data $f$.

\begin{thm}[Compactly supported data]\label{thm:rate_doubling1} Let $\Omega$ be an open, bounded and connected subset of $\R^n$ with $\mathrm{C}^2$ boundary. Assume that $f$ is Lipschitz and is compactly supported in $\Omega$. Let $u^\varepsilon$ be the unique solution to \eqref{eq:PDEeps} and $u$ be the unique solution to \eqref{eq:PDE0}. Then there exists a constant $C$ depends on the support of $f$ such that
\begin{equation*}
|u^\varepsilon(x) - u(x)| \leq C\sqrt{\varepsilon} \quad  \text{for} \quad x\in \Omega_{\varepsilon}.
\end{equation*}
Furthermore, on any compact subset $K\subset\Omega\backslash \mathrm{supp}(f)$, $|u^\varepsilon - u| = \mathcal{O}(\varepsilon)$.
\end{thm}

\begin{proof} Without loss of generality, we can assume that $f$ is supported in $\Omega_\kappa$ where $0<\kappa < \delta_{0}$.
Let $g_\kappa = u^\varepsilon$ on $\partial\Omega_{\kappa}$. Then the solution $u^\varepsilon$ of \eqref{eq:PDEeps} also solves
\begin{equation*}
    \left\{
  \begin{aligned}
  u^\varepsilon(x) + |Du^\varepsilon(x)|^p-\varepsilon \Delta u^\varepsilon(x) &=0 \;\qquad \text{in } U_\kappa ,\\
  u^\varepsilon(x) &= +\infty \quad \text{on } \partial \Omega,\\
  u^\varepsilon(x) &= g_\kappa \;\;\quad \text{on } \partial \Omega_{\kappa},
    \end{aligned}
\right.
\end{equation*}
in $U_\kappa= \Omega \setminus \overline{\Omega}_{\kappa} = \{x\in \Omega: 0< d(x) < \kappa\}$. Let $\tilde{u}^\varepsilon\in \mathrm{C}^2(U_\kappa)$ be the solution to the following problem in the annulus $U_\kappa= \Omega \setminus \overline{\Omega}_{\kappa}$ (Theorem \ref{thm:wellposed1<p<2})
\begin{equation*}
    \left\{
        \begin{aligned}
            \tilde{u}^\varepsilon(x) + |D\tilde{u}^\varepsilon(x)|^p-\varepsilon \Delta \tilde{u}^\varepsilon(x) &=0 \;\qquad \text{in } U_\kappa ,\\
            \tilde{u}^\varepsilon(x) &= +\infty \quad \text{on } \partial U = \partial \Omega\cup \partial \Omega_{\kappa}.
        \end{aligned}
    \right.
\end{equation*}
Here the boundary condition is understood as $\tilde{u}^\varepsilon(x)\to \infty$ as $d_\kappa(x)\to 0$ where $d_\kappa(\cdot)$ is the distance function from the boundary of the annulus $U_\kappa$, i.e.,
\begin{equation*}
    d_\kappa(x) = \min \big\lbrace \mathrm{dist}(x,\partial \Omega_\kappa),\mathrm{dist}(x,\partial\Omega)  \big\rbrace \leq d(x) \qquad\text{for}\;x\in U_\kappa.
\end{equation*}
Since $f = 0$ in $\overline{U}_\kappa$, by Lemma \ref{lem:f=0} $u=0$ in $\overline{U}_\kappa$. Hence it is also the unique state-constraint solution to
\begin{equation*}
    \left\{
        \begin{aligned}
            u(x)+ |Du(x)|^p &=0 \quad \text{in } U_\kappa ,\\
            u(x)+ |Du(x)|^p &\geq 0 \quad \text{on } \partial U_\kappa = \partial \Omega \cup\partial \Omega_{\kappa}.
        \end{aligned}
    \right.
\end{equation*}
The vanishing viscosity of $\tilde{u}^\varepsilon \to 0$ in $U_\kappa$ can be quantified by Theorem \ref{thm:rate_doubling0}, which gives us
\begin{equation*}
\begin{split}
    &0\leq \tilde{u}^\varepsilon(x) \leq \frac{\nu C_\alpha \varepsilon^{\alpha+1}}{d_\kappa(x)^\alpha}+C_3\left(\frac{\varepsilon}{\delta_{0,U_\kappa}}\right)^{\alpha+2}\qquad\quad\;\text{for}\;p<2,\\
    &0\leq \tilde{u}^\varepsilon(x) \leq \nu \varepsilon \log\left(\frac{1}{d_\kappa(x)}\right)+C_\nu\left(\frac{\varepsilon}{\delta_{0,U_\kappa}}\right)^{2}\qquad\text{for}\;p=2,
\end{split}    
\end{equation*}
for $x\in U_\kappa$. From \eqref{e:delta_kappa} and the comparison principle in $U_\kappa$, we have
\begin{align}
    &0\leq u^\varepsilon(x) \leq \tilde{u}^\varepsilon(x)  \leq \frac{\nu C_\alpha\varepsilon^{\alpha+1}}{d_\kappa(x)^{\alpha}} + C_3\left(\frac{4\varepsilon}{\kappa}\right)^{\alpha+2} \qquad\quad\;\text{for}\;p<2, \qquad \label{annulus2}\\
    &0\leq u^\varepsilon(x)\leq \tilde{u}^\varepsilon(x) \leq \nu \varepsilon \log\left(\frac{1}{d_\kappa(x)}\right)+C_\nu\left(\frac{4\varepsilon}{\kappa}\right)^{2}\qquad\text{for}\;p=2,\label{annulus2p=2}
\end{align}
for $x\in U_\kappa$. We proceed with the doubling variable method. For $p<2$, consider the auxiliary functional 
\begin{equation*}
    \Phi(x,y)= u^\varepsilon(x) - u(y) -\frac{C_0|x-y|^2}{\sigma} - \frac{\nu C_\alpha \varepsilon^{\alpha +1}}{d(x)^\alpha}, \qquad (x,y)\in \overline{\Omega}\times \overline{\Omega},
\end{equation*}
where $C_0$ is the Lipschitz constant of $u$ from \eqref{e:C0}, $\sigma\in (0,1)$. The fact that $\displaystyle d(x)^\alpha u^\varepsilon(x) \to C_\alpha \varepsilon^{\alpha+1}$ as $d(x) \to 0$ implies
\begin{equation*}
    \max_{(x,y) \in \overline{\Omega} \times \overline{\Omega}} \Phi(x,y) = \Phi(x_\sigma, y_\sigma) \qquad\text{for some}\;(x_\sigma,y_\sigma) \in \Omega \times \overline{\Omega}.
\end{equation*}
From $\Phi(x_\sigma, y_\sigma) \geq \Phi(x_\sigma, x_\sigma)$, we can deduce that
\begin{equation}\label{e:sigma}
    \left| x_\sigma - y_\sigma \right| \leq \sigma.  
\end{equation}
If $ d(x_\sigma) \geq \frac{1}{2}\kappa$, since $x\mapsto \Phi(x,y_\sigma)$ has a maximum over $\Omega$ at $x=x_\sigma$, the subsolution test for $u^\varepsilon(x)$ gives us
\begin{align}\label{e:subsln}
    &u^\varepsilon(x_\sigma) + \left|\frac{2C_0(x_\sigma - y_\sigma)}{\sigma} -  \frac{\nu C_\alpha\alpha \varepsilon^{\alpha+1} \nabla d(x_\sigma)}{d(x_\sigma)^{\alpha+1}}\right|^p - f(x_\sigma)\nonumber\\
    &\qquad -\varepsilon\left(\frac{2nC_0}{\sigma}+ \frac{\nu C_\alpha\alpha(\alpha+1) \varepsilon^{\alpha+1}|\nabla d(x_\sigma)|^2}{d(x_\sigma)^{\alpha+2}} - \frac{\nu C_\alpha\alpha \varepsilon^{\alpha+1}\Delta d(x_\sigma)}{d(x_\sigma)^{\alpha+1}}\right) \leq 0.
\end{align}
Since $y\mapsto \Phi(x_\sigma,y)$ has a maximum over $\overline{\Omega}$ at $y = y_\sigma$, the supersolution test for $u(y)$ gives us
\begin{align}\label{e:supersln}
    u(y_\sigma) + \left|\frac{2C_0(x_\sigma - y_\sigma)}{\sigma}\right|^p - f(y_\sigma) \geq 0.
\end{align}
For simplicity, let us define
\begin{equation*}
    \xi_\sigma := \frac{2C_0(x_\sigma - y_\sigma)}{\sigma} \qquad\text{and}\qquad \zeta_\sigma :=- \frac{\nu C_\alpha\alpha \varepsilon^{\alpha+1} \nabla d(x_\sigma)}{d(x_\sigma)^{\alpha+1}}.
\end{equation*}
From \eqref{e:sigma} and the fact that we are considering $d(x_\sigma) \geq \frac{1}{2}\kappa$, 
\begin{equation*}
    |\xi_\sigma|\leq 2C_0 \qquad\text{and}\qquad |\zeta_\sigma| \leq \nu K_1 C_\alpha\alpha \left(\frac{\varepsilon}{d(x_\sigma)}\right)^{\alpha+1} \leq \nu K_1 C_\alpha \alpha  \left(\frac{2\varepsilon}{\kappa}\right)^{\alpha+1}.
\end{equation*}
Using the inequality \eqref{e:ineq} with $\gamma = p > 1$, we deduce that
\begin{align}\label{e:estia}
    \Big||\xi_\sigma +\zeta_\sigma|^p - |\xi_\sigma|^p \Big| &\leq p\Big(|\xi_\sigma|+|\zeta_\sigma|\Big)^{p-1}|\zeta_\sigma|\nonumber\\
    &\leq p\left[2C_0+\nu K_1 C_\alpha\alpha \left(\frac{2\varepsilon}{\kappa}\right)^{\alpha+1}\right]^{p-1}\nu K_1 C_\alpha\alpha \left(\frac{2\varepsilon}{\kappa}\right)^{\alpha+1}.
\end{align}
Using \eqref{e:estia} together with \eqref{e:subsln}, \eqref{e:supersln} and $|f(x_\sigma) - f(y_\sigma)|\leq C|x_\sigma - y_\sigma|\leq C\sigma$, we obtain that
\begin{align*}
     u^\varepsilon(x_\sigma) - u(y_\sigma) \leq & p\left(2C_0+\nu K_1 C_\alpha \alpha\left( \frac{2\varepsilon}{\kappa}\right)^{\alpha+1}\right)^{p-1}\nu K_1 C_\alpha \alpha \left(\frac{2\varepsilon}{\kappa}\right)^{\alpha+1} + C\sigma\nonumber\\
    &+2nC_0\left(\frac{\varepsilon}{\sigma}\right) + \nu K_1^2C_\alpha \alpha(\alpha+1)\left(\frac{2\varepsilon}{\kappa}\right)^{\alpha+2} + \nu K_2 C_\alpha \alpha  \left(\frac{2\varepsilon}{\kappa}\right)^{\alpha+1}\varepsilon\nonumber\\
    &\leq C\left[\sigma + \frac{\varepsilon}{\sigma} + \left(1+\left(\frac{\varepsilon}{\kappa}\right)^{\alpha+1}\right)^{p-1}\left(\frac{\varepsilon}{\kappa}\right)^{\alpha+1} + \left(\frac{\varepsilon}{\kappa}\right)^{\alpha+2} \right].
    %&\leq C\sqrt{\varepsilon}
\end{align*}
By the fact that if $\gamma\in [0,1]$ then $(1+x)^\gamma \leq 1+x^\gamma$ for $x\in [0,1]$, we can get that, as $0<p-1\leq 1$,
\begin{equation*}
    \left(1+\left(\frac{\varepsilon}{\kappa}\right)^{\alpha+1}\right)^{p-1} \leq 1+ \left(\frac{\varepsilon}{\kappa}\right).
\end{equation*}
Therefore,
\begin{equation*}
     u^\varepsilon(x_\sigma) - u(y_\sigma) \leq C\left[\sigma + \frac{\varepsilon}{\sigma} + \left(\frac{\varepsilon}{\kappa}\right)^{\alpha+1} + \left(\frac{\varepsilon}{\kappa}\right)^{\alpha+2}\right]
\end{equation*}
where $C$ is independent of $\kappa$ and $\varepsilon$. Choosing $\sigma = \sqrt{\varepsilon}$, we obtain that (with $\kappa$ being fixed)
\begin{equation}\label{e:final1}
    \Phi(x_\sigma,y_\sigma) \leq u^\varepsilon(x_\sigma) - u(y_\sigma) \leq C\sqrt{\varepsilon}.
\end{equation}
If $d(x_\sigma) < \frac{1}{2}\kappa$, then $x_\sigma \in U_\kappa$ and furthermore $\mathrm{dist}(x_\sigma,\partial \Omega_\kappa) > \frac{1}{2}\kappa$. Indeed, for any $y\in\partial\Omega$ and $z\in \partial\Omega_k$, we have $|x_\sigma - z| + |x_\sigma - y| \geq |y-z|$. Taking the infimum over all $y\in \partial\Omega$, we deduce that
\begin{equation*}
    |x_\sigma - z| + d(x_\sigma) \geq \inf_{y\in \partial \Omega}|y-z| = d(z) = \kappa
\end{equation*}
since $z\in \partial\Omega_k = \{x \in \Omega: d(x) = \kappa\}$. Thus, we have $|x_\sigma - z| \geq \kappa - d(x_\sigma) > \frac{1}{2}\kappa$ for all $z\in \partial\Omega_k$, which implies that $\mathrm{dist}(x_\sigma,\partial\Omega_k)>\frac{1}{2}\kappa$ and therefore $d_\kappa(x_\sigma) = d(x_\sigma)$. By \eqref{annulus2} and the fact that $u \geq 0$, we have
\begin{align}
    \Phi(x_\sigma, y_\sigma)\leq u^\varepsilon(x_\sigma) - \frac{\nu C_\alpha \varepsilon^{\alpha+1}}{d(x_\sigma)^\alpha} \leq C_3\left(\frac{4\varepsilon}{\kappa}\right)^{\alpha+2} \label{e:final2}.
\end{align}
%\textcolor{red}{(I just read this one more time and feel confused here as the RHS should be $\frac{\nu C_\alpha \epsilon^{\alpha + 1}}{d(x_\sigma)}+C_3(\frac{4\epsilon}{\kappa})^{\alpha+2}$. But then the following arguement doesn't make sense. I think the inequality should be something like 
%\begin{equation*}
%\begin{aligned}
%    \Phi(x_\sigma, y_\sigma) &\leq u^\varepsilon (x_\sigma)- \frac{\theta  C_\alpha  \varepsilon^{\alpha+1}}{d(x_\sigma)^\alpha}\nonumber\\
 %   &= u^\varepsilon (x_\sigma)-\frac{\nu C_\alpha  \varepsilon^{\alpha+1}}{d_\kappa(x_\sigma)^\alpha}
 %   \leq C_3\left(\frac{4\varepsilon}{\kappa}\right)^{\alpha+2}
%\end{aligned}   
%\end{equation*} 
%where we choose $\theta = \nu$, similar to the proof in draft01. The conclusion isn't much more different.)}
\noindent
Since $\Phi(x,x) \leq \Phi(x_\sigma,y_\sigma)$ for all $x\in \Omega$, we obtain from \eqref{e:final1} and \eqref{e:final2} that
\begin{equation*}
    u^\varepsilon(x)-u(x)-\frac{\nu C_\alpha  \varepsilon^{\alpha+1}}{d(x)^\alpha} \leq C\sqrt{\varepsilon} +C_3\left(\frac{4\varepsilon}{\kappa}\right)^{\alpha+2}.
\end{equation*}
Therefore
\begin{equation*}
   -C \sqrt{\varepsilon}\leq  u^\varepsilon(x) - u(x) \leq C\left(\sqrt{\varepsilon}+\left(\frac{\varepsilon}{\kappa}\right)^{\alpha+2}\right) + \frac{\nu C_\alpha  \varepsilon^{\alpha+1}}{d(x)^\alpha} 
\end{equation*}
for every $x\in \Omega$. It follows that $|u^\varepsilon(x)-u(x)| \leq C\sqrt{\varepsilon}$ for $x\in \Omega_\varepsilon$. 
If $x\in U_k$, i.e., $f(x) = 0$, then \eqref{annulus2} gives us a better rate $\mathcal{O}(\varepsilon)$ in any compact set $K$ that is disjoint from $\mathrm{supp}(f)$. \\

\noindent For $p=2$, we consider instead the functional 
\begin{equation*}
    \Phi(x,y) = u^\varepsilon(x) - u(y) - \frac{C_0|x-y|^2}{\sigma} - \theta \varepsilon \mathrm{log}\left(\frac{1}{d(x)}\right), \qquad (x,y)\in \overline{\Omega}\times \overline{\Omega}
\end{equation*}
for some $\theta>1$. Similar to the previous case where $p<2$, the maximum of $\Phi$ must occur at $(x_\sigma,y_\sigma)\in \Omega\times\overline{\Omega}$ and $|x_\sigma - y_\sigma|\leq \sigma$. If $d(x_\sigma)\geq \frac{1}{2}\kappa$, by the subsolution test for $u^\varepsilon(x)$, we have
\begin{align}\label{e:subslnp=2}
    u^\varepsilon(x_\sigma)&+ \left|\frac{2C_0(x_\sigma - y_\sigma)}{\sigma} - \theta \varepsilon \frac{\nabla d(x_\sigma)}{d(x_\sigma)}\right|^2  -f(x_\sigma) \nonumber\\
    &- 2nC_0\left(\frac{\varepsilon}{\sigma}\right) - \theta |\nabla d(x_\sigma)|^2 \left(\frac{\varepsilon}{d(x_\sigma)}\right)^2  + \theta \Delta d(x_\sigma)\left(\frac{\varepsilon^2}{d(x_\sigma)}\right) \leq 0.
\end{align}
By the supersolution test for $u(y)$, we have
\begin{equation}\label{e:superslnp=2}
    u(y_\sigma) + \left|\frac{2C_0(x_\sigma - y_\sigma)}{\sigma}\right|^2 - f(y_\sigma) \geq 0.
\end{equation}
Subtracting \eqref{e:superslnp=2} from \eqref{e:subslnp=2}, we have
\begin{align*}
    u^\varepsilon(x_\sigma) - u(y_\sigma) &\leq \left(4C_0+ \theta\varepsilon\frac{\nabla d(x_\sigma)}{d(x_\sigma)}\right)\left(\theta \varepsilon \frac{ \nabla d(x_\sigma)}{d(x_\sigma)}\right) \\
    &+ C\sigma + 2nC_0 \left(\frac{\varepsilon}{\sigma}\right)+ \theta |\nabla d(x_\sigma)|^2 \left(\frac{\varepsilon}{d(x_\sigma)}\right)^2 + \theta|\Delta d(x_\sigma)| \frac{\varepsilon^2}{d(x_\sigma)}.
\end{align*}
Using $d(x_\sigma)\geq \frac{1}{2}\kappa$ and bounds on $d(x)$ as in \eqref{boundond}, we see that
\begin{align}\label{p=2a}
    \Phi(x_\sigma,y_\sigma)&\leq u^\varepsilon(x_\sigma) - u(y_\sigma)\nonumber \\
    &\leq 4K_1^2\theta(1+\theta)\left(\frac{\varepsilon}{\kappa}\right)^2 + C\sigma + 2nC_0\left( \frac{\varepsilon}{\sigma}\right) + 2\theta(K_2\varepsilon+4C_0K_1)\left(\frac{\varepsilon}{\kappa}\right)\nonumber\\
    &\leq C\left(\sigma+\frac{\varepsilon}{\sigma} + \frac{\varepsilon}{\kappa} + \left(\frac{\varepsilon}{\kappa}\right)^2\right) \leq C\sqrt{\varepsilon}
\end{align}
where we choose $\sigma = \sqrt{\varepsilon}$. 

\noindent
If $d(x_\sigma)<\frac{1}{2}\kappa$, then $x_\sigma\in U_\kappa$. Similar to the previous argument, we have $d_\kappa(x_\sigma) = d(x_\sigma)$ and from \eqref{annulus2p=2}
\begin{equation}\label{p=2b}
    \Phi(x_\sigma,y_\sigma) \leq \nu\varepsilon \log\left(\frac{2}{\kappa}\right) + C_\nu\left(\frac{4\varepsilon}{\kappa}\right)^2 \leq C\left(\varepsilon+\varepsilon\,|\log(\kappa)| + \left(\frac{\varepsilon}{\kappa}\right)^2\right).
\end{equation}
Since $\Phi(x,x)\leq \Phi(x_\sigma,y_\sigma)$ for $x\in \Omega$, we obtain from \eqref{p=2a} and \eqref{p=2b} that
\begin{equation*}
    u^\varepsilon(x) - u(x) - \theta\varepsilon\log\left(\frac{1}{d(x)}\right) \leq C\sqrt{\varepsilon} +C\varepsilon \left(1+|\log(\kappa)| + \frac{\varepsilon}{\kappa^2}\right).
\end{equation*}
Let $\theta\to 1^+$ and we obtain 
\begin{equation*}
    -C\sqrt{\varepsilon}\leq u^\varepsilon(x) - u(x) \leq C \left(\sqrt{\varepsilon}+\varepsilon+\varepsilon|\log(\kappa)| + \left(\frac{\varepsilon}{\kappa}\right)^2\right) +  \varepsilon \log\left(\frac{1}{d(x)}\right), \qquad x\in \Omega.
\end{equation*}
Similar to the previous case, in the region where $0< d(x)\leq \varepsilon$, $|u^\varepsilon-u|=\mathcal{O}(\sqrt{\varepsilon})$ since $\varepsilon|\log(\varepsilon)|\leq \sqrt{\varepsilon}$. Thus the conclusion follows.
\end{proof}

%\begin{rem} From \cite{Lasry1989} we know that, if $p\in \left(\frac{3}{2},2\right]$ then $u^\varepsilon(x) - \frac{C_\alpha\varepsilon^{\alpha+1}}{d(x)^\alpha}$ is uniformly bounded in $\Omega$. Theorem \ref{thm:rate_doubling1} says that, if $f$ is compactly supported in $\Omega$ then 
%\begin{equation*}
%    u^\varepsilon(x) - (\nu+1)\frac{C_\alpha\varepsilon^{\alpha+1}}{d(x)^\alpha}
%\end{equation*}
%is uniformly bounded from above in $\Omega$ for all $p\in (1,2)$. Recall that $\nu = 1+C_2\delta_{0,\Omega}$ from Lemma \ref{lem:super_refined}, we can always choose $\delta_0$ small so that roughly $\nu\approx 2^+$.
%(\textcolor{red}{This sounds really weird.})
%\end{rem}

%\textcolor{orange}{(I think Theorem \ref{thm:rate_doubling1} only says $u^\epsilon-\frac{3C_\alpha \epsilon^{\alpha+1}}{d(x)^\alpha}$ is uniformly bounded above by some constant $C$. It doesn't say $u^\epsilon-\frac{3C_\alpha \epsilon^{\alpha+1}}{d(x)^\alpha}$ is uniformly bounded below. )}


\begin{rem} For general data Lipschitz $f\in \mathrm{C}(\overline{\Omega})$, it is natural to think about using cutoff function argument. Let $\chi_{{\kappa}}\in \mathrm{C}_c^\infty(\Omega)$ such that $0\leq \chi_{\kappa}\leq 1$, $\chi_\kappa = 1$ in $\Omega_{2\kappa}$ and $\mathrm{supp}\;\chi_\kappa\subset\Omega_\kappa$. Let $u^\varepsilon_\kappa\in \rmC^2(\Omega)\cap\overline{\Omega}$ solves \eqref{eq:PDEeps} with datum $f\chi_{\kappa}$. Then as $u^\varepsilon_\kappa\to u^\varepsilon$ as $\kappa\to 0$ (since $f\chi_\kappa\to f$ in the weak$^*$ topology of $L^\infty(\Omega)$, we have a continuity of solution with respect to data in this topology \cite[Remark II.1]{Lasry1989}). However, it is not clear at the moment how to quantify this rate of convergence, since $f\chi_\kappa$ does not converge to $f$ in the uniform norm, unless $f = 0$ on $\partial\Omega$. 
\end{rem}
 


\subsection{A rate for zero boundary data}


\begin{thm}[Zero boundary data]\label{thm:rate_doubling2} Let $\Omega$ be an open, bounded and connected subset of $\R^n$ with $\mathrm{C}^2$ boundary. Assume that $f$ is Lipschitz such that $f = 0$ on $\partial\Omega$. Let $u^\varepsilon$ be the unique solution to \eqref{eq:PDEeps} and $u$ be the unique solution to \eqref{eq:PDE0}. Then there exists a constant $C$ such that
\begin{equation*}
|u^\varepsilon(x) - u(x)| \leq C\sqrt{\varepsilon} \quad  \text{for} \quad x\in \Omega_{\varepsilon}.
\end{equation*}
\end{thm}

\begin{proof} Let $L = \Vert \nabla f\Vert_{L^\infty(\Omega)}$ be the Lipschitz constant of $f$. For $\kappa>0$ small such that $0<\kappa<\delta_0$ and $x\in \Omega\backslash \Omega_\kappa$, let $x_0$ be the projection of $x$ onto $\partial\Omega$. We observe that
\begin{equation}\label{e:bound_aa}
    f(x) = f(x) - f(x_0) \leq L|x-x_0| = L\kappa.
\end{equation}
Define
\begin{equation*}
    g_\kappa(x) = 
    \begin{cases}
        0           &\qquad\text{if}\;0\leq d(x) \leq \frac{\kappa}{2},\\
        2L\left(d(x)-\frac{\kappa}{2}\right)       &\qquad\text{if}\;\frac{\kappa}{2}\leq d(x) \leq \kappa.
    \end{cases}
\end{equation*}

%\textcolor{orange}{(A question is why we can assume $f\in C^1(\overline{\Omega})$? If you smooth $f$ up by mollifier, the support of $f$ might go outside of $\Omega$. Another question is I think you want $g_k$ to be $2L(d(x)-\frac{\kappa}{2})$ if $\frac{\kappa}{2} \leq d(x) \leq \kappa $. )}
\noindent
It is clear that for $x\in \partial\Omega_k$, $g_\kappa(x) = L\kappa \geq f(x)$ since \eqref{e:bound_aa}. Therefore, we can define the following continuous function
\begin{equation*}
    f_\kappa(x) = 
    \begin{cases}
        0             &\qquad\text{if}\;0\leq d(x) \leq \frac{\kappa}{2},\\
        \min \left\lbrace g_\kappa(x), f(x) \right\rbrace &\qquad\text{if}\;\frac{\kappa}{2}\leq d(x) \leq \kappa,\\
        f(x) &\qquad\text{if}\;\kappa \leq d(x).
    \end{cases}
\end{equation*}
\begin{figure}[h]
    \centering
    \includegraphics[scale=0.35]{Drafts and notes/fig1.png}
    \caption{Graph of the function $f_\kappa$.}
    \label{fig:f_kappa}
\end{figure}
A graph of $f_\kappa$ is given in Figure \ref{fig:f_kappa}. The continuity at $x\in \partial\Omega_\kappa$ comes from the fact that when $d(x) =\kappa$, we have $g_k(x) = L\kappa \geq f(x)$ by \eqref{e:bound_aa}. It is clear that $f_\kappa$ is Lipschitz with $\Vert f_\kappa\Vert_{L^\infty(\Omega)}\leq L$ as well and $f_\kappa\to f$ uniformly as $\kappa\to 0$. Indeed, we have $0\leq f_\kappa \leq f$ and
\begin{equation*}
    0\leq \max_{x\in \overline{\Omega}} (f(x) - f_\kappa(x)) \leq \max_{x\in \overline{\Omega}\backslash \overline{\Omega}_\kappa} (f(x) - f_\kappa(x)) = \max_{x\in \overline{\Omega}\backslash \overline{\Omega}_\kappa} f(x) \leq L\kappa.
\end{equation*}
Let $u^\varepsilon_\kappa\in \rmC^2(\Omega)\cap\overline{\Omega}$ be the solution to \eqref{eq:PDEeps} with datum $f\chi_{\kappa}$ and $u_k\in \mathrm{C}(\overline{\Omega})$ be the corresponding solution to \eqref{eq:PDE0} with datum $f{\chi_\kappa}$. By comparison principle (\cite[Corollary II.1]{Lasry1989}) we have
\begin{equation}\label{e:bound_2aa}
    0\leq u^\varepsilon(x) - u^\varepsilon_\kappa(x) \leq L\kappa \qquad\text{for}\;x\in \Omega.
\end{equation}
By the comparison principle for \eqref{eq:PDE0}, we also have
\begin{equation}\label{e:bound_2ab}
    0\leq u(x) - u_\kappa(x) \leq L\kappa \qquad\text{for}\;x\in \Omega.
\end{equation}
If $1<p<2$, by Theorem \ref{thm:rate_doubling1}, there exists a constant $C$ independent of $\kappa$ such that
\begin{equation}\label{e:bound_2ac}
    -C\sqrt{\varepsilon}\leq u^\varepsilon_\kappa(x) - u_\kappa(x)\leq C\left[\sqrt{\varepsilon} + \left(\frac{\varepsilon}{\kappa}\right)^{\alpha+1} + \left(\frac{\varepsilon}{\kappa}\right)^{\alpha+2} + \frac{\varepsilon^{\alpha+1}}{d(x)^\alpha}\right], \qquad x\in \Omega.
\end{equation}
Combining \eqref{e:bound_2aa}, \eqref{e:bound_2ab} and \eqref{e:bound_2ac}, we obtain
\begin{equation*}
\begin{split}
   -C\sqrt{\varepsilon}\leq u^\varepsilon(x) - u(x) &= \Big(u^\varepsilon(x) - u^\varepsilon_\kappa(x)\Big) + \Big(u^\varepsilon_\kappa(x) - u_\kappa(x)\Big) + \Big(u_\kappa(x) - u(x)\Big) \\
    &\leq L\kappa + C\left[\sqrt{\varepsilon} + \left(\frac{\varepsilon}{\kappa}\right)^{\alpha+1} + \left(\frac{\varepsilon}{\kappa}\right)^{\alpha+2} + \frac{\varepsilon^{\alpha+1}}{d(x)^\alpha}\right], \qquad x\in \Omega.
\end{split}
\end{equation*}
Choose $\kappa = \sqrt{\varepsilon}$. We deduce that
\begin{equation*}
    -C\sqrt{\varepsilon}\leq u^\varepsilon(x) - u(x) \leq C\sqrt{\varepsilon} + \frac{C\varepsilon^{\alpha+1}}{d(x)^\alpha}
\end{equation*}
for $x\in \Omega$ and the conclusion follows. \\

\noindent If $p=2$, by Theorem \ref{thm:rate_doubling1}, there exists a constant $C$ independent of $\kappa$ such that
\begin{equation}\label{e:bound_2acp=2}
    -C\sqrt{\varepsilon}\leq u^\varepsilon_\kappa(x) - u_\kappa(x) \leq C\left[\sqrt{\varepsilon} + \varepsilon + \varepsilon |\log(\kappa)|+\left(\frac{\varepsilon}{\kappa}\right)^2 + \varepsilon\log\left(\frac{1}{d(x)}\right)\right], \qquad x\in \Omega.
\end{equation}
Combining \eqref{e:bound_2aa}, \eqref{e:bound_2ab} and \eqref{e:bound_2acp=2}, we obtain
\begin{equation*}
\begin{split}
   -C\sqrt{\varepsilon}\leq u^\varepsilon(x) - u(x) &= \Big(u^\varepsilon(x) - u^\varepsilon_\kappa(x)\Big) + \Big(u^\varepsilon_\kappa(x) - u_\kappa(x)\Big) + \Big(u_\kappa(x) - u(x)\Big) \\
    &\leq L\kappa + C\left[\sqrt{\varepsilon} + \varepsilon|\log(\kappa)| + \left(\frac{\varepsilon}{\kappa}\right)^{2} + \varepsilon\log\left(\frac{1}{d(x)}\right) \right], \qquad x\in \Omega.
\end{split}
\end{equation*}
Choose $\kappa = \varepsilon$. We deduce that
\begin{equation*}
    -C\sqrt{\varepsilon}\leq u^\varepsilon(x) - u(x) \leq C\sqrt{\varepsilon} +\varepsilon\log\left(\frac{1}{d(x)}\right)
\end{equation*}
for $x\in \Omega$ and the conclusion follows. 
\end{proof}

\subsection{A generalization to some unbounded data} The approach we use allow us to apply to the situation where the data $f_\varepsilon$ blows up with some certain rates near the boundary such that $f_\varepsilon\to f$ in $L^1(\Omega)$. To simplify the presentation, we will assume $f\in \mathrm{C}^1(\Omega)$ and $f$ is bounded from below satisfying 
\begin{equation*}
    f(x) = g(x) + \frac{\gamma \varepsilon^{\alpha+2}}{d(x)^{\alpha+2}}
\end{equation*}
where $g\in \mathrm{C}^1(\overline{\Omega})$with $g = 0$ on $\partial\Omega$ and $g\geq 0$ in $\Omega$.

%\begin{thm}[General bounded Lipschitz data]\label{thm:rate_doubling1} Let $\Omega$ be an open, bounded and connected subset of $\R^n$ with $\mathrm{C}^2$ boundary. Let $u^\varepsilon$ be the unique solution to \eqref{eq:PDEeps} and $u$ be the unique solution to \eqref{eq:PDE0}. Then there exists a constant $C$ such that
%\begin{equation*}
%|u^\varepsilon(x) - u(x)| \leq C\sqrt{\varepsilon} \quad  \text{for} \quad x\in \Omega_{\varepsilon}.
%\end{equation*}
%\end{thm}

%\begin{proof} Let $\kappa \in (0,1)$ and $\chi_{\kappa}\in \mathrm{C}_c^\infty(\mathbb{R}^n)$ such that $0\leq \chi_\kappa \leq 1$, $\chi = 1$ in $\Omega_{2\kappa}$ and $\mathrm{supp}(\chi_\kappa)\subset \Omega_\kappa$ and 
%\begin{equation*}
%    \max_{x\in \overline{\Omega}}|\nabla \chi_\kappa(x)| \leq \frac{C}{\kappa} .
%\end{equation*}
%Denote $f_\kappa(x) = f(x)\chi_\kappa(x)$ for $x\in \overline{\Omega}$. It is clear that 
%\begin{equation*}
%    \mathrm{Lip}(f_\kappa)\leq \frac{C \max  f}{\kappa} + \mathrm{Lip}(f).
%\end{equation*}
%Applying Theorem \ref{thm:rate_doubling1} for 

%\end{proof}


%\nocite{*}


%\clearpage


%\section{Rate of convergence via nonlinear adjoint method}

\begin{appendices}
\section{Proofs of some preliminary results}
\begin{proof}[Proof of Theorem \ref{thm:grad_1}] Let $\theta\in (0,1)$ be chosen later, $\varphi\in \mathrm{C}_c^\infty(\Omega)$, $0\leq \varphi\leq 1$, $\mathrm{supp}\;\varphi\subset \Omega$ and $\varphi = 1$ on $\Omega_\delta$ such that
\begin{equation}\label{e:ass_power}
    |\Delta \varphi(x)| \leq C|\varphi(x)|^\theta \qquad\text{and}\qquad |D \varphi(x)|^2 \leq C\varphi^{1+\theta}
\end{equation}
for $x\in \Omega$ and $C = C(\delta,\theta)$.
Let $w(x) = |Du(x)|^2$ for $x\in \Omega$. The goal is to derive an equation for $\varphi w$. The equation for $w$ is given by
\begin{equation*}
    -\varepsilon \Delta w + p|D u|^{p-2}D u \cdot D w + 2  w - D f\cdot D u + \varepsilon |D^2u|^2 = 0 \qquad\text{in}\;\Omega.
\end{equation*}
Using that, we can derive an equation for $(\varphi w)$ as follows.
\begin{align*}
    &-\varepsilon \Delta (\varphi w) + p|D u|^{p-2}D u \cdot D (\varphi w) + 2  (\varphi w) + \varepsilon \varphi|D^2u|^2 + 2\varepsilon \frac{D \varphi}{\varphi}\cdot D (\varphi w) \\
    &\qquad\qquad\qquad\qquad = \varphi(D f\cdot D u) + p|D u|^{p-2}(D u\cdot D \varphi)w -\varepsilon w \Delta \varphi + 2\varepsilon \frac{|D \varphi|^2}{\varphi}w
    \qquad\text{in}\;\mathrm{supp}\;\varphi.
\end{align*}
Assume that $\varphi w$ achieves its maximum over $\overline{\Omega}$ at $x_0\in \Omega$. Then we can further assume that $x_0\in \mathrm{supp}\;\varphi$, since otherwise the maximum of $\varphi w$ over $\overline{\Omega}$ is zero. By the classical maximum principle 
\begin{equation*}
    -\varepsilon \Delta(\varphi w)(x_0)\geq 0 \qquad\text{and}\qquad |D(\varphi w)(x_0)| = 0.
\end{equation*}
Using that in the equation of $\varphi w$ above, we obtain that
\begin{equation*}
    \varepsilon \varphi|D^2u|^2 \leq  \varphi (Df\cdot Du)+ p|Du|^{p-1} |D\varphi|w + \varepsilon w |\Delta\varphi|  + 2\varepsilon  w\frac{|D\varphi|^2}{\varphi},
\end{equation*}
 where all terms are evaluated at $x_0$. Using \eqref{e:ass_power} we have
\begin{equation}\label{e:est_for_D^2u}
    \varepsilon \varphi|D^2u|^2 \leq  \varphi |Df|w^{\frac{1}{2}}+ Cp w^{\frac{p-1}{2}+1} \varphi^{\frac{1+\theta}{2}} + C\varepsilon w \varphi^{\theta} + 2C\varepsilon  w\varphi^\theta.
\end{equation}
By Cauchy-Schwartz inquality, we have $n|D^2u|^2\geq (\Delta u)^2$, thus if $n\varepsilon < 1$ then
\begin{equation}\label{e:est_for_D^2u_2}
    \varepsilon |D^2u|^2 \geq \frac{(\varepsilon \Delta u)^2}{n\varepsilon} \geq (\varepsilon \Delta u)^2 = \left(  u + |Du|^p - f\right)^2 \geq |Du|^{2p} - 2C_0|Du|^p \geq \frac{|Du|^{2p}}{2} - 2K_0.
\end{equation}
Using \eqref{e:est_for_D^2u_2} in \eqref{e:est_for_D^2u} we obtain that
\begin{equation*}
    \varphi\left(\frac{1}{2}w^p - 2K_0\right) \leq \varphi |Df|w^{\frac{1}{2}}+ Cp w^{\frac{p-1}{2}+1} \varphi^{\frac{1+\theta}{2}} + 3C\varepsilon w \varphi^{\theta}.
\end{equation*}
Multiplying both sides by $\varphi^{p-1}$, we deduce that
\begin{align*}
    (\varphi w)^p \leq 4K_0\varphi^{p-1} + 2\Vert Df\Vert_{L^\infty}\varphi^p w^{\frac{1}{2}} + 2Cp \varphi^{\frac{2p+\theta - 1}{2}}w^{\frac{p+1}{2}} + 6C\varepsilon \varphi^{p+\theta - 1}w.
\end{align*}
Choose $2p+\theta -1 \geq p+1$, i.e., $p+\theta\geq 2$. This is always possible with the requirement $\theta \in (0,1)$, as $1<p <\infty$. We deduce that 
\begin{equation}\label{e:est_for_D^2u_3}
    (\varphi w)^p \leq C\left(1+ (\varphi w)^\frac{1}{2} + (\varphi w)^\frac{p+1}{2} +(\varphi w)\right)
\end{equation}
As a polynomial in $z = (\varphi w)(x_0)$, this implies that $(\varphi w)(x_0)\leq C$ where $C$ depends on coefficients of the right hand side of \eqref{e:est_for_D^2u_3}, which implies our desired gradient bound since $\overline{\Omega}_\delta\subset \mathrm{supp}\;\varphi$.
\end{proof}



\begin{proof} [Proof of Theorem \ref{thm:wellposed1<p<2}] If $p\in (1,2)$, we use the ansatz $ u(x) = C_\varepsilon d(x)^{-\alpha}$ to find a solution for \eqref{eq:PDEeps}. Plugging it into \eqref{eq:PDEeps}, we obtain that 
\begin{equation*}
    |Du (x)|^p = \frac{(\alpha C_\varepsilon)^p }{d(x)^{p(\alpha+1)}}|D d(x)|^p \qquad\text{and}\qquad\varepsilon\Delta u(x) = \frac{\varepsilon C_\varepsilon\alpha(\alpha+1)}{d(x)^{\alpha+2}}|D d(x)|^2 - \frac{\varepsilon C_\varepsilon\alpha}{d(x)^{\alpha+1}}\Delta d(x).
\end{equation*}
Since $|D d(x)| = 1$ for $x$ near $\partial\Omega$, as $x\to \partial \Omega$, the highest explosive order terms are
\begin{equation*}
         C_\varepsilon^p \alpha^p d^{-(\alpha+1)p}  -\varepsilon C_\varepsilon \alpha(\alpha+1)d^{-(\alpha+2)}.
\end{equation*}
Setting the above to be zero, we deduce that 
\begin{equation}\label{e:relation}
    \displaystyle\alpha = \frac{2-p}{p-1} \qquad\text{and}\qquad C_\varepsilon = \left(\frac{1}{\alpha}(\alpha+1)^\frac{1}{p-1}\right) \varepsilon^{\frac{1}{p-1}} = \frac{1}{\alpha}(\alpha+1)^{\alpha+1}\varepsilon^{\alpha+1}.
\end{equation}
For $0<\delta < \frac{1}{2}\delta_0$ and $\eta$ small, define
\begin{equation*}
\begin{split}
    \overline{w}_{\eta,\delta}(x) &= \frac{(C_\alpha+\eta)\varepsilon^{\alpha+1}}{(d(x)-\delta)^\alpha} + M_\eta, \qquad x\in \Omega_\delta,\\
    \underline{w}_{\eta,\delta}(x) &= \frac{(C_\alpha-\eta)\varepsilon^{\alpha+1}}{(d(x)+\delta)^\alpha} - M_\eta, \qquad x\in \Omega^\delta,
\end{split}
\end{equation*}
where $C_\alpha = \frac{1}{\alpha} (\alpha+1)^{\alpha+1} $, $M_\eta$ to be chosen. We will show that $\overline{w}_{\eta,\delta}$ is a supersolution of \eqref{eq:PDEeps} in $\Omega_\delta$, while $\underline{w}_{\eta,\delta}$ is a subsolution of \eqref{eq:PDEeps} in $\Omega^\delta$. We have 
\begin{align*}
    \mathcal{L}^\varepsilon\left[\overline{w}_{\eta,\delta}\right](x) &= \frac{ (C_\alpha + \eta)\varepsilon^{\alpha+1}}{(d(x)-\delta)^\alpha} + M_\eta + \frac{(C_\alpha+\eta)^p \alpha^p\varepsilon^{\alpha+2}}{(d(x)-\delta)^{\alpha+2}}|D d(x)|^p - f(x) \\
    & \qquad\qquad\qquad\qquad\qquad - \frac{(C_\alpha+\eta)\alpha(\alpha+1)\varepsilon^{\alpha+2}}{(d(x)-\delta)^{\alpha+2}}|D d(x)|^2 + \frac{(C_\alpha+\eta)\alpha \varepsilon^{\alpha+2}}{(d(x)-\delta)^{\alpha+1}}\Delta d(x)\\
    &\geq M_\eta - f(x) + \underbrace{\frac{\nu C_\alpha\alpha(\alpha+1)\varepsilon^{\alpha+2}}{(d(x)-\delta)^{\alpha+2}}\left[\nu^{p-1}|Dd(x)|^{p-1} - |Dd(x)|^2 + \frac{(d(x)-\delta)\Delta d(x)}{\alpha+1}\right]}_{I}
\end{align*}
where we use $(C_\alpha\alpha)^p = C_\alpha\alpha(\alpha+1)$ and $\nu = \frac{C_\alpha+\eta}{C_\alpha} \in (1,2)$ for small $\eta$. Let
\begin{equation*}
    \delta_\eta : = \frac{\alpha+1}{K_2}\left[\nu^{p-1}-1\right] \to 0 \qquad\text{as}\qquad \eta \to 0.
\end{equation*}
Recall \eqref{boundond} and there are two cases to be considered.
\begin{itemize}
    \item If $0< d(x)-\delta <\delta_\eta < \delta_0$ for $\eta$ small, then $|Dd(x)| = 1$, and thus $I\geq 0$. Hence $\mathcal{L}^\varepsilon\left[\overline{w}_{\eta,\delta}\right]\geq 0$ if we choose $M_\eta \geq \max_{\overline{\Omega}} f$.
    
    \item If $d(x)-\delta\geq \delta_\eta$, then 
    \begin{equation*}
        I \leq \left(\frac{1}{\delta_\eta}\right)^{\alpha+2}\nu C_\alpha \alpha(\alpha+1)\left[\nu^{p-1}K_1^{p-1}+K_1^2+K_2K_0\right]\varepsilon^{\alpha+2}.
    \end{equation*}
    Therefore, if we choose $M_\eta = \sup_{\overline{\Omega}} f + C\varepsilon^{\alpha+2}$ for $C$ large enough (depending on $\eta$), then $\mathcal{L}^\varepsilon\left[\overline{w}_{\eta,\delta}\right]\geq 0$.
\end{itemize}
Therefore, $\overline{w}_{\eta,\delta}$ is a supersolution in $\Omega_\delta$. Similarly, we have 
\begin{align*}
    \mathcal{L}_\varepsilon\left[\underline{w}_{\eta,\delta}\right](x) &= \frac{ (C_\alpha - \eta)\varepsilon^{\alpha+1}}{(d(x)+\delta)^\alpha} - M_\eta + \frac{(C_\alpha-\eta)^p \alpha^p\varepsilon^{\alpha+2}}{(d(x)+\delta)^{\alpha+2}}|D d(x)|^p - f(x) \\
    & \qquad\qquad\qquad\qquad\qquad - \frac{(C_\alpha-\eta)\alpha(\alpha+1)\varepsilon^{\alpha+2}}{(d(x)+\delta)^{\alpha+2}}|D d(x)|^2 + \frac{(C_\alpha-\eta)\alpha \varepsilon^{\alpha+2}}{(d(x)+\delta)^{\alpha+1}}\Delta d(x)\\
    &= - M_\eta - f(x) \\
    &+\underbrace{\frac{\nu C_\alpha\alpha(\alpha+1)\varepsilon^{\alpha+2}}{(d(x)+\delta)^{\alpha+2}}\left[\nu^{p-1}|Dd(x)|^{p-1} - |Dd(x)|^2 + \frac{(d(x)+\delta)\Delta d(x)}{\alpha+1}+\frac{(d(x)+\delta)^2}{\alpha(\alpha+1)\varepsilon}\right]}_{J}
\end{align*}
where $\nu = \frac{C_\alpha-\eta}{C_\alpha}\in (0,1)$ for small $\eta$. Let
\begin{equation*}
    \delta_\eta := \left(1-\nu^{p-1}\right)\left(\frac{1+K_2\alpha\varepsilon}{\alpha(\alpha+1)\varepsilon}\right) \to 0 \qquad\text{as} \qquad \eta\to 0.
\end{equation*}
Recall \eqref{boundond}, and there are two cases.
\begin{itemize}
    \item $0<d(x)+\delta<\delta_\eta < \delta_0$ for $\eta$ small, then $|Dd(x)| = 1$ , and thus $J\leq 0$. Hence $\mathcal{L}^\varepsilon\left[\underline{w}_{\eta,\delta}\right]\leq 0$ if we choose $M_\eta \geq -\max_{\Omega}f$.
    \item If $d(x)+\delta\geq \delta_n$, then
    \begin{equation*}
        |J|\leq \left(\frac{1}{\delta_n}\right)^{\alpha+2} \nu C_\alpha\alpha(\alpha+1)\left[K_1^{p-1}+K_1^2 + \frac{K_0K_2}{\alpha+1} + \frac{K_0^2}{\alpha(\alpha+1)\varepsilon}\right]\varepsilon^{\alpha+2}
    \end{equation*}
     Therefore, if we choose $M_\eta = -\sup_{\overline{\Omega}} f - C\varepsilon^{\alpha+2}$ for $C$ large enough (depending on $\eta$), then $\mathcal{L}^\varepsilon\left[\underline{w}_{\eta,\delta}\right]\leq 0$.
\end{itemize}
For $p=2$, we use the ansatz $u(x) = -C_\varepsilon \log(d(x))$ instead. Similarly to the previous case, we find $u(x) = -\varepsilon\log(d(x))$. For $0<\delta<\frac{1}{2}\delta_0$, we define 
\begin{equation*}
    \begin{split}
        &\overline{w}_{\eta,\delta}(x) = -(1+\eta)\varepsilon\log(d(x)-\delta) + M_\eta, \qquad x\in \Omega_\delta,\\
        &\underline{w}_{\eta,\delta}(x) = -(1-\eta)\varepsilon\log(d(x)-\delta) - M_\eta, \qquad x\in \Omega^\delta,
    \end{split}
\end{equation*}
where $M_\eta$ is to be chosen so that $\overline{w}_{\eta,\delta}(x)$ is a supersolution in $\Omega_\delta$ and $\underline{w}_{\eta,\delta}$ is a subsolution in $\Omega^\delta$. We omit the computations, as they are similar to the previous case.
\smallskip

\noindent We divide the rest of the proof into 3 steps. We first construct a minimal solution, then a maximal solution to \eqref{eq:PDEeps}, and finally show that they are equal to conclude the existence and uniqueness of solution to \eqref{eq:PDEeps}.
\smallskip

\paragraph{\textbf{Step 1.}} There exists a minimal solution $\underline{u}\in \mathrm{C}^2(\Omega)$ of \eqref{eq:PDEeps} such that $v\geq \underline{u}$ for any other solution $v\in \mathrm{C}^2(\Omega)$ solving \eqref{eq:PDEeps}.

\begin{proof} Let $w_{\eta,\delta}\in \mathrm{C}^2(\Omega)$ solve
    \begin{equation}\label{e:w_def}
    \begin{cases}
        \mathcal{L}^\varepsilon\left[w_{\eta,\delta}\right] = 0 &\qquad\text{in}\;\Omega,\\
        \qquad w_{\eta,\delta} = \underline{w}_{\eta,\delta} &\qquad\text{on}\;\partial\Omega.
    \end{cases}
    \end{equation}
    \begin{itemize}
        \item Fix $\eta>0$. As $\delta\to 0$, the value of $\underline{w}_{\eta,\delta}$ blows up on the boundary. Therefore, by the standard comparison principle for the second-order elliptic equation with Dirichlet boundary, $\delta_1 \leq  \delta_2$ implies $w_{\eta,\delta_1}\geq  w_{\eta,\delta_2}$ on $\overline{\Omega}$. 
        \item For $\delta'>0$, since $\underline{w}_{\eta,\delta'}$ is a subsolution in $\overline{\Omega}$ with finite boundary, we obtain that
            \begin{equation}\label{e:cp_delta1}
                0<\delta \leq \delta'\qquad\Longrightarrow\qquad \underline{w}_{\eta,\delta'} \leq w_{\eta_,\delta'}\leq w_{\eta,\delta} \qquad\text{on}\;\overline{\Omega}.
            \end{equation}
        \item Similarly, since $\overline{w}_{\eta,\delta'}$ is a supersolution on $\Omega_{\delta'}$ with infinity value on the boundary $\partial\Omega_{\delta'}$, by comparison principle
            \begin{equation}\label{e:cp_delta2}
                w_{\eta,\delta} \leq \overline{w}_{\eta, \delta'} \qquad\text{in}\;\Omega_{\delta'} \qquad\Longrightarrow\qquad w_{\eta,\delta} \leq \overline{w}_{\eta,0} \qquad\text{in}\;\Omega.
            \end{equation}
    \end{itemize}
    \noindent From \eqref{e:cp_delta1} and \eqref{e:cp_delta2}, we have
    \begin{equation}\label{e:cp_delta3}
        0<\delta \leq \delta'\qquad\Longrightarrow\qquad \underline{w}_{\eta,\delta'} \leq w_{\eta_,\delta'}\leq w_{\eta,\delta} \leq \overline{w}_{\eta,0} \qquad\text{in}\;\Omega.
    \end{equation}
    Thus, $\{w_{\eta,\delta}\}_{\delta>0}$ is locally bounded in $L^{\infty}_{\mathrm{loc}}(\Omega)$ ($\{w_{\eta,\delta}\}_{\delta>0}$ is uniformly bounded from below). Using the local gradient estimate for $w_{\eta,\delta}$ solving \eqref{e:w_def}, we deduce that $\{w_{\eta,\delta}\}_{\delta>0}$ is locally bounded in $W^{1,\infty}_{\mathrm{loc}}(\Omega)$. Plugging it back into the defining equation \eqref{e:w_def}, we have that $\{w_{\eta,\delta}\}_{\delta>0}$ is locally bounded in $W^{2,r}_{\mathrm{loc}}(\Omega)$ for all $r<\infty$.
    
    \noindent Local boundedness of $\{w_{\eta,\delta}\}_{\delta>0}$ in $W^{2,r}_{\mathrm{loc}}(\Omega)$ implies weak$^*$ compactness, that is, there exists a function $u\in W^{2,r}_{\mathrm{loc}}(\Omega)$ such that (via subsequence and monotonicity)
    \begin{equation*}
        w_{\eta,\delta} \rup u \qquad\text{weakly in}\;W^{2,r}_{\mathrm{loc}}(\Omega),\qquad \text{and}\qquad
        w_{\eta,\delta} \to u \qquad\text{strongly in}\;W^{1,r}_{\mathrm{loc}}(\Omega).
    \end{equation*}
    In particular, $w_{\eta,\delta}\to u$ in $\mathrm{C}^1_{\mathrm{loc}}(\Omega)$ thanks to Sobolev compact embedding. Let us rewrite the equation $\mathcal{L}^\varepsilon\left[w_{\eta,\delta}\right] = 0$ as $\varepsilon\Delta w_{\eta,\delta}(x) = F[w_{\eta,\delta}](x)$ in $U$ for $U\subset\subset \Omega$ where
    \begin{equation*}
        F[w_{\eta,\delta}](x) =    w_{\eta,\delta}(x) + H(x,Dw_{\eta,\delta}(x)).
    \end{equation*}
    Since $w_{\eta,\delta}\to u$ in $\mathrm{C}^1(U)$ as $\delta\to 0$, we have $F[w_{\eta,\delta}](x) \to F(x)$ uniformly in $U$ where 
    \begin{equation*}
        F(x) =   u(x) + H(x,Du(x)).
    \end{equation*}
    In the limit, as $\delta\to 0$, we obtain that $u\in L^2(U)$ is a weak solution of $\varepsilon\Delta u = F$ in $U$ where $F$ is continuous, thus $u\in \mathrm{C}^2(\Omega)$ and by stability $u$ solves $\mathcal{L}^\varepsilon[u] = 0$ in $\Omega$. From \eqref{e:cp_delta3} we also have
    \begin{equation*}
        \underline{w}_{\eta,0} \leq u \leq \overline{w}_{\eta,0} \qquad\text{in}\;\Omega.
    \end{equation*}
    It is clear that $u(x)\to \infty$ as $\mathrm{dist}(x,\partial\Omega)\to 0$ with the precise rate like \eqref{rate_p<2} or \eqref{rate_p=2}. By construction, $u$ may depend on $\eta$. Next, we show that $u$ is the minimal solution of $\mathcal{L}^\varepsilon[u] = 0$ in $\Omega$ such that $u = +\infty$ on $\partial\Omega$, thus consequently showing that $u$ is independent of $\eta$.
    
    \noindent Let $v\in W^{2,r}(\Omega)$ for all $r<\infty$ and $v$ solves \eqref{eq:PDEeps}. Fix $\delta>0$. Then since $v(x)\to \infty$ as $x\to \partial\Omega$ while $w_{\eta,\delta}$ remains bounded on $\partial \Omega$, comparison principle yields
    \begin{equation*}
        v\geq w_{\eta,\delta} \qquad\text{in} \; \Omega.
    \end{equation*}
    Let $\delta\to 0$ we deduce that $v\geq u$ on $\Omega$. This concludes that $u$ is the minimal solution in $W^{2,r}(\Omega)(\forall\,r<\infty)$ and thus $u$ is independent of $\eta>0$. 
\end{proof}


\paragraph{\textbf{Step 2.}} There exists a maximal solution $\overline{u}\in \mathrm{C}^2(\Omega)$ of \eqref{eq:PDEeps} such that $v\leq \overline{u}$ for any other solution $v\in \mathrm{C}^2(\Omega)$ solving \eqref{eq:PDEeps}.


\begin{proof} For each $\delta>0$, let us denote $u_\delta\in \mathrm{C}^2(\Omega_\delta)$ to be the minimal solution to $\mathcal{L}[u_\delta] = 0$ in $\Omega_\delta$ and $u_\delta = +\infty$ on $\partial\Omega_\delta$. By the comparison principle, for every $\eta>0$, there holds
\begin{equation*}
    \underline{w}_{\eta,\delta} \leq u_\delta \leq \overline{w}_{\eta,\delta} \qquad\text{in}\;\Omega_\delta,
\end{equation*}
and
\begin{equation*}
    0<\delta<\delta' \qquad \Longrightarrow\qquad u_\delta \leq u_\delta' \qquad\text{in}\;\Omega_{\delta'}.
\end{equation*}
The monotoniciy, together with the local boundedness of $\{u_\delta\}_{\delta>0}$ in $W^{2,r}(\Omega)$, implies there exists $u\in W^{2,r}(\Omega)$ for all $r<\infty$ such that $u_\delta\to u$ strongly in $\rmC^1_{\mathrm{loc}}(\Omega)$. Therefore, using the equation $\mathcal{L}[u_\delta] = 0$ in $\Omega_\delta$ and the regularity of Laplace equation, we deduce that $u\in \mathrm{C}^2(\Omega)$ solves \eqref{eq:PDEeps} and 
\begin{equation*}
    \underline{w}_{\eta,0} \leq u\leq \overline{w}_{\eta,0} \qquad\text{in}\;\Omega
\end{equation*}
for all $\eta>0$. From the construction, as $u_\delta$ is independent of $\eta$, it is clear that $u$ is also independent of $\eta$. We now show that $u$ is the maximal solution of \eqref{eq:PDEeps}. Let $v\in\rmC^2(\Omega)$ solve \eqref{eq:PDEeps}. Clearly $v\leq u_\delta$ on $\Omega_\delta$. Therefore, as $\delta \to 0$, we have $v\leq u$.
\end{proof}
\noindent In conclusion, we have found a minimal solution $\underline{u}$ and a maximal solution $\overline{u}$ in $\rmC^2(\Omega)$ such that
\begin{equation}\label{e:chain}
    \underline{w}_{\eta,0} \leq \underline{u}\leq \overline{u}\leq \overline{w}_{\eta,0} \qquad\text{in}\;\Omega
\end{equation}
for any $\eta>0$. This extra parameter $\eta$ now enables us to show that $\overline{u} = \underline{u}$ in $\Omega$. The key ingredient here is the convexity in the gradient of the operator.
\smallskip
\paragraph{\textbf{Step 3.}} We have $\overline{u}\equiv \underline{u}$ in $\Omega$. Therefore, the solution to \eqref{eq:PDEeps} in $\mathrm{C}^2(\Omega)$ is unique.

\begin{proof} Let $\theta\in (0,1)$. Define $w_\theta = \theta \overline{u} + (1-\theta) ^{-1}\inf_{\Omega} f$. If $w_\theta$ is a subsolution to \eqref{eq:PDEeps}, then we can use comparison principle to obtain that
\begin{equation*}
    w_\theta = \theta \overline{u} + (1-\theta) ^{-1}\inf_{\Omega} f\leq \underline{u} \qquad\text{in}\;\Omega.
\end{equation*}
If that is the case, let $\theta\to 1$ we conclude that $\overline{u} \leq \underline{u}$. There is a problem here. As they are both explosive solutions, to use comparison principle, we need to show that $w_\theta \leq \underline{u}$ in a neighborhood of $\partial\Omega$. From \eqref{e:chain}, we see that
\begin{align*}
    &1\leq \frac{\overline{u}(x)}{\underline{u}(x)} \leq \frac{\overline{w}_{\eta,0}(x)}{\underline{w}_{\eta,0}(x)} = \frac{(C_\alpha+\eta)+ M_\eta d(x)^\alpha}{(C_\alpha-\eta)- M_\eta d(x)^\alpha},& 1<p<2,\\
    &1 \leq \frac{\overline{u}(x)}{\underline{u}(x)} \leq \frac{\overline{w}_{\eta,0}(x)}{\underline{w}_{\eta,0}(x)} = \frac{-(1+\eta)\log(d(x)) + M_\eta}{-(1-\eta)\log(d(x))-M_\eta}, & p=2,
\end{align*}
for $x\in \Omega$. Hence
\begin{align*}
     &1\leq  \lim_{d(x)\to 0} \left(\frac{\overline{u}(x)}{\underline{u}(x)}\right) \leq \frac{C_\alpha+\eta}{C_\alpha-\eta}, & 1< p < 2,\\
     &1\leq  \lim_{d(x)\to 0} \left(\frac{\overline{u}(x)}{\underline{u}(x)}\right) \leq \frac{-(1+\eta)}{-(1-\eta)}, & p = 2.
\end{align*}
Since $\eta>0$ is chosen arbitrary, we obtain
\begin{equation*}
     \lim_{d(x)\to 0} \left(\frac{\overline{u}(x)}{\underline{u}(x)}\right) = 1.
\end{equation*}
 This means for any $\varsigma\in(0,1)$, we can find $\delta(\varsigma)>0$ small such that on $\Omega\backslash \Omega_\delta$, one has
\begin{equation*}
\frac{\overline{u}(x)}{\underline{u}(x)}\leq (1+\varsigma)     \qquad\Longrightarrow\qquad \left(\frac{1}{1+\varsigma}\right)\overline{u}(x) \leq \underline{u}(x) \qquad\text{in}\; \Omega\backslash \Omega_\delta.
\end{equation*}
For a given $\theta\in (0,1)$, we can always choose $\varsigma$ small enough such that $(1+\varsigma)^{-1} > \theta$, which will imply that $\underline{u}(x) \geq \theta \overline{u}(x) + (1-\theta) ^{-1}\left(\inf_\Omega f\right)$ for $0< d(x) < \delta' < \delta$.
\end{proof}
\noindent This finishes the well-posedness of \eqref{eq:PDEeps} for $1<p\leq2$. 
\end{proof}


\begin{proof}[Proof of Lemma \ref{lem:max}] The proof is a variation of Perron's method (see \cite{Capuzzo-Dolcetta1990}). Let $\varphi\in \rmC(\overline{\Omega})$ and $x_0\in \overline{\Omega}$ such that $u(x_0) = \varphi(x_0)$ and $u-\varphi$ has a global strict minimum over $\overline{\Omega}$ at $x_0$ and that 
\begin{equation}\label{eq:max_a1}
      \varphi(x_0) + H(x_0,D\varphi(x_0)) < 0.
\end{equation}
Let $\varphi^\varepsilon(x) = \varphi(x) - |x-x_0|^2 + \varepsilon$ for $x\in \overline{\Omega}$. Let $\delta > 0$. We see that for $x\in \partial B(x_0,\delta)\cap \overline{\Omega}$,
\begin{equation*}
    \varphi^\varepsilon(x) = \varphi(x) - \delta^2 +\varepsilon \leq \varphi(x) - \varepsilon
\end{equation*}
if $2\varepsilon \leq \delta^2$. We observe that
\begin{equation*}
    \begin{split}
    \varphi^\varepsilon(x) - \varphi(x_0)  &= \varphi(x)-\varphi(x_0) + \varepsilon - |x-x_0|^2 \\
    D\phi^\varepsilon(x) - D\phi(x_0) &= D\varphi(x) - D\varphi(x_0) - 2(x-x_0)
    \end{split}
\end{equation*}
for $x\in B(x,\delta)\cap \overline{\Omega}$. We deduce from \eqref{eq:max_a1}, the continuity of $H(x,p)$ near $(x_0,D\varphi(x_0))$ and the fact that $\varphi\in \rmC^1(\overline{\Omega})$ that if $\delta$ is small enough and $0<2\varepsilon < \delta^2$, then
\begin{equation}\label{eq:max_a2}
      \varphi^\varepsilon(x)+H(x,D\varphi^\varepsilon(x)) < 0 \qquad\text{for}\;x\in B(x_0,\delta)\cap \overline{\Omega}.
\end{equation}
We have found $\phi^\varepsilon\in \mathrm{C}^1(\overline{\Omega})$ such that $\varphi^\varepsilon(x_0)>u(x_0)$, $\varphi^\varepsilon<u$ on $\partial B(x_0,\delta)\cap \overline{\Omega}$ and \eqref{eq:max_a2}. Let
\begin{equation*}
    \tilde{u}(x) = \begin{cases}
    \max \big\lbrace u(x),\phi^\varepsilon(x) \big\rbrace &x\in B(x_0,\delta)\cap \overline{\Omega},\\
    u(x)&x\notin B(x_0,\delta)\cap \overline{\Omega}.\\
    \end{cases}
\end{equation*}
We see that $\tilde{u}\in \rmC(\overline{\Omega})$ is a subsolution of \eqref{S_0} in $\Omega$ with $\tilde{u}(x_0) > u(x_0)$, which is a contradiction. Thus, $u$ is a supersolution of \eqref{S_0} on $\overline{\Omega}$.
\end{proof}


\section{Differentiability with respect to the parameter}
\noindent For the vanishing viscosity problem with the Dirichlet boundary condition,
\begin{equation}
\label{dir}
\left\{
  \begin{aligned}
    H(x, Du^\epsilon(x)) &= \epsilon \Delta u^\epsilon \quad \, \text{in } U, \\
              u^\epsilon &= 0 \quad \qquad \text{on } \partial U,
  \end{aligned}
\right.
\end{equation}
where  $H(x,p)$ is $C^\infty(\overline{U}\times \mathbb{R}^n)$, $\displaystyle \frac{H(x,p)}{|p|} \to \infty$ uniformly in $x$ as $|p| \to \infty$ and $\displaystyle \sup_{x\in U}|D_xH(x,p)|\leq C(1+|p|)$, we want to show the smooth dependence of $u^\epsilon$ on $\epsilon$.
Formally, if we differentiate \eqref{dir} with respect to $\epsilon$, we get
\begin{equation}
\label{dir_dif}
\left\{
  \begin{aligned}
    D_pH(x, Du^\epsilon(x))\cdot Du^\epsilon_\epsilon &= \epsilon \Delta u^\epsilon_\epsilon +\Delta u^\epsilon \quad \, \text{in } U, \\
              u^\epsilon_\epsilon &= 0 \quad \qquad  \qquad  \text{on } \partial U.
  \end{aligned}
\right.
\end{equation}
By Schaefer's fixed point theorem and the maximal principle, $u^\epsilon_\epsilon$ is the unique solution in $C^{2,\alpha}(\overline{U})$ of \eqref{dir_dif}. 

\noindent The main idea is, we look at the difference quotients $\displaystyle \frac{u^{\epsilon+h}-u^\epsilon}{h}$and prove that as $h \to 0^+$, they converge to a limiting function $w^{\ast}$ in the uniform norm  such that $w^{\ast}$ solves \eqref{dir_dif}. Since \eqref{dir_dif} has a unique solution, we have $$u^\epsilon_\epsilon=\lim_{h \to 0^+}\frac{u^{\epsilon+h}-u^\epsilon}{h}.$$

\subsection{Solution $u^\epsilon \in C^{2,\alpha}(\overline{U})$ exists} We use Schaufer's fixed point theorem as follows.

\begin{thm} Suppose $X$ is a Banach space. Let $A:X \to X$ be continuous and compact. Assume the set $\{u\in X : u=  A[u]$ for some $0 \leq    \leq 1\}$ is bounded. Then $A$ has a fixed point $u =A[u]$.
\end{thm}
\noindent Fix $0<\alpha<1$. Let $X=C^{1,\alpha}(\overline{U})$. Given $u \in X=C^{1,\alpha}(\overline{U})$, we look at the linear PDE
\begin{equation}
\label{fix}
\left\{
  \begin{aligned}
   \epsilon \Delta v &= H(x, Du) \quad \, \text{in } U, \\
              v &= 0 \qquad \qquad \text{on } \partial U.
  \end{aligned}
\right.
\end{equation}
\noindent
Estimate the Holder norm of RHS
$$\|H(x, Du)\|_{C^{0, \alpha}(\overline{U})}:=\sup_{x\in\overline{U}} |H(x,Du(x))| + \sup_{x, y \in \overline{U}}\frac{|H(x, Du(x))-H(y, Du(y))|}{|x-y|^\alpha}.$$
Since $Du$ is bounded and H is smooth,
\begin{equation}
    \sup_{x\in\overline{U}} |H(x,Du(x))| \leq C.
\end{equation}

\begin{equation}
    \begin{aligned}
  &\sup_{x, y \in \overline{U}}\frac{|H(x, Du(x))-H(y, Du(y))|}{|x-y|^\alpha}\\
  \leq &  \sup_{x, y \in \overline{U}}\frac{|H(x, Du(x))-H(y, Du(x))|}{|x-y|^\alpha} + \sup_{x, y \in \overline{U}}\frac{|H(y, Du(x))-H(y, Du(y))|}{|x-y|^\alpha} \\
  =&    \sup_{x, y \in \overline{U}}\frac{|\int_0^1D_xH(y+\theta (x-y), Du(x))d\theta \cdot (x-y)|}{|x-y|^\alpha}\\& +
     \sup_{x, y \in \overline{U}}\frac{|\int_0^1D_pH(y, Du(y)+\theta (Du(x)-Du(y)))d\theta \cdot (Du(x)-Du(y))|}{|x-y|^\alpha}\\
  \leq &  C\sup_{x, y \in \overline{U}} (1+|Du(x)|)|x-y|^{1-\alpha} +  C \sup_{x, y \in \overline{U}}\frac{| (Du(x)-Du(y))|}{|x-y|^\alpha}\\
 \leq & C(1 + \|u\|_{C^{1,\alpha}(\overline{U})})
    \end{aligned}
\end{equation}
since $Du$ is bounded on $\overline{U}$, $D_pH \in C^\infty (\overline{U} \times \mathbb{R}^n)$ and
$\displaystyle \sup_{x\in U}|D_xH(x,p)|\leq C(1+|p|)$. Therefore,
\begin{equation}
 \|H(x, Du)\|_{C^{0, \alpha}(\overline{U})} \leq C(1 + \|u\|_{C^{1,\alpha}(\overline{U})}).
\end{equation}
By Schauder estimates, there exists a unique solution $v \in C^{2,\alpha }(\overline{U})$ such that
\begin{equation}
\label{schauder}
    \|v\|_{C^{2, \alpha}(\overline{U})} \leq C  \|H(x, Du)\|_{C^{0, \alpha}(\overline{U})}. 
\end{equation}
\noindent Define operator $A$ on $X := C^{1, \alpha} (\overline{U}) $ by $A[u]=v$. So 
\begin{equation}
     \|A[u]\|_{C^{2, \alpha}(\overline{U})} \leq C(1 + \|u\|_{C^{1,\alpha}(\overline{U})}),
\end{equation}
\noindent
 and thus $A$ is continuous and compact. (i.e., if $\{u_k\}_{k=1}^\infty$ is bounded in $X=C^{1, \alpha}(\overline{U})$, then $\{A[u_k]\}_{k=1}\infty$ is bounded in $C^{2,\alpha} (\overline{U})$, thus precompact in $C^{1, \alpha}(\overline{U})$. Lemma 6.36 in Gilbarg and Trudinger. )
 
\noindent Next we try to bound $\{u\in X : u=  A[u]$ for some $0 \leq    \leq 1\}$. If $u=  A[u]$, the PDE becomes
 
\begin{equation}
\label{lamb}
\left\{
  \begin{aligned}
   \epsilon \Delta u &=   H(x, Du) \quad \, \text{in } U, \\
              u &= 0 \qquad \qquad \text{on } \partial U.
  \end{aligned}
\right.
\end{equation}

\noindent Calderon-Zygmund estimates tell us that if we have
\begin{equation}
\label{cald}
\left\{
  \begin{aligned}
  - \Delta v &=\Tilde{f}   \qquad \, \text{in } U, \\
              v &= 0 \qquad \text{on } \partial U.
  \end{aligned}
\right.
\end{equation}
and $\Tilde{f} \in {L^p(U)}$ for some $p\in (1, \infty)$, then $v \in W^{2,p}(U)$ and $$\|v\|_{w^{2,p}(U)} \leq C\|\Tilde{f}\|_{L^p(U)}.$$ 

\noindent Apply to \eqref{lamb} and we get

\begin{equation}
    \|u\|_{w^{2,p}(U)} \leq C\|H(x, Du)\|_{L^p(U)}
\end{equation}


(We want RHS to be bounded by some constant so that later we can choose $p$ larger than $n$ to  conclude $ \|u\|_{C^{1, \alpha}(U)} \leq   C\|u\|_{w^{2,p}(U)} \leq C$ by Morrey's estimate.) By a priori estimate, if we assume the solution of \eqref{dir} $u$ exists, then $\| Du\|_{L^\infty} \leq C_0$ where $C_0$ is independent of $\epsilon$. We can modify $H$ to get a new $\Tilde{H}$ so that it is smooth, $\Tilde{H} = H$ for $|p|<C_1$ and $H(x,p)=C_1+1$ for $|p|>C_1+1$ for some constant $C_1$ ($C_1$ is definitely larger than $C^0$). Moreover, the same prior estimate is correct for $\Tilde{H}$. Namely, if the solution $\Tilde{u}$ to \eqref{dir} with $H$ replaced by $\Tilde{H}$ exists, then $\| D\Tilde{u}\|_{L^\infty} \leq C_0$. (The way we choose this $C_1$ is that we work back from the beginning of the proof of a priori estimate for $u$, on both the boundary and interior of $\Omega$, and modify $H$ so that all the proofs go through and the same a priori estimate holds for the equation with $\Tilde{H}$.) Now with $\Tilde{H}$, we go through the same argument of Schaufer's fixed point theorem from the very beginning, and \eqref{cald} reads
\begin{equation}
       \|\Tilde{u}\|_{w^{2,p}(U)} \leq C\|\Tilde{H}(x, D\Tilde{u})\|_{L^p(U)} \leq C(1+\|D\Tilde{u}\|_{L^\infty}) \leq C.
\end{equation}
Choose $p=2n$ and $\displaystyle \alpha =\frac{1}{2}$. We have $\{\Tilde{u}\in X : \Tilde{u}=  A[\Tilde{u}]$ for some $0 \leq    \leq 1\}$ is bounded in $X= C^{1, \frac{1}{2}}(
\overline{U})$.
Thus Schaefer's fixed point theorem implies the equation \eqref{dir} with $H$ replaced by $\Tilde{H}$ has a solution $\Tilde{u} \in C^{2,\alpha} (\overline{\Omega})$. Since $\| D\Tilde{u}\|_{L^\infty} \leq C_0$, $\Tilde{u}$ also solves the original equation \eqref{dir}.



\subsection{Uniqueness}
Let $u$ and $v$ be two solutions to \eqref{dir} and $w := u-v$. Then we have
\begin{equation}
\begin{aligned}
    &-\epsilon \Delta (u-v) = H(x, Dv)-H(x, Du)\\
    \Rightarrow &-\epsilon \Delta w   =\int_0^1 D_pH(x, tDv+(1-t)Du) \cdot (Dv-Du)dt\\
    \Rightarrow &-\epsilon \Delta w  +\int_0^1 D_pH(x, tDv+(1-t)Du)dt \cdot Dw = 0.
\end{aligned}
\end{equation}
By the strong maximum principle, $w \equiv 0$.



\subsection{Smooth dependence on $\epsilon$}
Fix $\epsilon >0$. Let  $$w^h(x):=\frac{u^{\epsilon+h}(x)-u^\epsilon(x)}{h} \in C^{2,\alpha}(\overline{U}).$$

A little computation shows that $w^h$ solves

\begin{equation}
\label{dir_quo}
\left\{
  \begin{aligned}
   \epsilon \Delta w^h(x) + \frac{\epsilon}{\epsilon + h}\Delta u^\epsilon &= \frac{\epsilon}{\epsilon +h} \int_0^1 D_pH(x, Du^\epsilon+\theta (Du^{\epsilon+h}-Du^\epsilon)) d\theta \cdot Dw^h \quad \, \text{in } U, \\
              w^h &= 0 \quad \qquad \text{on } \partial U.
  \end{aligned}
\right.
\end{equation}
From the existence proof, we know $\|u^\epsilon\|_{C^{2,\alpha}(\overline{U})} \leq C$ uniformly in $\epsilon$. So $\|Du^{\epsilon+h}-Du^\epsilon\|_{C^{0,\alpha}(\overline{U})}$and $\|\Delta u\|_{C^{0,\alpha}(\overline{U})}
$ is uniformly bounded in $h$.  

By Schauder estimates, $\{w^h\}_{h>0} \subset C^{2, \alpha}(\overline{U})$ are bounded, hence is precompact in $ C^{2, \beta}(\overline{U})$ for any $\beta < \alpha$. Therefore, there exits a subsequence $\{w^{h_j}\}_{j=1}^\infty$ such that $w^{h_j} \to w^\ast$ for some $w^\ast \in  C^{2, \beta}(\overline{U}) $ and $w^\ast$ solves \eqref{dir_dif}. This implies $w^h \to w^\ast$ in $C^{2, \beta}(\overline{U})$.
\end{appendices}

\section*{Acknowledgement}
The authors would like to express their appreciation to Hung V. Tran for his invaluable guidance.



\bibliography{zzzzlibrary}{}
%\bibliographystyle{ieeetr}
\bibliographystyle{acm}










\end{document}
