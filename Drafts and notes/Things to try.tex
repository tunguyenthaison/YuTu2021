\documentclass[10pt]{article}
\usepackage[T1]{fontenc}

\usepackage{xcolor}
\usepackage{fullpage}
\usepackage{mathrsfs}
\usepackage{amsmath, amsthm}
%\usepackage[utf8]{vietnam} 
%\usepackage[utf8]{inputenc}
%\usepackage[english,vietnam]{babel}

%\usepackage{fontspec}
%\setmainfont[Ligatures=TeX]{Linux Libertine O}
%\usepackage{polyglossia}
%\setmainlanguage{vietnamese}

%\usepackage[bitstream-charter]{mathdesign}
%\usepackage{eucal}
\usepackage{geometry}
\geometry{verbose,tmargin=2cm,bmargin=2cm,lmargin=2.5cm,rmargin=2.2cm,headheight=2.5cm}

%\usepackage[tracking]{microtype}
\usepackage[sc,osf]{mathpazo}   % With old-style figures and real %smallcaps.
\linespread{1.025}              % Palatino leads a little more leading
% Euler for math and numbers
\usepackage[euler-digits,small]{eulervm}


\usepackage{frcursive}
\usepackage{calligra}
\newcommand{\setfont}[2]{{\fontfamily{#1}\selectfont #2}}



\theoremstyle{plain}
\newtheorem{thm}{Theorem}
\newtheorem{ass}{Assumption}
\renewcommand{\theass}{}
\newtheorem{defn}{Definition}
\newtheorem{quest}{Question}
\newtheorem{com}{Comment}
\newtheorem{ex}{Example}
\newtheorem{lem}[thm]{Lemma}
\newtheorem{cor}[thm]{Corollary}
\newtheorem{prop}[thm]{Proposition}
\theoremstyle{remark}
\newtheorem{rem}{\bf{Remark}}
%\numberwithin{equation}{section}



\usepackage[framemethod=TikZ]{mdframed}
%\newcounter{theo}[section]\setcounter{theo}{0}

%\newcounter{theo}[]\setcounter{theo}{0}
%\renewcommand{\thetheo}{\arabic{section}.\arabic{theo}}
%\renewcommand{\thetheo}{\arabic{theo}}
\newenvironment{theo}[2][]{%
\refstepcounter{theo}%
\ifstrempty{#1}%
{\mdfsetup{%
frametitle={%
\tikz[baseline=(current bounding box.east),outer sep=0pt]
\node[anchor=east,rectangle,fill=blue!20]
%{\strut Theorem~\thetheo};}}
{\strut Question~\thetheo};}}
}%
{\mdfsetup{%
frametitle={%
\tikz[baseline=(current bounding box.east),outer sep=0pt]
\node[anchor=east,rectangle,fill=blue!20]
{\strut Theorem~\thetheo:~#1};}}%
}%
\mdfsetup{innertopmargin=10pt,linecolor=blue!20,%
linewidth=2pt,topline=true,%
frametitleaboveskip=\dimexpr-\ht\strutbox\relax
}
\begin{mdframed}[]\relax%
\label{#2}}{\end{mdframed}}



\begin{document}



\begin{center}
{\LARGE \textsc{Quasilinear equations}}\\
%{Từ Nguyễn Thái Sơn}\\
%{\setfont{calligra}{Son Nguyen Thai Tu}}\\
%{\setfont{frc}{March 13, 2021}}\\
{\textit{March 24, 2021}}
\end{center}



\begin{center}
--------------------------------------------------------------------------------------------------------------------
\end{center}

\begin{enumerate}
    \item Try the special case $\Omega = B(0,1)$, note that we have $\Delta d(x) = 0$ in one dimension. Question: what is the difference between the rotational case (radial) vs one dimensional? 
    \item Try out the special case $f(x) = f(|x|)$ in $B(0,1)$, in which we can reduce it to one-dimensional setting.
    \item Try out the specific case $f(x) = 1-|x|$ on $(-1,1)$ and $f(x) = 0$ in $1\leq |x|\leq 2$ on the domain $\Omega = (-2,2)$. This is a simple case of the compactly supported data.
    \item Idea for general data: Assume $u^\varepsilon\to u$ as $\varepsilon\to 0$. We compare $u^\varepsilon$ and $v^\varepsilon$ where $1<p\leq 2$ and
\begin{equation}\label{eq:PDEeps}
    \begin{cases}
   \mathcal{L}[u^\varepsilon] =  u^\varepsilon(x) + |Du^\varepsilon(x)|^p - f(x) - \varepsilon \Delta u^\varepsilon(x) = 0 &\qquad
    \text{in}\;\Omega, \vspace{0cm}\\
   u^\varepsilon(x) = +\infty &\qquad
    \text{on}\;\partial\Omega
    \end{cases} \tag{PDE$_\varepsilon$}
\end{equation}
and
\begin{equation}\label{eq:vPDEeps}
    \begin{cases}
   \mathcal{L}[v^\varepsilon] =  v^\varepsilon(x) + |Dv^\varepsilon(x)|^p - f(x) - \varepsilon \Delta v^\varepsilon(x) = 0 &\qquad
    \text{in}\;\Omega, \vspace{0cm}\\
   v^\varepsilon(x) = u(x) &\qquad
    \text{on}\;\partial\Omega.
    \end{cases} 
\end{equation}
We know that $|v^\varepsilon - u|\leq C\sqrt{\varepsilon}$, thus the goal is to compare $u^\varepsilon$ and $v^\varepsilon$. Let $w^\varepsilon(x) = u^\varepsilon(x) - v^\varepsilon(x)$, we have
\begin{equation}\label{eq:wPDEeps}
    \begin{cases}
w^\varepsilon(x) + |Du^\varepsilon(x)|^p  - |Dv^\varepsilon|^p - \varepsilon \Delta w^\varepsilon(x) = 0 &\qquad
    \text{in}\;\Omega, \vspace{0cm}\\
   w^\varepsilon(x) = +\infty &\qquad
    \text{on}\;\partial\Omega.
    \end{cases} 
\end{equation}
The good thing about \eqref{eq:wPDEeps} is, the data $f$ disappears and furthermorethe new data, for example with $p=2$ can be written as
\begin{equation}\label{eq:wwPDEeps}
    \begin{cases}
w^\varepsilon(x) + |Dw^\varepsilon(x)|^2 +\underbrace{2Dw^\varepsilon(x)\cdot Dv^\varepsilon(x)}_{g
(x)} - \varepsilon \Delta w^\varepsilon(x) = 0 &\qquad
    \text{in}\;\Omega, \vspace{0cm}\\
   w^\varepsilon(x) = +\infty &\qquad
    \text{on}\;\partial\Omega.
    \end{cases} 
\end{equation}
and $g(x) \approx \frac{\varepsilon \nabla d(x)\cdot \nabla v^\varepsilon(x)}{d(x)} + |Dv^\varepsilon(x)|^2$, which is small in some sense depends on $\varepsilon$. Question: can we get a rate of convergence $w^\varepsilon(x)\to 0$? What is the behavior of $Dv^\varepsilon(x)$ as $\varepsilon\to 0$? Can we know more besides $|v^\varepsilon - u|\leq C\sqrt{\varepsilon}$?
\item Building a better supersolution for compactly supported data. Using the above $v^\varepsilon$ in \eqref{eq:vPDEeps}, can we somehow show
\begin{equation*}
    u^\varepsilon(x) \leq v^\varepsilon(x) + \frac{\nu C_\alpha \varepsilon^{\alpha+1}}{d(x)^\alpha} 
\end{equation*}
in $\Omega$ when we assume $\mathrm{supp}\;f \subset \Omega_{\delta_0}$? 
\item Showing $u^\varepsilon$ is very small in the region where $f = 0$, for example $\mathrm{supp}\;f \subset \Omega_{\delta_0}$, using stochastic control. The idea behind this is, $u(x) = 0$ whenever $f(x) = 0$ (we assume $f\geq 0$) and $u^\varepsilon\to u$, thus $u^\varepsilon$ has to be very small where $f = 0$, thus there is a hope to build a better supersolution, to show that 
\begin{equation*}
       u^\varepsilon(x) \leq \frac{\nu C_\alpha \varepsilon^{\alpha+1}}{d(x)^\alpha} 
\end{equation*}
where $x$ in the region where $f(x) = 0$. Specifically, solution $u^\varepsilon$ to \eqref{eq:PDEeps} can be written as
\begin{equation*}
    u^\varepsilon(x) = \inf_{a(\cdot)\in \mathcal{A}}  \mathbb{E}\left[\int_0^\infty e^{-t}\left(\frac{(\alpha+1)^{\alpha+1}}{(\alpha+2)^{\alpha+2}}|a(X_t)|^{\alpha+2} + f(X_t)\right)dt \right]
\end{equation*}
where $q = \alpha+2$ satisfies $\frac{1}{q} + \frac{1}{p} = 1$, where
\begin{equation*}
    \begin{cases}
    dX_t = a(X_t)dt + \varepsilon\sqrt{2} d\mathbb{B}_t\\
    \;\;X_0 = x
    \end{cases}
\end{equation*}
and $\mathcal{A}$ is the set of continuous function $a(\cdot):\Omega \to \mathbb{R}$ such that the state $X_t\in \overline{\Omega}$ with probability 1 for all time $t\geq 0$ and for all initial points $x\in \Omega$. Here $\mathbb{B}_t$ is the standard Brownian motion. The hope is that, the optimal Markovian control is given by
\begin{equation*}
    a_0(x) = p|\nabla u^\varepsilon(x)|^{p-2}\nabla u^\varepsilon(x).
\end{equation*}
Hence, note that
\begin{equation*}
    \frac{1}{qp^{\alpha+1}} = \frac{(\alpha+1)^{\alpha+1}}{(\alpha+2)^{\alpha+2}}
\end{equation*}
we have
\begin{equation*}
   \frac{1}{qp^{\alpha+1}} |a_0(X_t)|^q =  \frac{1}{qp^{\alpha+1}}p^q |\nabla u^\varepsilon(x)|^p = \frac{1}{pq}|\nabla u^\varepsilon(x)|^p \approx \frac{1}{pq}\frac{(C_\alpha \alpha)^p\varepsilon^{\alpha+2}}{d(x)^{\alpha+2}}.
\end{equation*}
In fact, this estimate can be make rigorously from \cite{alessio_asymptotic_2006}, that is there is a constant $C>0$ such that
\begin{equation*}
    |\nabla u ^\varepsilon(x)|\leq \frac{C\varepsilon^{\alpha+1}}{d(x)^{\alpha+1}}
\end{equation*}
for $x\in \Omega$.
\end{enumerate}




%\bibliography{zzzzlibrary}{}
%\bibliographystyle{ieeetr}
%\bibliographystyle{acm}
\end{document}

    
    
    
    